% Options for packages loaded elsewhere
\PassOptionsToPackage{unicode}{hyperref}
\PassOptionsToPackage{hyphens}{url}
%
\documentclass[
  a4paper,
]{article}
\usepackage{amsmath,amssymb}
\usepackage{setspace}
\usepackage{iftex}
\ifPDFTeX
  \usepackage[T1]{fontenc}
  \usepackage[utf8]{inputenc}
  \usepackage{textcomp} % provide euro and other symbols
\else % if luatex or xetex
  \usepackage{unicode-math} % this also loads fontspec
  \defaultfontfeatures{Scale=MatchLowercase}
  \defaultfontfeatures[\rmfamily]{Ligatures=TeX,Scale=1}
\fi
\usepackage{lmodern}
\ifPDFTeX\else
  % xetex/luatex font selection
\fi
% Use upquote if available, for straight quotes in verbatim environments
\IfFileExists{upquote.sty}{\usepackage{upquote}}{}
\IfFileExists{microtype.sty}{% use microtype if available
  \usepackage[]{microtype}
  \UseMicrotypeSet[protrusion]{basicmath} % disable protrusion for tt fonts
}{}
\makeatletter
\@ifundefined{KOMAClassName}{% if non-KOMA class
  \IfFileExists{parskip.sty}{%
    \usepackage{parskip}
  }{% else
    \setlength{\parindent}{0pt}
    \setlength{\parskip}{6pt plus 2pt minus 1pt}}
}{% if KOMA class
  \KOMAoptions{parskip=half}}
\makeatother
\usepackage{xcolor}
\usepackage[margin=1in]{geometry}
\usepackage{color}
\usepackage{fancyvrb}
\newcommand{\VerbBar}{|}
\newcommand{\VERB}{\Verb[commandchars=\\\{\}]}
\DefineVerbatimEnvironment{Highlighting}{Verbatim}{commandchars=\\\{\}}
% Add ',fontsize=\small' for more characters per line
\usepackage{framed}
\definecolor{shadecolor}{RGB}{248,248,248}
\newenvironment{Shaded}{\begin{snugshade}}{\end{snugshade}}
\newcommand{\AlertTok}[1]{\textcolor[rgb]{0.94,0.16,0.16}{#1}}
\newcommand{\AnnotationTok}[1]{\textcolor[rgb]{0.56,0.35,0.01}{\textbf{\textit{#1}}}}
\newcommand{\AttributeTok}[1]{\textcolor[rgb]{0.13,0.29,0.53}{#1}}
\newcommand{\BaseNTok}[1]{\textcolor[rgb]{0.00,0.00,0.81}{#1}}
\newcommand{\BuiltInTok}[1]{#1}
\newcommand{\CharTok}[1]{\textcolor[rgb]{0.31,0.60,0.02}{#1}}
\newcommand{\CommentTok}[1]{\textcolor[rgb]{0.56,0.35,0.01}{\textit{#1}}}
\newcommand{\CommentVarTok}[1]{\textcolor[rgb]{0.56,0.35,0.01}{\textbf{\textit{#1}}}}
\newcommand{\ConstantTok}[1]{\textcolor[rgb]{0.56,0.35,0.01}{#1}}
\newcommand{\ControlFlowTok}[1]{\textcolor[rgb]{0.13,0.29,0.53}{\textbf{#1}}}
\newcommand{\DataTypeTok}[1]{\textcolor[rgb]{0.13,0.29,0.53}{#1}}
\newcommand{\DecValTok}[1]{\textcolor[rgb]{0.00,0.00,0.81}{#1}}
\newcommand{\DocumentationTok}[1]{\textcolor[rgb]{0.56,0.35,0.01}{\textbf{\textit{#1}}}}
\newcommand{\ErrorTok}[1]{\textcolor[rgb]{0.64,0.00,0.00}{\textbf{#1}}}
\newcommand{\ExtensionTok}[1]{#1}
\newcommand{\FloatTok}[1]{\textcolor[rgb]{0.00,0.00,0.81}{#1}}
\newcommand{\FunctionTok}[1]{\textcolor[rgb]{0.13,0.29,0.53}{\textbf{#1}}}
\newcommand{\ImportTok}[1]{#1}
\newcommand{\InformationTok}[1]{\textcolor[rgb]{0.56,0.35,0.01}{\textbf{\textit{#1}}}}
\newcommand{\KeywordTok}[1]{\textcolor[rgb]{0.13,0.29,0.53}{\textbf{#1}}}
\newcommand{\NormalTok}[1]{#1}
\newcommand{\OperatorTok}[1]{\textcolor[rgb]{0.81,0.36,0.00}{\textbf{#1}}}
\newcommand{\OtherTok}[1]{\textcolor[rgb]{0.56,0.35,0.01}{#1}}
\newcommand{\PreprocessorTok}[1]{\textcolor[rgb]{0.56,0.35,0.01}{\textit{#1}}}
\newcommand{\RegionMarkerTok}[1]{#1}
\newcommand{\SpecialCharTok}[1]{\textcolor[rgb]{0.81,0.36,0.00}{\textbf{#1}}}
\newcommand{\SpecialStringTok}[1]{\textcolor[rgb]{0.31,0.60,0.02}{#1}}
\newcommand{\StringTok}[1]{\textcolor[rgb]{0.31,0.60,0.02}{#1}}
\newcommand{\VariableTok}[1]{\textcolor[rgb]{0.00,0.00,0.00}{#1}}
\newcommand{\VerbatimStringTok}[1]{\textcolor[rgb]{0.31,0.60,0.02}{#1}}
\newcommand{\WarningTok}[1]{\textcolor[rgb]{0.56,0.35,0.01}{\textbf{\textit{#1}}}}
\usepackage{graphicx}
\makeatletter
\def\maxwidth{\ifdim\Gin@nat@width>\linewidth\linewidth\else\Gin@nat@width\fi}
\def\maxheight{\ifdim\Gin@nat@height>\textheight\textheight\else\Gin@nat@height\fi}
\makeatother
% Scale images if necessary, so that they will not overflow the page
% margins by default, and it is still possible to overwrite the defaults
% using explicit options in \includegraphics[width, height, ...]{}
\setkeys{Gin}{width=\maxwidth,height=\maxheight,keepaspectratio}
% Set default figure placement to htbp
\makeatletter
\def\fps@figure{htbp}
\makeatother
\setlength{\emergencystretch}{3em} % prevent overfull lines
\providecommand{\tightlist}{%
  \setlength{\itemsep}{0pt}\setlength{\parskip}{0pt}}
\setcounter{secnumdepth}{-\maxdimen} % remove section numbering
\ifLuaTeX
\usepackage[bidi=basic]{babel}
\else
\usepackage[bidi=default]{babel}
\fi
\babelprovide[main,import]{catalan}
% get rid of language-specific shorthands (see #6817):
\let\LanguageShortHands\languageshorthands
\def\languageshorthands#1{}
\ifLuaTeX
  \usepackage{selnolig}  % disable illegal ligatures
\fi
\usepackage{bookmark}
\IfFileExists{xurl.sty}{\usepackage{xurl}}{} % add URL line breaks if available
\urlstyle{same}
\hypersetup{
  pdftitle={U3. INFORMACIÓ DEL SISTEMA},
  pdfauthor={@tofermos 2024},
  pdflang={ca-ES},
  hidelinks,
  pdfcreator={LaTeX via pandoc}}

\title{U3. INFORMACIÓ DEL SISTEMA}
\author{@tofermos 2024}
\date{}

\begin{document}
\maketitle

{
\setcounter{tocdepth}{2}
\tableofcontents
}
\setstretch{1.5}
\newpage
\renewcommand\tablename{Tabla}

\section{1 Software de sistema}\label{software-de-sistema}

\subsection{1.1. Versió del sistema operatiu Linux
(Ubuntu)}\label{versiuxf3-del-sistema-operatiu-linux-ubuntu}

Per conèixer la versió d'Ubuntu que estàs utilitzant, utilitza una de
les següents ordres:

\begin{Shaded}
\begin{Highlighting}[]
\ExtensionTok{lsb\_release} \AttributeTok{{-}a}
\end{Highlighting}
\end{Shaded}

O també:

\begin{Shaded}
\begin{Highlighting}[]
\FunctionTok{cat}\NormalTok{ /etc/os{-}release}
\end{Highlighting}
\end{Shaded}

\subsection{1.2. Versió del nucli
(Kernel)}\label{versiuxf3-del-nucli-kernel}

Per veure la versió del nucli de Linux, pots utilitzar l'ordre:

\begin{Shaded}
\begin{Highlighting}[]
\FunctionTok{uname} \AttributeTok{{-}r}
\end{Highlighting}
\end{Shaded}

\subsection{1.3. Versió de l'entorn
d'escriptori}\label{versiuxf3-de-lentorn-descriptori}

Depenent de l'entorn d'escriptori que utilitzes, pots utilitzar aquestes
ordres per veure'n la versió:

\begin{itemize}
\item
  Gnome:

\begin{Shaded}
\begin{Highlighting}[]
\ExtensionTok{gnome{-}shell} \AttributeTok{{-}{-}version}
\end{Highlighting}
\end{Shaded}
\item
  KDE Plasma:

\begin{Shaded}
\begin{Highlighting}[]
\ExtensionTok{plasmashell} \AttributeTok{{-}{-}version}
\end{Highlighting}
\end{Shaded}
\item
  LXQt:

\begin{Shaded}
\begin{Highlighting}[]
\ExtensionTok{lxqt{-}session} \AttributeTok{{-}{-}version}
\end{Highlighting}
\end{Shaded}
\item
  Mate:

\begin{Shaded}
\begin{Highlighting}[]
\ExtensionTok{mate{-}about} \AttributeTok{{-}{-}version}
\end{Highlighting}
\end{Shaded}
\item
  Cinnamon (Mint):

\begin{Shaded}
\begin{Highlighting}[]
\ExtensionTok{cinnamon} \AttributeTok{{-}{-}version}
\end{Highlighting}
\end{Shaded}
\end{itemize}

\section{2 Hardware}\label{hardware}

\subsection{2.1. CPU (Processador)}\label{cpu-processador}

Per veure informació sobre la CPU, pots utilitzar:

\begin{Shaded}
\begin{Highlighting}[]
\ExtensionTok{lscpu}
\end{Highlighting}
\end{Shaded}

\subsection{2.2. Memòria RAM i Swap}\label{memuxf2ria-ram-i-swap}

Per veure l'estat de la memòria RAM i la Swap:

\begin{Shaded}
\begin{Highlighting}[]
\FunctionTok{free} \AttributeTok{{-}h}
\end{Highlighting}
\end{Shaded}

\subsection{2.3. Memòria secundària (discos
durs)}\label{memuxf2ria-secunduxe0ria-discos-durs}

Per veure informació sobre els discos durs i les particions, utilitza:

\begin{Shaded}
\begin{Highlighting}[]
\ExtensionTok{lsblk}
\end{Highlighting}
\end{Shaded}

O bé:

\begin{Shaded}
\begin{Highlighting}[]
\FunctionTok{df} \AttributeTok{{-}h}
\end{Highlighting}
\end{Shaded}

\subsection{2.4. Dispositius USB}\label{dispositius-usb}

Per veure els dispositius USB connectats al sistema:

\begin{Shaded}
\begin{Highlighting}[]
\ExtensionTok{lsusb}
\end{Highlighting}
\end{Shaded}

\subsection{2.5. Monitor}\label{monitor}

Per veure informació sobre el monitor, com la resolució o configuració
de la pantalla:

\begin{Shaded}
\begin{Highlighting}[]
\ExtensionTok{xrandr}
\end{Highlighting}
\end{Shaded}

\subsection{2.6. NIC (Targeta de xarxa)}\label{nic-targeta-de-xarxa}

Per obtenir informació sobre la targeta de xarxa (NIC):

\begin{Shaded}
\begin{Highlighting}[]
\ExtensionTok{ip}\NormalTok{ a}
\end{Highlighting}
\end{Shaded}

O bé:

\begin{Shaded}
\begin{Highlighting}[]
\ExtensionTok{lshw} \AttributeTok{{-}C}\NormalTok{ network}
\end{Highlighting}
\end{Shaded}

\section{3 Configuració}\label{configuraciuxf3}

\subsection{3.1. Idioma, teclat i regió}\label{idioma-teclat-i-regiuxf3}

Per veure informació sobre la configuració d'idioma i regió, pots
utilitzar:

\begin{Shaded}
\begin{Highlighting}[]
\ExtensionTok{localectl}
\end{Highlighting}
\end{Shaded}

Per veure la configuració del teclat:

\begin{Shaded}
\begin{Highlighting}[]
\ExtensionTok{setxkbmap} \AttributeTok{{-}print} \AttributeTok{{-}verbose}\NormalTok{ 10}
\end{Highlighting}
\end{Shaded}


\end{document}
