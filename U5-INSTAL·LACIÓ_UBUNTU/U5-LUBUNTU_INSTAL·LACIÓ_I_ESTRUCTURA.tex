% Options for packages loaded elsewhere
\PassOptionsToPackage{unicode}{hyperref}
\PassOptionsToPackage{hyphens}{url}
%
\documentclass[
  a4paper,
]{article}
\usepackage{amsmath,amssymb}
\usepackage{setspace}
\usepackage{iftex}
\ifPDFTeX
  \usepackage[T1]{fontenc}
  \usepackage[utf8]{inputenc}
  \usepackage{textcomp} % provide euro and other symbols
\else % if luatex or xetex
  \usepackage{unicode-math} % this also loads fontspec
  \defaultfontfeatures{Scale=MatchLowercase}
  \defaultfontfeatures[\rmfamily]{Ligatures=TeX,Scale=1}
\fi
\usepackage{lmodern}
\ifPDFTeX\else
  % xetex/luatex font selection
\fi
% Use upquote if available, for straight quotes in verbatim environments
\IfFileExists{upquote.sty}{\usepackage{upquote}}{}
\IfFileExists{microtype.sty}{% use microtype if available
  \usepackage[]{microtype}
  \UseMicrotypeSet[protrusion]{basicmath} % disable protrusion for tt fonts
}{}
\makeatletter
\@ifundefined{KOMAClassName}{% if non-KOMA class
  \IfFileExists{parskip.sty}{%
    \usepackage{parskip}
  }{% else
    \setlength{\parindent}{0pt}
    \setlength{\parskip}{6pt plus 2pt minus 1pt}}
}{% if KOMA class
  \KOMAoptions{parskip=half}}
\makeatother
\usepackage{xcolor}
\usepackage[margin=1in]{geometry}
\usepackage{color}
\usepackage{fancyvrb}
\newcommand{\VerbBar}{|}
\newcommand{\VERB}{\Verb[commandchars=\\\{\}]}
\DefineVerbatimEnvironment{Highlighting}{Verbatim}{commandchars=\\\{\}}
% Add ',fontsize=\small' for more characters per line
\usepackage{framed}
\definecolor{shadecolor}{RGB}{248,248,248}
\newenvironment{Shaded}{\begin{snugshade}}{\end{snugshade}}
\newcommand{\AlertTok}[1]{\textcolor[rgb]{0.94,0.16,0.16}{#1}}
\newcommand{\AnnotationTok}[1]{\textcolor[rgb]{0.56,0.35,0.01}{\textbf{\textit{#1}}}}
\newcommand{\AttributeTok}[1]{\textcolor[rgb]{0.13,0.29,0.53}{#1}}
\newcommand{\BaseNTok}[1]{\textcolor[rgb]{0.00,0.00,0.81}{#1}}
\newcommand{\BuiltInTok}[1]{#1}
\newcommand{\CharTok}[1]{\textcolor[rgb]{0.31,0.60,0.02}{#1}}
\newcommand{\CommentTok}[1]{\textcolor[rgb]{0.56,0.35,0.01}{\textit{#1}}}
\newcommand{\CommentVarTok}[1]{\textcolor[rgb]{0.56,0.35,0.01}{\textbf{\textit{#1}}}}
\newcommand{\ConstantTok}[1]{\textcolor[rgb]{0.56,0.35,0.01}{#1}}
\newcommand{\ControlFlowTok}[1]{\textcolor[rgb]{0.13,0.29,0.53}{\textbf{#1}}}
\newcommand{\DataTypeTok}[1]{\textcolor[rgb]{0.13,0.29,0.53}{#1}}
\newcommand{\DecValTok}[1]{\textcolor[rgb]{0.00,0.00,0.81}{#1}}
\newcommand{\DocumentationTok}[1]{\textcolor[rgb]{0.56,0.35,0.01}{\textbf{\textit{#1}}}}
\newcommand{\ErrorTok}[1]{\textcolor[rgb]{0.64,0.00,0.00}{\textbf{#1}}}
\newcommand{\ExtensionTok}[1]{#1}
\newcommand{\FloatTok}[1]{\textcolor[rgb]{0.00,0.00,0.81}{#1}}
\newcommand{\FunctionTok}[1]{\textcolor[rgb]{0.13,0.29,0.53}{\textbf{#1}}}
\newcommand{\ImportTok}[1]{#1}
\newcommand{\InformationTok}[1]{\textcolor[rgb]{0.56,0.35,0.01}{\textbf{\textit{#1}}}}
\newcommand{\KeywordTok}[1]{\textcolor[rgb]{0.13,0.29,0.53}{\textbf{#1}}}
\newcommand{\NormalTok}[1]{#1}
\newcommand{\OperatorTok}[1]{\textcolor[rgb]{0.81,0.36,0.00}{\textbf{#1}}}
\newcommand{\OtherTok}[1]{\textcolor[rgb]{0.56,0.35,0.01}{#1}}
\newcommand{\PreprocessorTok}[1]{\textcolor[rgb]{0.56,0.35,0.01}{\textit{#1}}}
\newcommand{\RegionMarkerTok}[1]{#1}
\newcommand{\SpecialCharTok}[1]{\textcolor[rgb]{0.81,0.36,0.00}{\textbf{#1}}}
\newcommand{\SpecialStringTok}[1]{\textcolor[rgb]{0.31,0.60,0.02}{#1}}
\newcommand{\StringTok}[1]{\textcolor[rgb]{0.31,0.60,0.02}{#1}}
\newcommand{\VariableTok}[1]{\textcolor[rgb]{0.00,0.00,0.00}{#1}}
\newcommand{\VerbatimStringTok}[1]{\textcolor[rgb]{0.31,0.60,0.02}{#1}}
\newcommand{\WarningTok}[1]{\textcolor[rgb]{0.56,0.35,0.01}{\textbf{\textit{#1}}}}
\usepackage{longtable,booktabs,array}
\usepackage{calc} % for calculating minipage widths
% Correct order of tables after \paragraph or \subparagraph
\usepackage{etoolbox}
\makeatletter
\patchcmd\longtable{\par}{\if@noskipsec\mbox{}\fi\par}{}{}
\makeatother
% Allow footnotes in longtable head/foot
\IfFileExists{footnotehyper.sty}{\usepackage{footnotehyper}}{\usepackage{footnote}}
\makesavenoteenv{longtable}
\usepackage{graphicx}
\makeatletter
\def\maxwidth{\ifdim\Gin@nat@width>\linewidth\linewidth\else\Gin@nat@width\fi}
\def\maxheight{\ifdim\Gin@nat@height>\textheight\textheight\else\Gin@nat@height\fi}
\makeatother
% Scale images if necessary, so that they will not overflow the page
% margins by default, and it is still possible to overwrite the defaults
% using explicit options in \includegraphics[width, height, ...]{}
\setkeys{Gin}{width=\maxwidth,height=\maxheight,keepaspectratio}
% Set default figure placement to htbp
\makeatletter
\def\fps@figure{htbp}
\makeatother
\setlength{\emergencystretch}{3em} % prevent overfull lines
\providecommand{\tightlist}{%
  \setlength{\itemsep}{0pt}\setlength{\parskip}{0pt}}
\setcounter{secnumdepth}{-\maxdimen} % remove section numbering
\ifLuaTeX
\usepackage[bidi=basic]{babel}
\else
\usepackage[bidi=default]{babel}
\fi
\babelprovide[main,import]{catalan}
% get rid of language-specific shorthands (see #6817):
\let\LanguageShortHands\languageshorthands
\def\languageshorthands#1{}
\ifLuaTeX
  \usepackage{selnolig}  % disable illegal ligatures
\fi
\usepackage{bookmark}
\IfFileExists{xurl.sty}{\usepackage{xurl}}{} % add URL line breaks if available
\urlstyle{same}
\hypersetup{
  pdftitle={U5. LUBUNTU. INSTAL·LACIÓ I ESTRUCTURA},
  pdfauthor={@tofermos 2024},
  pdflang={ca-ES},
  hidelinks,
  pdfcreator={LaTeX via pandoc}}

\title{U5. LUBUNTU. INSTAL·LACIÓ I ESTRUCTURA}
\author{@tofermos 2024}
\date{}

\begin{document}
\maketitle

{
\setcounter{tocdepth}{2}
\tableofcontents
}
\setstretch{1.5}
\newpage
\renewcommand\tablename{Tabla}

\section{1. El sistema operatiu GNU/Linux. Estructura i
característiques}\label{el-sistema-operatiu-gnulinux.-estructura-i-caracteruxedstiques}

\subsection{1.1 El sistema operatiu
GNU/Linux}\label{el-sistema-operatiu-gnulinux}

\begin{itemize}
\tightlist
\item
  \textbf{GNU/Linux} és un sistema operatiu lliure i de codi obert.
\item
  Basat en el nucli \textbf{Linux} i eines del projecte \textbf{GNU}.
\item
  Escalable: pot ser utilitzat en dispositius menuts o grans servidors.
\end{itemize}

\subsection{1.2 Estructura}\label{estructura}

\begin{enumerate}
\def\labelenumi{\arabic{enumi}.}
\tightlist
\item
  \textbf{Nucli (Kernel)}:

  \begin{itemize}
  \item
    Gestió de recursos: memòria, CPU, dispositius.
  \item
    Per veure la versió:

\begin{Shaded}
\begin{Highlighting}[]
\FunctionTok{uname} \AttributeTok{{-}r}
\end{Highlighting}
\end{Shaded}
  \item
    Exemple de resultat: \texttt{5.15.0-83-generic}.
  \end{itemize}
\item
  \textbf{Shell}:

  \begin{itemize}
  \tightlist
  \item
    Interfície que permet interactuar amb el sistema.
  \item
    Exemples: \textbf{bash}, \textbf{sh}, \textbf{dash}.
  \end{itemize}
\item
  \textbf{Utilitats}:

  \begin{itemize}
  \tightlist
  \item
    Comandes bàsiques: \texttt{ls}, \texttt{cp}, \texttt{mv}.
  \end{itemize}
\item
  \textbf{Aplicacions}:

  \begin{itemize}
  \tightlist
  \item
    Programes per a usuaris finals, com navegadors o editors.
  \end{itemize}
\item
  \textbf{Sistema d'arxius}:

  \begin{itemize}
  \tightlist
  \item
    Estructura jeràrquica que comença a \texttt{/}.
  \end{itemize}
\end{enumerate}

\section{1.3 Característiques}\label{caracteruxedstiques}

\begin{itemize}
\tightlist
\item
  \textbf{Lliure i obert}: Es pot modificar i distribuir.
\item
  \textbf{Multiusuari}: Diversos usuaris alhora.
\item
  \textbf{Multitasca}: Execució de múltiples processos.
\item
  \textbf{Segur}: Model de permisos.
\item
  \textbf{Portabilitat}: Funciona en diferents plataformes.
\end{itemize}

\section{2. Evolució històrica}\label{evoluciuxf3-histuxf2rica}

\begin{itemize}
\tightlist
\item
  \textbf{1991}: Linus Torvalds crea el nucli Linux.
\item
  \textbf{1992}: Es llança com a programari lliure.
\item
  \textbf{1994}: Primera versió estable (Linux 1.0).
\item
  \textbf{Actualitat}: Moltes distribucions basades en Linux.
\end{itemize}

\section{3. Distribucions}\label{distribucions}

\begin{itemize}
\item
  Una distribució és una combinació del nucli, utilitats GNU, i altres
  programes.
\item
  Exemples de distribucions:

  \begin{longtable}[]{@{}ll@{}}
  \toprule\noalign{}
  \textbf{Distribució} & \textbf{Orientació} \\
  \midrule\noalign{}
  \endhead
  \bottomrule\noalign{}
  \endlastfoot
  Ubuntu & Usuaris nous, suport comunitari. \\
  Debian & Estable, per servidors. \\
  Arch Linux & Personalitzable, usuaris experts. \\
  \end{longtable}
\end{itemize}

\section{4. Instal·lació de Ubuntu}\label{installaciuxf3-de-ubuntu}

\begin{enumerate}
\def\labelenumi{\arabic{enumi}.}
\tightlist
\item
  Descarrega la imatge ISO des d'\href{https://ubuntu.com/}{ubuntu.com}.
\item
  Grava-la en un USB amb eines com \textbf{Rufus} o \textbf{Etcher}.
\item
  Arranca des del USB i segueix el procés d'instal·lació:

  \begin{itemize}
  \tightlist
  \item
    Idioma, fus horari, particions.
  \end{itemize}
\item
  Inicia sessió després de la instal·lació.
\end{enumerate}

\section{5. Entorn gràfic}\label{entorn-gruxe0fic}

\subsection{5.1 Característiques dels gestors de finestres GNOME i
KDE}\label{caracteruxedstiques-dels-gestors-de-finestres-gnome-i-kde}

\begin{longtable}[]{@{}
  >{\raggedright\arraybackslash}p{(\columnwidth - 4\tabcolsep) * \real{0.2875}}
  >{\raggedright\arraybackslash}p{(\columnwidth - 4\tabcolsep) * \real{0.3500}}
  >{\raggedright\arraybackslash}p{(\columnwidth - 4\tabcolsep) * \real{0.3625}}@{}}
\toprule\noalign{}
\begin{minipage}[b]{\linewidth}\raggedright
\textbf{Característica}
\end{minipage} & \begin{minipage}[b]{\linewidth}\raggedright
\textbf{GNOME}
\end{minipage} & \begin{minipage}[b]{\linewidth}\raggedright
\textbf{KDE}
\end{minipage} \\
\midrule\noalign{}
\endhead
\bottomrule\noalign{}
\endlastfoot
\textbf{Disseny} & Minimalista i senzill & Personalitzable \\
\textbf{Consum de recursos} & Relatiu alt & Relatiu moderat \\
\textbf{Eines pròpies} & Nautilus, GNOME Terminal & Dolphin, Konsole \\
\end{longtable}

\subsection{5.2 Altres gestors de
finestres}\label{altres-gestors-de-finestres}

\begin{longtable}[]{@{}ll@{}}
\toprule\noalign{}
\textbf{Gestor} & \textbf{Característica} \\
\midrule\noalign{}
\endhead
\bottomrule\noalign{}
\endlastfoot
\textbf{Fluxbox} & Lleuger i ràpid, minimalista. \\
\textbf{LXDE} & Lleuger, ideal per a equips antics. \\
\textbf{Xfce} & Equilibri entre lleugeresa i funcionalitat. \\
\end{longtable}

\subsection{5.3 Personalització de
l'escriptori}\label{personalitzaciuxf3-de-lescriptori}

\begin{itemize}
\tightlist
\item
  \textbf{Canviar fons de pantalla}:

  \begin{itemize}
  \tightlist
  \item
    Botó dret a l'escriptori → \textbf{Configuració del fons}.
  \end{itemize}
\item
  \textbf{Afegir ginys}:

  \begin{itemize}
  \tightlist
  \item
    Menú → \textbf{Ginys} o \textbf{Accessoris}.
  \end{itemize}
\end{itemize}

\subsection{5.4 Aplicacions}\label{aplicacions}

\begin{itemize}
\tightlist
\item
  Navegador web: \textbf{Firefox}.
\item
  Suite ofimàtica: \textbf{LibreOffice}.
\item
  Reproductor multimèdia: \textbf{VLC}.
\end{itemize}

\subsection{5.5 Llocs}\label{llocs}

\subsubsection{Exploradors de fitxers}\label{exploradors-de-fitxers}

Un explorador de fitxers és l'eina gràfica que permet navegar pel
sistema d'arxius. Cada entorn gràfic sol utilitzar un explorador
diferent.

\begin{longtable}[]{@{}
  >{\raggedright\arraybackslash}p{(\columnwidth - 6\tabcolsep) * \real{0.1301}}
  >{\raggedright\arraybackslash}p{(\columnwidth - 6\tabcolsep) * \real{0.1789}}
  >{\raggedright\arraybackslash}p{(\columnwidth - 6\tabcolsep) * \real{0.3415}}
  >{\raggedright\arraybackslash}p{(\columnwidth - 6\tabcolsep) * \real{0.3496}}@{}}
\toprule\noalign{}
\begin{minipage}[b]{\linewidth}\raggedright
\textbf{Explorador}
\end{minipage} & \begin{minipage}[b]{\linewidth}\raggedright
\textbf{Entorn associat}
\end{minipage} & \begin{minipage}[b]{\linewidth}\raggedright
\textbf{Característiques}
\end{minipage} & \begin{minipage}[b]{\linewidth}\raggedright
\textbf{Com instal·lar-lo}
\end{minipage} \\
\midrule\noalign{}
\endhead
\bottomrule\noalign{}
\endlastfoot
\textbf{Nautilus} & GNOME & Senzill i integrat amb GNOME. &
\texttt{sudo\ apt\ install\ nautilus} \\
\textbf{Thunar} & Xfce & Lleuger i ràpid. &
\texttt{sudo\ apt\ install\ thunar} \\
\textbf{Dolphin} & KDE & Altament personalitzable i complet. &
\texttt{sudo\ apt\ install\ dolphin} \\
\textbf{Caja} & MATE & Similar a Nautilus, però amb opcions clàssiques.
& \texttt{sudo\ apt\ install\ caja} \\
\textbf{PCManFM} & LXDE & Lleuger i senzill. &
\texttt{sudo\ apt\ install\ pcmanfm} \\
\end{longtable}

\subsubsection{Seleccionar i canviar
d'explorador}\label{seleccionar-i-canviar-dexplorador}

Pots instal·lar diversos exploradors i executar-los independentment, per
exemple:

\begin{Shaded}
\begin{Highlighting}[]
\ExtensionTok{nautilus} \KeywordTok{\&}
\ExtensionTok{thunar} \KeywordTok{\&}
\end{Highlighting}
\end{Shaded}

\begin{itemize}
\tightlist
\item
  Configura el predeterminat editant la configuració del sistema o
  utilitzant eines com \textbf{xdg-mime}.
\end{itemize}

\subsubsection{Desinstal·lació
d'exploradors}\label{desinstallaciuxf3-dexploradors}

Si desitges eliminar un explorador de fitxers:

\begin{Shaded}
\begin{Highlighting}[]
\FunctionTok{sudo}\NormalTok{ apt remove }\OperatorTok{\textless{}}\NormalTok{nom\_explorador}\OperatorTok{\textgreater{}}
\FunctionTok{sudo}\NormalTok{ apt autoremove}
\end{Highlighting}
\end{Shaded}

Per exemple, per eliminar \textbf{Caja}:

\begin{Shaded}
\begin{Highlighting}[]
\FunctionTok{sudo}\NormalTok{ apt remove caja}
\FunctionTok{sudo}\NormalTok{ apt autoremove}
\end{Highlighting}
\end{Shaded}

\subsection{5.6 Preferències}\label{preferuxe8ncies}

\begin{itemize}
\item
  \textbf{Pantalla}: Resolució i configuració de monitors.
\item
  \textbf{Ratolí i teclat}: Velocitat, mapes de tecles.
\item
  Comanda per accedir:

\begin{Shaded}
\begin{Highlighting}[]
\ExtensionTok{lxappearance}
\end{Highlighting}
\end{Shaded}
\end{itemize}

\subsection{5.7 Administració}\label{administraciuxf3}

\begin{itemize}
\tightlist
\item
  Gestió d'usuaris:

  \begin{itemize}
  \item
    Afegir un usuari:

\begin{Shaded}
\begin{Highlighting}[]
\FunctionTok{sudo}\NormalTok{ adduser nom\_usuari}
\end{Highlighting}
\end{Shaded}
  \item
    Veure usuaris:

\begin{Shaded}
\begin{Highlighting}[]
\FunctionTok{cat}\NormalTok{ /etc/passwd}
\end{Highlighting}
\end{Shaded}
  \end{itemize}
\item
  Gestió de processos:

  \begin{itemize}
  \item
    Veure processos actius:

\begin{Shaded}
\begin{Highlighting}[]
\ExtensionTok{top}
\end{Highlighting}
\end{Shaded}
  \end{itemize}
\item
  Gestió de paquets:

  \begin{itemize}
  \item
    Actualitzar el sistema:

\begin{Shaded}
\begin{Highlighting}[]
\FunctionTok{sudo}\NormalTok{ apt update }\KeywordTok{\&\&} \FunctionTok{sudo}\NormalTok{ apt upgrade}
\end{Highlighting}
\end{Shaded}
  \end{itemize}
\end{itemize}

\subsection{5.8 Instal·lació i desinstal·lació de diferents
escriptoris}\label{installaciuxf3-i-desinstallaciuxf3-de-diferents-escriptoris}

\subsubsection{Instal·lació
d'escriptoris}\label{installaciuxf3-descriptoris}

Pots instal·lar escriptoris alternatius en Ubuntu utilitzant el gestor
de paquets \textbf{apt}. Després d'instal·lar-los, pots seleccionar quin
entorn utilitzar en la pantalla d'inici de sessió.

\begin{longtable}[]{@{}
  >{\raggedright\arraybackslash}p{(\columnwidth - 4\tabcolsep) * \real{0.1545}}
  >{\raggedright\arraybackslash}p{(\columnwidth - 4\tabcolsep) * \real{0.3909}}
  >{\raggedright\arraybackslash}p{(\columnwidth - 4\tabcolsep) * \real{0.4545}}@{}}
\toprule\noalign{}
\begin{minipage}[b]{\linewidth}\raggedright
\textbf{Escriptori}
\end{minipage} & \begin{minipage}[b]{\linewidth}\raggedright
\textbf{Com instal·lar-lo}
\end{minipage} & \begin{minipage}[b]{\linewidth}\raggedright
\textbf{Com accedir després d'instal·lar}
\end{minipage} \\
\midrule\noalign{}
\endhead
\bottomrule\noalign{}
\endlastfoot
\textbf{GNOME} & \texttt{sudo\ apt\ install\ ubuntu-desktop} & Tria
\textbf{GNOME} a la pantalla d'inici de sessió. \\
\textbf{LXDE} & \texttt{sudo\ apt\ install\ lubuntu-desktop} & Tria
\textbf{LXDE} a la pantalla d'inici de sessió. \\
\textbf{Xfce} & \texttt{sudo\ apt\ install\ xubuntu-desktop} & Tria
\textbf{Xfce} a la pantalla d'inici de sessió. \\
\textbf{KDE/Plasma} & \texttt{sudo\ apt\ install\ kubuntu-desktop} &
Tria \textbf{Plasma} a la pantalla d'inici de sessió. \\
\textbf{Fluxbox} & \texttt{sudo\ apt\ install\ fluxbox} & Tria
\textbf{Fluxbox} a la pantalla d'inici de sessió. \\
\end{longtable}

\subsubsection{Desinstal·lació
d'escriptoris}\label{desinstallaciuxf3-descriptoris}

Per desinstal·lar un escriptori, elimina els paquets associats.
Assegura't de no eliminar dependències crítiques del sistema.

\begin{enumerate}
\def\labelenumi{\arabic{enumi}.}
\item
  \textbf{GNOME}:

\begin{Shaded}
\begin{Highlighting}[]
\FunctionTok{sudo}\NormalTok{ apt remove ubuntu{-}desktop}
\FunctionTok{sudo}\NormalTok{ apt autoremove}
\end{Highlighting}
\end{Shaded}
\item
  \textbf{LXDE}:

\begin{Shaded}
\begin{Highlighting}[]
\FunctionTok{sudo}\NormalTok{ apt remove lubuntu{-}desktop}
\FunctionTok{sudo}\NormalTok{ apt autoremove}
\end{Highlighting}
\end{Shaded}
\item
  \textbf{Xfce}:

\begin{Shaded}
\begin{Highlighting}[]
\FunctionTok{sudo}\NormalTok{ apt remove xubuntu{-}desktop}
\FunctionTok{sudo}\NormalTok{ apt autoremove}
\end{Highlighting}
\end{Shaded}
\item
  \textbf{KDE/Plasma}:

\begin{Shaded}
\begin{Highlighting}[]
\FunctionTok{sudo}\NormalTok{ apt remove kubuntu{-}desktop}
\FunctionTok{sudo}\NormalTok{ apt autoremove}
\end{Highlighting}
\end{Shaded}
\item
  \textbf{Fluxbox}:

\begin{Shaded}
\begin{Highlighting}[]
\FunctionTok{sudo}\NormalTok{ apt remove fluxbox}
\FunctionTok{sudo}\NormalTok{ apt autoremove}
\end{Highlighting}
\end{Shaded}
\end{enumerate}

\subsubsection{Comprovar escriptoris
instal·lats}\label{comprovar-escriptoris-installats}

\begin{itemize}
\tightlist
\item
  Llista els paquets instal·lats relacionats amb els entorns
  d'escriptori
\end{itemize}

: \texttt{bash\ \ \ dpkg\ -l\ \textbar{}\ grep\ desktop}

\subsubsection{Selecció de l'escriptori
predeterminat}\label{selecciuxf3-de-lescriptori-predeterminat}

\begin{itemize}
\tightlist
\item
  Si vols canviar l'escriptori predeterminat:

  \begin{enumerate}
  \def\labelenumi{\arabic{enumi}.}
  \tightlist
  \item
    Modifica la configuració de la pantalla d'inici de sessió.
  \item
    O selecciona'l manualment al moment d'iniciar sessió. ```
  \end{enumerate}
\end{itemize}

Aquesta versió completada integra tot el que necessites sobre
instal·lació i desinstal·lació d'escriptoris i exploradors de fitxers,
amb els passos corresponents per a un entorn Ubuntu. També he inclòs
exemples per a cada acció amb les ordres necessàries.

\end{document}
