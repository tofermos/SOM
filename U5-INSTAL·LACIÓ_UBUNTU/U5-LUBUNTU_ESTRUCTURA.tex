% Options for packages loaded elsewhere
\PassOptionsToPackage{unicode}{hyperref}
\PassOptionsToPackage{hyphens}{url}
%
\documentclass[
  a4paper,
]{article}
\usepackage{amsmath,amssymb}
\usepackage{setspace}
\usepackage{iftex}
\ifPDFTeX
  \usepackage[T1]{fontenc}
  \usepackage[utf8]{inputenc}
  \usepackage{textcomp} % provide euro and other symbols
\else % if luatex or xetex
  \usepackage{unicode-math} % this also loads fontspec
  \defaultfontfeatures{Scale=MatchLowercase}
  \defaultfontfeatures[\rmfamily]{Ligatures=TeX,Scale=1}
\fi
\usepackage{lmodern}
\ifPDFTeX\else
  % xetex/luatex font selection
\fi
% Use upquote if available, for straight quotes in verbatim environments
\IfFileExists{upquote.sty}{\usepackage{upquote}}{}
\IfFileExists{microtype.sty}{% use microtype if available
  \usepackage[]{microtype}
  \UseMicrotypeSet[protrusion]{basicmath} % disable protrusion for tt fonts
}{}
\makeatletter
\@ifundefined{KOMAClassName}{% if non-KOMA class
  \IfFileExists{parskip.sty}{%
    \usepackage{parskip}
  }{% else
    \setlength{\parindent}{0pt}
    \setlength{\parskip}{6pt plus 2pt minus 1pt}}
}{% if KOMA class
  \KOMAoptions{parskip=half}}
\makeatother
\usepackage{xcolor}
\usepackage[margin=1in]{geometry}
\usepackage{color}
\usepackage{fancyvrb}
\newcommand{\VerbBar}{|}
\newcommand{\VERB}{\Verb[commandchars=\\\{\}]}
\DefineVerbatimEnvironment{Highlighting}{Verbatim}{commandchars=\\\{\}}
% Add ',fontsize=\small' for more characters per line
\usepackage{framed}
\definecolor{shadecolor}{RGB}{248,248,248}
\newenvironment{Shaded}{\begin{snugshade}}{\end{snugshade}}
\newcommand{\AlertTok}[1]{\textcolor[rgb]{0.94,0.16,0.16}{#1}}
\newcommand{\AnnotationTok}[1]{\textcolor[rgb]{0.56,0.35,0.01}{\textbf{\textit{#1}}}}
\newcommand{\AttributeTok}[1]{\textcolor[rgb]{0.13,0.29,0.53}{#1}}
\newcommand{\BaseNTok}[1]{\textcolor[rgb]{0.00,0.00,0.81}{#1}}
\newcommand{\BuiltInTok}[1]{#1}
\newcommand{\CharTok}[1]{\textcolor[rgb]{0.31,0.60,0.02}{#1}}
\newcommand{\CommentTok}[1]{\textcolor[rgb]{0.56,0.35,0.01}{\textit{#1}}}
\newcommand{\CommentVarTok}[1]{\textcolor[rgb]{0.56,0.35,0.01}{\textbf{\textit{#1}}}}
\newcommand{\ConstantTok}[1]{\textcolor[rgb]{0.56,0.35,0.01}{#1}}
\newcommand{\ControlFlowTok}[1]{\textcolor[rgb]{0.13,0.29,0.53}{\textbf{#1}}}
\newcommand{\DataTypeTok}[1]{\textcolor[rgb]{0.13,0.29,0.53}{#1}}
\newcommand{\DecValTok}[1]{\textcolor[rgb]{0.00,0.00,0.81}{#1}}
\newcommand{\DocumentationTok}[1]{\textcolor[rgb]{0.56,0.35,0.01}{\textbf{\textit{#1}}}}
\newcommand{\ErrorTok}[1]{\textcolor[rgb]{0.64,0.00,0.00}{\textbf{#1}}}
\newcommand{\ExtensionTok}[1]{#1}
\newcommand{\FloatTok}[1]{\textcolor[rgb]{0.00,0.00,0.81}{#1}}
\newcommand{\FunctionTok}[1]{\textcolor[rgb]{0.13,0.29,0.53}{\textbf{#1}}}
\newcommand{\ImportTok}[1]{#1}
\newcommand{\InformationTok}[1]{\textcolor[rgb]{0.56,0.35,0.01}{\textbf{\textit{#1}}}}
\newcommand{\KeywordTok}[1]{\textcolor[rgb]{0.13,0.29,0.53}{\textbf{#1}}}
\newcommand{\NormalTok}[1]{#1}
\newcommand{\OperatorTok}[1]{\textcolor[rgb]{0.81,0.36,0.00}{\textbf{#1}}}
\newcommand{\OtherTok}[1]{\textcolor[rgb]{0.56,0.35,0.01}{#1}}
\newcommand{\PreprocessorTok}[1]{\textcolor[rgb]{0.56,0.35,0.01}{\textit{#1}}}
\newcommand{\RegionMarkerTok}[1]{#1}
\newcommand{\SpecialCharTok}[1]{\textcolor[rgb]{0.81,0.36,0.00}{\textbf{#1}}}
\newcommand{\SpecialStringTok}[1]{\textcolor[rgb]{0.31,0.60,0.02}{#1}}
\newcommand{\StringTok}[1]{\textcolor[rgb]{0.31,0.60,0.02}{#1}}
\newcommand{\VariableTok}[1]{\textcolor[rgb]{0.00,0.00,0.00}{#1}}
\newcommand{\VerbatimStringTok}[1]{\textcolor[rgb]{0.31,0.60,0.02}{#1}}
\newcommand{\WarningTok}[1]{\textcolor[rgb]{0.56,0.35,0.01}{\textbf{\textit{#1}}}}
\usepackage{longtable,booktabs,array}
\usepackage{calc} % for calculating minipage widths
% Correct order of tables after \paragraph or \subparagraph
\usepackage{etoolbox}
\makeatletter
\patchcmd\longtable{\par}{\if@noskipsec\mbox{}\fi\par}{}{}
\makeatother
% Allow footnotes in longtable head/foot
\IfFileExists{footnotehyper.sty}{\usepackage{footnotehyper}}{\usepackage{footnote}}
\makesavenoteenv{longtable}
\usepackage{graphicx}
\makeatletter
\def\maxwidth{\ifdim\Gin@nat@width>\linewidth\linewidth\else\Gin@nat@width\fi}
\def\maxheight{\ifdim\Gin@nat@height>\textheight\textheight\else\Gin@nat@height\fi}
\makeatother
% Scale images if necessary, so that they will not overflow the page
% margins by default, and it is still possible to overwrite the defaults
% using explicit options in \includegraphics[width, height, ...]{}
\setkeys{Gin}{width=\maxwidth,height=\maxheight,keepaspectratio}
% Set default figure placement to htbp
\makeatletter
\def\fps@figure{htbp}
\makeatother
\setlength{\emergencystretch}{3em} % prevent overfull lines
\providecommand{\tightlist}{%
  \setlength{\itemsep}{0pt}\setlength{\parskip}{0pt}}
\setcounter{secnumdepth}{-\maxdimen} % remove section numbering
\ifLuaTeX
\usepackage[bidi=basic]{babel}
\else
\usepackage[bidi=default]{babel}
\fi
\babelprovide[main,import]{catalan}
% get rid of language-specific shorthands (see #6817):
\let\LanguageShortHands\languageshorthands
\def\languageshorthands#1{}
\ifLuaTeX
  \usepackage{selnolig}  % disable illegal ligatures
\fi
\usepackage{bookmark}
\IfFileExists{xurl.sty}{\usepackage{xurl}}{} % add URL line breaks if available
\urlstyle{same}
\hypersetup{
  pdftitle={U5. LUBUNTU. ESTRUCTURA},
  pdfauthor={@tofermos 2024},
  pdflang={ca-ES},
  hidelinks,
  pdfcreator={LaTeX via pandoc}}

\title{U5. LUBUNTU. ESTRUCTURA}
\author{@tofermos 2024}
\date{}

\begin{document}
\maketitle

{
\setcounter{tocdepth}{2}
\tableofcontents
}
\setstretch{1.5}
\newpage
\renewcommand\tablename{Tabla}

\section{1. El sistema operatiu
GNU/Linux.}\label{el-sistema-operatiu-gnulinux.}

\subsection{1.1 Característiques}\label{caracteruxedstiques}

\begin{itemize}
\tightlist
\item
  \textbf{GNU/Linux} és un sistema operatiu lliure i de codi obert.
\item
  Basat en el nucli \textbf{Linux} i eines del projecte \textbf{GNU}.
\item
  Escalable: pot ser utilitzat en dispositius menuts o grans servidors.
\end{itemize}

\subsection{1.2 Distribucions}\label{distribucions}

Una \textbf{distribució} Linux (distro) és un sistema operatiu basat en
el nucli Linux que inclou programari específic, eines d'administració,
gestors de paquets i una configuració personalitzada. Cada distribució
s'adapta a necessitats concretes:ús general, educació, servidors,
sistemes lleugers\ldots{}

Exemples de distribucions:

\begin{longtable}[]{@{}ll@{}}
\toprule\noalign{}
\textbf{Distribució} & \textbf{Orientació} \\
\midrule\noalign{}
\endhead
\bottomrule\noalign{}
\endlastfoot
Ubuntu & Usuaris nous, suport comunitari. \\
Debian & Estable, per servidors. \\
Arch Linux & Personalitzable, usuaris experts. \\
Fedora & \\
Manjaro & \\
\end{longtable}

Als cicles de FP ens centrarem en \textbf{Ubuntu}

\section{2 UBUNTU}\label{ubuntu}

\subsection{2.1 Característiques}\label{caracteruxedstiques-1}

\begin{itemize}
\tightlist
\item
  \textbf{Lliure i obert}: Es pot modificar i distribuir.
\item
  \textbf{Multiusuari}: Diversos usuaris alhora.
\item
  \textbf{Multitasca}: Execució de múltiples processos.
\item
  \textbf{Segur}: Model de permisos.
\item
  \textbf{Portabilitat}: Funciona en diferents plataformes.
\end{itemize}

\subsection{2.2 La Instal·lació de
Ubuntu}\label{la-installaciuxf3-de-ubuntu}

\textbf{Resum dels pasos a seguir}

\begin{enumerate}
\def\labelenumi{\arabic{enumi}.}
\tightlist
\item
  Descarrega la imatge ISO des d'\href{https://ubuntu.com/}{ubuntu.com}.
\item
  Comprova que l'has descarregada bé ( codi HASH: \texttt{sha256sum},
  habitualment )
\item
  Grava-la en un PenDrive amb eines com \textbf{Ventoy} (o Rufus,
  Etcher).
\item
  Modifica el BootOrder de la UEFI per arrancar des del PenDrive.
\item
  Segueix el procés d'instal·lació: Partició SWAP (si en necessites),
  regióm idioma, fus horari, altres particions.
\item
  Canvia el BootOrder i incia sessió.
\end{enumerate}

\subsection{2.3 Versió d'Ubuntu}\label{versiuxf3-dubuntu}

Amb l'ordre:

\begin{Shaded}
\begin{Highlighting}[]
\ExtensionTok{lsb\_release} \AttributeTok{{-}a}
\end{Highlighting}
\end{Shaded}

o llegint el fitxer de configuració

\begin{Shaded}
\begin{Highlighting}[]
\FunctionTok{cat}\NormalTok{ /etc/os{-}release}
\end{Highlighting}
\end{Shaded}

Una vegada més veiem com en un fitxer de text pla es guarda informació
del sistema. En este cas:

\begin{itemize}
\tightlist
\item
  Nom del SO
\item
  Distribució
\item
  Versió
\end{itemize}

I el fitxer el manté una ordre: \texttt{lsb\_release}

\section{3 Estructura}\label{estructura}

\subsection{3.1 Nucli (Kernel)}\label{nucli-kernel}

\begin{itemize}
\tightlist
\item
  Gestió de recursos: memòria, CPU, dispositius.
\item
  Per veure la versió: \texttt{bash\ \ \ \ \ \ uname\ -r}
\item
  Exemple de resultat: \texttt{6.8.0-49-generic}.
\end{itemize}

\subsection{3.2 Shell}\label{shell}

\subsubsection{Definicio i tipus}\label{definicio-i-tipus}

\begin{itemize}
\tightlist
\item
  Interfície que permet interactuar amb el sistema amb les seues ordres.
\item
  Exemples: \textbf{bash}, \textbf{sh}, \textbf{dash}.
\end{itemize}

Els shells diponible sels tenim al fitxer /etc/shells

\begin{Shaded}
\begin{Highlighting}[]
\FunctionTok{cat}\NormalTok{ /etc/shells}
\end{Highlighting}
\end{Shaded}

Cada usuari té un assigant (ho podem vore a /etc/passwd) però es pot
canviar amb: chsh ( change shell)

\begin{Shaded}
\begin{Highlighting}[]
\FunctionTok{sudo}\NormalTok{ chsh }\AttributeTok{{-}s}\NormalTok{ /bin/bash tomas}
\end{Highlighting}
\end{Shaded}

El canvi es vorà immediatament en /etc/passwd però no s'aplicarà fins
que no reiniciem sessió amb l'usuari.

\subsubsection{Ordres}\label{ordres}

\begin{itemize}
\tightlist
\item
  Comandes bàsiques: \texttt{ls}, \texttt{cp}, \texttt{mv}.
\end{itemize}

\subsubsection{Utilitats}\label{utilitats}

\texttt{apt}, ǹet-tools\texttt{,}mdadm',

\subsubsection{Curiositat}\label{curiositat}

Podem instal·lar un emulador del shell més avançat de MS-Windows a
Ubuntu: el PowerShell.

\subsection{3.3 Entorn gràfic}\label{entorn-gruxe0fic}

\subsubsection{3.3.1 Entorn d'escriptori}\label{entorn-descriptori}

\begin{itemize}
\item
  Cada Distro en porta un Entorn per defecte que podem canviar i tindre
  un altre o més d'un i triar en l'inici de sessió d'usuari.
\item
  Un \textbf{Entorn d'escriptori} és el conjunt d'elements gràfics que
  defineixen la interfície visual del sistema operatiu. Inclou:

  \begin{itemize}
  \item
    Gestor de finestres. (a)
  \item
    Gestor de fitxers.
  \item
    Menús, barres d'eines, icones i aplicacions predeterminades.
  \item
    Emulador de terminal.

    \begin{enumerate}
    \def\labelenumi{(\alph{enumi})}
    \tightlist
    \item
      Usat per tots els altres components ( està un nivell per baix)
    \end{enumerate}
  \end{itemize}
\item
  Tot que \textbf{cada eçentorn d'escriptori té el seu gestor de
  finestres i gestor de fitxers per defecte però podeu canviar-los} i es
  manté l'estructura.
\item
  No tots els canvis funcionen perfectament.
\end{itemize}

\subsubsection{3.3.2 Gestor d'arxius}\label{gestor-darxius}

Un \textbf{gestor d'arxius} és una aplicació que permet navegar,
organitzar, copiar, moure i gestionar fitxers i carpetes del sistema. És
una peça essencial en qualsevol entorn d'escriptori.

Per defecte cada Entorn d'Escriptori en duu un però podem instal·lar i
desinstal·lar-ne, tindre'n més d'un\ldots{}

\textbf{Exemples:} Nautilus (Gnome), Dolphin (KDE Plasma), Thunar
(Xfce), PCManFM-Qt (LXQt).

\subsubsection{3.3.3 Emulador de terminal}\label{emulador-de-terminal}

Un \textbf{emulador de terminal} és una aplicació que proporciona accés
a la línia de comandes. Permet interactuar directament amb el sistema
operatiu, executar ordres, i gestionar processos i fitxers de manera
avançada.

Per defecte cada Entorn d'Escriptori en duu un però (igual que amb el
Gestor de Fitxers) podem instal·lar i desinstal·lar-ne, tindre'n més
d'un\ldots{}

\textbf{Exemples:} Gnome Terminal, Konsole (KDE), Xfce Terminal,
QTerminal (LXQt).

\subsubsection{3.3.4 Gestor de Finestres (+ informació al Punt
6)}\label{gestor-de-finestres-informaciuxf3-al-punt-6}

El gestor de finestres és un component més bàsic per al funcionament de
l'entorn gràfic en general. Controla les finestres de totes les
aplicacions, no només del gestor de fitxers. Per tant, està en un nivell
inferior al Gestor de Finestres, navegadors web, editors de text,
terminals, etc.

\emph{(+ informació al Punt 6: Fora de temari en SOM)}

\begin{center}\rule{0.5\linewidth}{0.5pt}\end{center}

\section{\texorpdfstring{4 L'\emph{ecosistema} gràfic de
Linux}{4 L'ecosistema gràfic de Linux}}\label{lecosistema-gruxe0fic-de-linux}

\subsection{4.1 Quadre resum}\label{quadre-resum}

\begin{longtable}[]{@{}
  >{\raggedright\arraybackslash}p{(\columnwidth - 10\tabcolsep) * \real{0.1137}}
  >{\raggedright\arraybackslash}p{(\columnwidth - 10\tabcolsep) * \real{0.1185}}
  >{\raggedright\arraybackslash}p{(\columnwidth - 10\tabcolsep) * \real{0.2512}}
  >{\raggedright\arraybackslash}p{(\columnwidth - 10\tabcolsep) * \real{0.1517}}
  >{\raggedright\arraybackslash}p{(\columnwidth - 10\tabcolsep) * \real{0.1991}}
  >{\raggedright\arraybackslash}p{(\columnwidth - 10\tabcolsep) * \real{0.1659}}@{}}
\toprule\noalign{}
\begin{minipage}[b]{\linewidth}\raggedright
\textbf{Distribució}
\end{minipage} & \begin{minipage}[b]{\linewidth}\raggedright
\textbf{Entorn d'escriptori}
\end{minipage} & \begin{minipage}[b]{\linewidth}\raggedright
\textbf{Usos habituals}
\end{minipage} & \begin{minipage}[b]{\linewidth}\raggedright
\textbf{Paquet}
\end{minipage} & \begin{minipage}[b]{\linewidth}\raggedright
\textbf{Gestor d'arxius per defecte} (paquet)
\end{minipage} & \begin{minipage}[b]{\linewidth}\raggedright
\textbf{Emulador de terminal per defecte} (paquet)
\end{minipage} \\
\midrule\noalign{}
\endhead
\bottomrule\noalign{}
\endlastfoot
\textbf{Lubuntu} & \textbf{LXQt} & Perfecte per a equips molt antics o
amb pocs recursos. & \texttt{lubuntu-desktop} & \textbf{PCManFM-Qt
(\texttt{pcmanfm-qt})} & \textbf{QTerminal (\texttt{qterminal})} \\
\textbf{Ubuntu} & \textbf{Gnome} & Ideal per a usuaris que busquen
simplicitat i modernitat. & \texttt{ubuntu-gnome-desktop} &
\textbf{Nautilus (Files) (\texttt{nautilus})} & \textbf{Gnome Terminal
(\texttt{gnome-terminal})} \\
\textbf{Linux Mint} & \textbf{Cinnamon} & Per a usuaris que busquen un
entorn similar a Windows. & \texttt{cinnamon-desktop-environment} &
\textbf{Nemo (\texttt{nemo})} & \textbf{Gnome Terminal
(\texttt{gnome-terminal})} \\
\textbf{Kubuntu} & \textbf{KDE Plasma} & Per a usuaris que volen
personalització extrema i eines avançades. & \texttt{kde-plasma-desktop}
& \textbf{Dolphin (\texttt{dolphin})} & \textbf{Konsole
(\texttt{konsole})} \\
\textbf{Xubuntu} & \textbf{Xfce} & Ideal per a equips antics o usuaris
que prefereixen un entorn lleuger i funcional. &
\texttt{-\/-\/-xubuntu-desktop} & \textbf{Thunar (\texttt{thunar})} &
\textbf{Xfce Terminal (\texttt{xfce4-terminal})} \\
\textbf{Ubuntu Mate} & \textbf{Mate} & Per usuaris que prefereixen una
experiència clàssica i retro. & \texttt{ubuntu-mate-desktop} &
\textbf{Caja (\texttt{caja})} & \textbf{Mate Terminal
(\texttt{mate-terminal})} \\
\textbf{Ubuntu Budgie} & \textbf{Budgie} & Perfecte per a usuaris que
volen una experiència moderna i neta. & \texttt{ubuntu-budgie-desktop} &
\textbf{Nautilus (Files) (\texttt{nautilus})} & \textbf{Tilix
(\texttt{tilix})} \\
\textbf{Ubuntu Unity} & \textbf{Unity} & Per a usuaris nostàlgics
d'Ubuntu (pre-Gnome 2017). & \texttt{ubuntu-unity-desktop} &
\textbf{Nautilus (Files) (\texttt{nautilus})} & \textbf{Gnome Terminal
(\texttt{gnome-terminal})} \\
\end{longtable}

\subsection{4.2 Instal·lació i desinstal·lació d'un entorn
d'escriptori}\label{installaciuxf3-i-desinstallaciuxf3-dun-entorn-descriptori}

\begin{itemize}
\item
  \textbf{Instal·lació}:

\begin{Shaded}
\begin{Highlighting}[]
\FunctionTok{sudo}\NormalTok{ apt install ubuntu{-}gnome{-}desktop}
\end{Highlighting}
\end{Shaded}
\item
  \textbf{Desinstal·lació}:

\begin{Shaded}
\begin{Highlighting}[]
\FunctionTok{sudo}\NormalTok{ apt remove }\AttributeTok{{-}{-}purge}\NormalTok{ ubuntu{-}gnome{-}desktop}
\FunctionTok{sudo}\NormalTok{ apt autoremove }\AttributeTok{{-}{-}purge}
\end{Highlighting}
\end{Shaded}
\end{itemize}

\textbf{El gestor de pantalles}

Quan instal·les un nou escriptori (com ara GNOME en Lubuntu, que
utilitza per defecte LXQt), potser et demane que tries quin
\textbf{gestor de pantalles (display manager)} prefreixes usar.

El gestor de pantalles és el programa que s'encarrega de la interfície
d'inici de sessió gràfica. És el que veus quan encens l'ordinador i et
demana que introduesques el teu usuari, contrasenya i l'escriptori
gràfic o sessió vols iniciar (per exemple, GNOME, KDE, LXQt, etc.).

\begin{itemize}
\tightlist
\item
  gdm3 (GNOME Display Manager): Dissenyat per a GNOME, però pot
  gestionar altres entorns gràfics.
\item
  sddm (Simple Desktop Display Manager): Sovint utilitzat per entorns
  lleugers com LXQt o KDE Plasma.
\end{itemize}

\begin{figure}
\centering
\includegraphics{png/avisGestorPantalla.png}
\caption{\emph{Figura 1: gestor de pantalles}}
\end{figure}

Si vols prioritzar l'ús de GNOME (el que acabes d'instal·lar),
selecciona gdm3. Si prefereixes mantenir l'entorn lleuger de Lubuntu
(LXQt), pots optar per sddm.

Si en algun moment necessites canviar de gestor de pantalles, pots
fer-ho amb l'a comand'ordre

\begin{Shaded}
\begin{Highlighting}[]
\FunctionTok{sudo}\NormalTok{ dpkg{-}reconfigure gdm3}
\end{Highlighting}
\end{Shaded}

\subsection{4.3 Instal·lar i desinstal·lar un gestor de
fitxers}\label{installar-i-desinstallar-un-gestor-de-fitxers}

\begin{itemize}
\item
  \textbf{Instal·lació}:

\begin{Shaded}
\begin{Highlighting}[]
\FunctionTok{sudo}\NormalTok{ apt install dolphin}
\end{Highlighting}
\end{Shaded}
\item
  \textbf{Desinstal·lació}:

\begin{Shaded}
\begin{Highlighting}[]
\FunctionTok{sudo}\NormalTok{ apt remove }\AttributeTok{{-}{-}purge}\NormalTok{ dolphin}
\end{Highlighting}
\end{Shaded}
\end{itemize}

\subsection{4.4 Instal·lar i desinstal·lar un emulador de
terminal}\label{installar-i-desinstallar-un-emulador-de-terminal}

\begin{itemize}
\item
  \textbf{Instal·lació}:

\begin{Shaded}
\begin{Highlighting}[]
\FunctionTok{sudo}\NormalTok{ apt install tilix}
\end{Highlighting}
\end{Shaded}
\item
  \textbf{Desinstal·lació}:

\begin{Shaded}
\begin{Highlighting}[]
\FunctionTok{sudo}\NormalTok{ apt remove }\AttributeTok{{-}{-}purge}\NormalTok{ tilix}
\end{Highlighting}
\end{Shaded}
\end{itemize}

\begin{quote}
\textbf{Nota:}

Després de desinstal·lar qualsevol paquet, és recomanable executar:

\begin{Shaded}
\begin{Highlighting}[]
\FunctionTok{sudo}\NormalTok{ apt autoremove }\AttributeTok{{-}{-}purge}
\end{Highlighting}
\end{Shaded}

Això elimina dependències innecessàries per alliberar espai al sistema.
\end{quote}

\subsection{4.5 Dreceres (combinació de
tecles)}\label{dreceres-combinaciuxf3-de-tecles}

Lligat a l'anterior punt, pot ser interessant accedir a l'emulador de
terminal o al gestor de fitxers gràfic amb una sola combinació de
tecles.

\begin{figure}
\centering
\includegraphics{png/Drecera0.png}
\caption{Figura1: Accedir a la configuració de dreceres}
\end{figure}

\subsubsection{Modificar una drecera
existent:}\label{modificar-una-drecera-existent}

Amb \emph{Control + T} obrim el QTerminal

\begin{figure}
\centering
\includegraphics{png/Drecera1.png}
\caption{Figura2: Veiem quin programa crida}
\end{figure}

Canviem el nom del terminal\ldots{}

Ara obrirem el Mate-terminal que hem instal·lat.

\begin{figure}
\centering
\includegraphics{png/Drecera2.png}
\caption{Figura3: Fem el canvi de programa}
\end{figure}

\subsubsection{Crear una drecera nova}\label{crear-una-drecera-nova}

Amb \emph{Control + E} obrirem el gestor de fitxers Caja (que devem
instal·lar)

\begin{figure}
\centering
\includegraphics{png/Drecera3.png}
\caption{Figura4: Creem una nova drecera}
\end{figure}

\subsection{4.6 Aplicacions}\label{aplicacions}

\begin{itemize}
\tightlist
\item
  Navegador web: \textbf{Firefox}.
\item
  Suite ofimàtica: \textbf{LibreOffice}.
\item
  Reproductor multimèdia: \textbf{VLC}.
\end{itemize}

Habitualment entenem per aplicacions les que són gràfiques però no hem
d'oblidar que hi ha moltes eines que són aplicacions que s'executen en
mode CLI, encara que són software de sistemes per regla genearal. Les
hem anomenat al punt 1.2: apt, net-tools, nano, mdadm\ldots{}

La instal·lació ja hem vist que podem fer-la amb l'eina \emph{apt}. Més
avant ho repassarem i vorem altres formes.

\section{5 Conclusions}\label{conclusions}

Què podem dir ja de Linux o d'Ubuntu amb el que hem vist i el poc que
sabem de Windows o Android?

\textbf{Delimitació o capes:}

La primera cosa a destacar al món d'Ubuntu és la clara separació entre
nuclis, shell i GUI. Dins del GUI, veiem que també queda ben delimitat
el gestor de finestres, el gestors d'arxius dins del Escriptori.

\textbf{Diversitat i compatibilitat:}

La segona qüestió a destacar és la quantitat de Escriptoris, Gestors de
finestres, Gestors de fitxers i, fins i tot d'emuladors de terminals! i
la intercanviabilitat.

\emph{Compte que només estem veient una part del món Ubuntu. La galaxia
Linux és més gran i friqui encara!}

Tot açò junt a característiques com ser multiusuari (poder tindre
sessions de terminal de distints usuaris al mateix temps) fa que siga
ideal per a una introducció al món del Sistema Operatius.

Podem fer proves al nostre entorn de proves (màquines virtuals de
VirtualBox) instal·lant Escriptoris, Gestors o Terminals, però una
vegada tinguem clar quin volem, interessa alleugerir la instal·lació amb
només el que anem a usar.

\section{6 (Fora de temari) Gestor de
finestres}\label{fora-de-temari-gestor-de-finestres}

El gestor de finestres és un component més bàsic per al funcionament de
l'entorn gràfic en general. Controla les finestres de totes les
aplicacions, no només del gestor de fitxers. Per tant, està en un nivell
inferior al Gestor de Finestres, navegadors web, editors de text,
terminals, etc.

\begin{longtable}[]{@{}
  >{\raggedright\arraybackslash}p{(\columnwidth - 8\tabcolsep) * \real{0.1390}}
  >{\raggedright\arraybackslash}p{(\columnwidth - 8\tabcolsep) * \real{0.2086}}
  >{\raggedright\arraybackslash}p{(\columnwidth - 8\tabcolsep) * \real{0.2032}}
  >{\raggedright\arraybackslash}p{(\columnwidth - 8\tabcolsep) * \real{0.2193}}
  >{\raggedright\arraybackslash}p{(\columnwidth - 8\tabcolsep) * \real{0.2299}}@{}}
\toprule\noalign{}
\begin{minipage}[b]{\linewidth}\raggedright
\textbf{Entorn d'Escriptori}
\end{minipage} & \begin{minipage}[b]{\linewidth}\raggedright
\textbf{Gestor de Finestres Predeterminat}
\end{minipage} & \begin{minipage}[b]{\linewidth}\raggedright
\textbf{Gestor de Fitxers Predeterminat}
\end{minipage} & \begin{minipage}[b]{\linewidth}\raggedright
\textbf{Alternatives de Gestor de Finestres}
\end{minipage} & \begin{minipage}[b]{\linewidth}\raggedright
\textbf{Alternatives de Gestor de Fitxers}
\end{minipage} \\
\midrule\noalign{}
\endhead
\bottomrule\noalign{}
\endlastfoot
\textbf{GNOME} & Mutter & Nautilus (Files) & Openbox, i3, bspwm &
Thunar, PCManFM, Dolphin \\
\textbf{LXQt} & Openbox (o altres lleugers) & PCManFM-Qt & i3, Fluxbox,
bspwm & Thunar, Dolphin, Nautilus \\
\textbf{MATE} & Marco & Caja & Openbox, i3, bspwm & Nautilus, Dolphin,
Thunar \\
\textbf{KDE Plasma} & KWin & Dolphin & Openbox, i3, bspwm & Thunar,
Nautilus, PCManFM \\
\textbf{Xfce} & Xfwm & Thunar & Openbox, i3, bspwm & Nautilus, Dolphin,
PCManFM \\
& & & & \\
\textbf{Unity (antic)} & Compiz & Nautilus & Openbox, Mutter & Thunar,
Dolphin, PCManFM \\
& & & & \\
\end{longtable}

\textbf{Recomanacions}

\begin{itemize}
\tightlist
\item
  \textbf{Compatibilitat}: Quan canvies un gestor de finestres o un
  gestor de fitxers, assegura't que és compatible amb l'entorn que
  utilitzes. Això és especialment important en entorns com GNOME o KDE,
  que tenen una integració molt ajustada.
\item
  \textbf{Flexibilitat}: En entorns més lleugers com \textbf{LXQt},
  \textbf{LXDE} o \textbf{Xfce}, és més fàcil combinar components
  diferents sense trencar l'experiència de l'escriptori.
\item
  \textbf{Minimalisme}: Si utilitzes un gestor de finestres minimalista
  com \textbf{i3} o \textbf{bspwm}, sovint hauràs de configurar
  manualment el gestor de fitxers i altres components.
\end{itemize}

\begin{quote}
\textbf{Nota}

La MV de VirutalBox et permet fer les proves abans.
\end{quote}

\subsection{Canviar en la sessió de Gestor de
Finestres}\label{canviar-en-la-sessiuxf3-de-gestor-de-finestres}

Instal·la un nou Gestor de finestres

\begin{Shaded}
\begin{Highlighting}[]
\FunctionTok{sudo}\NormalTok{ apt install openbox}
\end{Highlighting}
\end{Shaded}

Substitueix Marco pel nou gestor:

\begin{Shaded}
\begin{Highlighting}[]
\ExtensionTok{openbox} \AttributeTok{{-}{-}replace}
\end{Highlighting}
\end{Shaded}

\subsection{Canvi permanent}\label{canvi-permanent}

Al nostre Lubuntu, com exemple, tenim el Gestor de finestres OpenBox.
Ara, instal·lem el i3 i el posem com predeterminat. El provarem

\begin{figure}
\centering
\includegraphics{png/AjustsGestorFinestres0.png}
\caption{\emph{Figura5: Canvi del gestor de finestres}}
\end{figure}

\begin{figure}
\centering
\includegraphics{png/AjustsGestorFinestres1.png}
\caption{\emph{Figura6: Canvi del gestor de finestres}}
\end{figure}

Vegem 3 Gestors de Fitxers amb el mateix Gestor de Finetres. Podem fer
canvis i comprovar que totes les combinacions no funcionen massa bé.

\begin{figure}
\centering
\includegraphics{png/3ambOpenBox.png}
\caption{\emph{Figura7: Caja, Thunar i PCMan-FM-Qt amb OpenBox}}
\end{figure}

\end{document}
