% Options for packages loaded elsewhere
\PassOptionsToPackage{unicode}{hyperref}
\PassOptionsToPackage{hyphens}{url}
%
\documentclass[
  12 pt,
  a4paper,
]{article}
\usepackage{amsmath,amssymb}
\usepackage{setspace}
\usepackage{iftex}
\ifPDFTeX
  \usepackage[T1]{fontenc}
  \usepackage[utf8]{inputenc}
  \usepackage{textcomp} % provide euro and other symbols
\else % if luatex or xetex
  \usepackage{unicode-math} % this also loads fontspec
  \defaultfontfeatures{Scale=MatchLowercase}
  \defaultfontfeatures[\rmfamily]{Ligatures=TeX,Scale=1}
\fi
\usepackage{lmodern}
\ifPDFTeX\else
  % xetex/luatex font selection
  \setmainfont[]{Times New Roman}
\fi
% Use upquote if available, for straight quotes in verbatim environments
\IfFileExists{upquote.sty}{\usepackage{upquote}}{}
\IfFileExists{microtype.sty}{% use microtype if available
  \usepackage[]{microtype}
  \UseMicrotypeSet[protrusion]{basicmath} % disable protrusion for tt fonts
}{}
\makeatletter
\@ifundefined{KOMAClassName}{% if non-KOMA class
  \IfFileExists{parskip.sty}{%
    \usepackage{parskip}
  }{% else
    \setlength{\parindent}{0pt}
    \setlength{\parskip}{6pt plus 2pt minus 1pt}}
}{% if KOMA class
  \KOMAoptions{parskip=half}}
\makeatother
\usepackage{xcolor}
\usepackage[margin=1in]{geometry}
\usepackage{graphicx}
\makeatletter
\def\maxwidth{\ifdim\Gin@nat@width>\linewidth\linewidth\else\Gin@nat@width\fi}
\def\maxheight{\ifdim\Gin@nat@height>\textheight\textheight\else\Gin@nat@height\fi}
\makeatother
% Scale images if necessary, so that they will not overflow the page
% margins by default, and it is still possible to overwrite the defaults
% using explicit options in \includegraphics[width, height, ...]{}
\setkeys{Gin}{width=\maxwidth,height=\maxheight,keepaspectratio}
% Set default figure placement to htbp
\makeatletter
\def\fps@figure{htbp}
\makeatother
\setlength{\emergencystretch}{3em} % prevent overfull lines
\providecommand{\tightlist}{%
  \setlength{\itemsep}{0pt}\setlength{\parskip}{0pt}}
\setcounter{secnumdepth}{-\maxdimen} % remove section numbering
\ifLuaTeX
\usepackage[bidi=basic]{babel}
\else
\usepackage[bidi=default]{babel}
\fi
\babelprovide[main,import]{spanish}
\ifPDFTeX
\else
\babelfont{rm}[]{Times New Roman}
\fi
% get rid of language-specific shorthands (see #6817):
\let\LanguageShortHands\languageshorthands
\def\languageshorthands#1{}
\ifLuaTeX
  \usepackage{selnolig}  % disable illegal ligatures
\fi
\usepackage{bookmark}
\IfFileExists{xurl.sty}{\usepackage{xurl}}{} % add URL line breaks if available
\urlstyle{same}
\hypersetup{
  pdfauthor={Tomàs Ferrandis Moscardó},
  pdflang={es-ES},
  hidelinks,
  pdfcreator={LaTeX via pandoc}}

\title{U7-ADMINISTRACIÓ D'UBUNTU}
\usepackage{etoolbox}
\makeatletter
\providecommand{\subtitle}[1]{% add subtitle to \maketitle
  \apptocmd{\@title}{\par {\large #1 \par}}{}{}
}
\makeatother
\subtitle{~Exercicis scripts}
\author{Tomàs Ferrandis Moscardó}
\date{}

\begin{document}
\maketitle

\setstretch{1.5}
\subsubsection{\texorpdfstring{\textbf{Exercici 1: Nom de l'usuari
actual}}{Exercici 1: Nom de l'usuari actual}}\label{exercici-1-nom-de-lusuari-actual}

Crea un script que mostre el nom de l'usuari que està executant el
sistema i el guarde en una variable.

\begin{center}\rule{0.5\linewidth}{0.5pt}\end{center}

\subsubsection{\texorpdfstring{\textbf{Exercici 2: Llistat de
fitxers}}{Exercici 2: Llistat de fitxers}}\label{exercici-2-llistat-de-fitxers}

Crea un script que, donat un directori com a paràmetre, mostre el
llistat de fitxers que conté i guarde eixe llistat en una variable.

\begin{center}\rule{0.5\linewidth}{0.5pt}\end{center}

\subsubsection{\texorpdfstring{\textbf{Exercici 3: Espai lliure al
disc}}{Exercici 3: Espai lliure al disc}}\label{exercici-3-espai-lliure-al-disc}

Crea un script que calcule l'espai lliure al disc i el mostre per
pantalla, guardant-lo també en una variable.

\begin{center}\rule{0.5\linewidth}{0.5pt}\end{center}

\subsubsection{\texorpdfstring{\textbf{Exercici 4: Número de fitxers en
un
directori}}{Exercici 4: Número de fitxers en un directori}}\label{exercici-4-nuxfamero-de-fitxers-en-un-directori}

Crea un script que conte quants fitxers hi ha en un directori
especificat per l'usuari i mostre el resultat per pantalla.

\begin{center}\rule{0.5\linewidth}{0.5pt}\end{center}

\subsubsection{\texorpdfstring{\textbf{Exercici 5: Llig el contingut
d'un
fitxer}}{Exercici 5: Llig el contingut d'un fitxer}}\label{exercici-5-llig-el-contingut-dun-fitxer}

Crea un script que llija el contingut d'un fitxer donat per l'usuari com
a paràmetre i el mostre per pantalla.

\begin{center}\rule{0.5\linewidth}{0.5pt}\end{center}

\subsubsection{\texorpdfstring{\textbf{Exercici 6: Informació del
sistema}}{Exercici 6: Informació del sistema}}\label{exercici-6-informaciuxf3-del-sistema}

Crea un script que mostre informació bàsica del sistema, com el nom de
l'ordinador, la versió del kernel i el temps que porta el sistema en
funcionament.

\begin{center}\rule{0.5\linewidth}{0.5pt}\end{center}

\subsubsection{\texorpdfstring{\textbf{Exercici 7: Paràmetres
múltiples}}{Exercici 7: Paràmetres múltiples}}\label{exercici-7-paruxe0metres-muxfaltiples}

Crea un script que lligca diversos paràmetres passats per l'usuari i els
mostre per pantalla.

\begin{center}\rule{0.5\linewidth}{0.5pt}\end{center}

\subsubsection{\texorpdfstring{\textbf{Exercici 8: Crear un
fitxer}}{Exercici 8: Crear un fitxer}}\label{exercici-8-crear-un-fitxer}

Crea un script que cree un fitxer amb un nom indicat per l'usuari com a
paràmetre i escriga un text predefinit dins d'eixe fitxer.

\begin{center}\rule{0.5\linewidth}{0.5pt}\end{center}

\subsubsection{\texorpdfstring{\textbf{Exercici 9: Comprovar si un
fitxer
existeix}}{Exercici 9: Comprovar si un fitxer existeix}}\label{exercici-9-comprovar-si-un-fitxer-existeix}

Crea un script que comprove si un fitxer donat per l'usuari existeix i
mostre un missatge indicant si està present o no.

\begin{center}\rule{0.5\linewidth}{0.5pt}\end{center}

\subsubsection{\texorpdfstring{\textbf{Exercici 10: Processos en
execució}}{Exercici 10: Processos en execució}}\label{exercici-10-processos-en-execuciuxf3}

Crea un script que mostre el nombre de processos que actualment estan en
execució al sistema i guarde eixa informació en una variable. A més, el
script hauria de mostrar el procés amb més consum de CPU.

\end{document}
