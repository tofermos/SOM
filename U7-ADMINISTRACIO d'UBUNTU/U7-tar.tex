% Options for packages loaded elsewhere
\PassOptionsToPackage{unicode}{hyperref}
\PassOptionsToPackage{hyphens}{url}
%
\documentclass[
  12 pt,
  a4paper,
]{article}
\usepackage{amsmath,amssymb}
\usepackage{setspace}
\usepackage{iftex}
\ifPDFTeX
  \usepackage[T1]{fontenc}
  \usepackage[utf8]{inputenc}
  \usepackage{textcomp} % provide euro and other symbols
\else % if luatex or xetex
  \usepackage{unicode-math} % this also loads fontspec
  \defaultfontfeatures{Scale=MatchLowercase}
  \defaultfontfeatures[\rmfamily]{Ligatures=TeX,Scale=1}
\fi
\usepackage{lmodern}
\ifPDFTeX\else
  % xetex/luatex font selection
  \setmainfont[]{Times New Roman}
\fi
% Use upquote if available, for straight quotes in verbatim environments
\IfFileExists{upquote.sty}{\usepackage{upquote}}{}
\IfFileExists{microtype.sty}{% use microtype if available
  \usepackage[]{microtype}
  \UseMicrotypeSet[protrusion]{basicmath} % disable protrusion for tt fonts
}{}
\makeatletter
\@ifundefined{KOMAClassName}{% if non-KOMA class
  \IfFileExists{parskip.sty}{%
    \usepackage{parskip}
  }{% else
    \setlength{\parindent}{0pt}
    \setlength{\parskip}{6pt plus 2pt minus 1pt}}
}{% if KOMA class
  \KOMAoptions{parskip=half}}
\makeatother
\usepackage{xcolor}
\usepackage[margin=1in]{geometry}
\usepackage{color}
\usepackage{fancyvrb}
\newcommand{\VerbBar}{|}
\newcommand{\VERB}{\Verb[commandchars=\\\{\}]}
\DefineVerbatimEnvironment{Highlighting}{Verbatim}{commandchars=\\\{\}}
% Add ',fontsize=\small' for more characters per line
\usepackage{framed}
\definecolor{shadecolor}{RGB}{248,248,248}
\newenvironment{Shaded}{\begin{snugshade}}{\end{snugshade}}
\newcommand{\AlertTok}[1]{\textcolor[rgb]{0.94,0.16,0.16}{#1}}
\newcommand{\AnnotationTok}[1]{\textcolor[rgb]{0.56,0.35,0.01}{\textbf{\textit{#1}}}}
\newcommand{\AttributeTok}[1]{\textcolor[rgb]{0.13,0.29,0.53}{#1}}
\newcommand{\BaseNTok}[1]{\textcolor[rgb]{0.00,0.00,0.81}{#1}}
\newcommand{\BuiltInTok}[1]{#1}
\newcommand{\CharTok}[1]{\textcolor[rgb]{0.31,0.60,0.02}{#1}}
\newcommand{\CommentTok}[1]{\textcolor[rgb]{0.56,0.35,0.01}{\textit{#1}}}
\newcommand{\CommentVarTok}[1]{\textcolor[rgb]{0.56,0.35,0.01}{\textbf{\textit{#1}}}}
\newcommand{\ConstantTok}[1]{\textcolor[rgb]{0.56,0.35,0.01}{#1}}
\newcommand{\ControlFlowTok}[1]{\textcolor[rgb]{0.13,0.29,0.53}{\textbf{#1}}}
\newcommand{\DataTypeTok}[1]{\textcolor[rgb]{0.13,0.29,0.53}{#1}}
\newcommand{\DecValTok}[1]{\textcolor[rgb]{0.00,0.00,0.81}{#1}}
\newcommand{\DocumentationTok}[1]{\textcolor[rgb]{0.56,0.35,0.01}{\textbf{\textit{#1}}}}
\newcommand{\ErrorTok}[1]{\textcolor[rgb]{0.64,0.00,0.00}{\textbf{#1}}}
\newcommand{\ExtensionTok}[1]{#1}
\newcommand{\FloatTok}[1]{\textcolor[rgb]{0.00,0.00,0.81}{#1}}
\newcommand{\FunctionTok}[1]{\textcolor[rgb]{0.13,0.29,0.53}{\textbf{#1}}}
\newcommand{\ImportTok}[1]{#1}
\newcommand{\InformationTok}[1]{\textcolor[rgb]{0.56,0.35,0.01}{\textbf{\textit{#1}}}}
\newcommand{\KeywordTok}[1]{\textcolor[rgb]{0.13,0.29,0.53}{\textbf{#1}}}
\newcommand{\NormalTok}[1]{#1}
\newcommand{\OperatorTok}[1]{\textcolor[rgb]{0.81,0.36,0.00}{\textbf{#1}}}
\newcommand{\OtherTok}[1]{\textcolor[rgb]{0.56,0.35,0.01}{#1}}
\newcommand{\PreprocessorTok}[1]{\textcolor[rgb]{0.56,0.35,0.01}{\textit{#1}}}
\newcommand{\RegionMarkerTok}[1]{#1}
\newcommand{\SpecialCharTok}[1]{\textcolor[rgb]{0.81,0.36,0.00}{\textbf{#1}}}
\newcommand{\SpecialStringTok}[1]{\textcolor[rgb]{0.31,0.60,0.02}{#1}}
\newcommand{\StringTok}[1]{\textcolor[rgb]{0.31,0.60,0.02}{#1}}
\newcommand{\VariableTok}[1]{\textcolor[rgb]{0.00,0.00,0.00}{#1}}
\newcommand{\VerbatimStringTok}[1]{\textcolor[rgb]{0.31,0.60,0.02}{#1}}
\newcommand{\WarningTok}[1]{\textcolor[rgb]{0.56,0.35,0.01}{\textbf{\textit{#1}}}}
\usepackage{graphicx}
\makeatletter
\def\maxwidth{\ifdim\Gin@nat@width>\linewidth\linewidth\else\Gin@nat@width\fi}
\def\maxheight{\ifdim\Gin@nat@height>\textheight\textheight\else\Gin@nat@height\fi}
\makeatother
% Scale images if necessary, so that they will not overflow the page
% margins by default, and it is still possible to overwrite the defaults
% using explicit options in \includegraphics[width, height, ...]{}
\setkeys{Gin}{width=\maxwidth,height=\maxheight,keepaspectratio}
% Set default figure placement to htbp
\makeatletter
\def\fps@figure{htbp}
\makeatother
\setlength{\emergencystretch}{3em} % prevent overfull lines
\providecommand{\tightlist}{%
  \setlength{\itemsep}{0pt}\setlength{\parskip}{0pt}}
\setcounter{secnumdepth}{-\maxdimen} % remove section numbering
\ifLuaTeX
\usepackage[bidi=basic]{babel}
\else
\usepackage[bidi=default]{babel}
\fi
\babelprovide[main,import]{spanish}
\ifPDFTeX
\else
\babelfont{rm}[]{Times New Roman}
\fi
% get rid of language-specific shorthands (see #6817):
\let\LanguageShortHands\languageshorthands
\def\languageshorthands#1{}
\ifLuaTeX
  \usepackage{selnolig}  % disable illegal ligatures
\fi
\usepackage{bookmark}
\IfFileExists{xurl.sty}{\usepackage{xurl}}{} % add URL line breaks if available
\urlstyle{same}
\hypersetup{
  pdflang={es-ES},
  hidelinks,
  pdfcreator={LaTeX via pandoc}}

\title{U7-ADMINISTRACIÓ D'UBUNTU. Empaquetar i comprimir (tar)}
\usepackage{etoolbox}
\makeatletter
\providecommand{\subtitle}[1]{% add subtitle to \maketitle
  \apptocmd{\@title}{\par {\large #1 \par}}{}{}
}
\makeatother
\subtitle{~Empaquetar i comprimir \textbf{tar}}
\author{}
\date{\vspace{-2.5em}}

\begin{document}
\maketitle

\setstretch{1.5}
\section{1 Introducció: Empaquetar i
comprimir}\label{introducciuxf3-empaquetar-i-comprimir}

En informàtica, \textbf{empaquetar} i \textbf{comprimir} són processos
diferents però relacionats:\\
- \textbf{Empaquetar}: Agrupar diversos fitxers en un sol arxiu sense
reduir-ne la mida.\\
- \textbf{Comprimir}: Reduir la mida d'un fitxer utilitzant algoritmes
com \texttt{gzip} o \texttt{bzip2}.

L'eina \texttt{tar} (Tape ARchive) s'utilitza per empaquetar fitxers en
un sol arxiu \texttt{.tar}. Si es combina amb \texttt{gzip}
(\texttt{.tar.gz}), també es poden comprimir.

\begin{center}\rule{0.5\linewidth}{0.5pt}\end{center}

\section{\texorpdfstring{2 L'ordre
\texttt{tar}}{2 L'ordre tar}}\label{lordre-tar}

\subsection{2.1 Empaquetar i comprimir}\label{empaquetar-i-comprimir}

\texttt{tar\ -cxf\ arxiu.tar\ carpeta/}

Aquest comandament empaqueta el contingut de \textbf{carpeta/} en un
arxiu \texttt{.tar}.

\begin{itemize}
\tightlist
\item
  \texttt{-c} → Crea un nou arxiu \texttt{.tar}.\\
\item
  \texttt{-x} → (Error) Aquesta opció s'utilitza per extreure, però aquí
  no és necessària.\\
\item
  \texttt{-f} → Especifica el nom del fitxer \texttt{.tar}.
\end{itemize}

\begin{center}\rule{0.5\linewidth}{0.5pt}\end{center}

\subsection{\texorpdfstring{2.2 Desempaquetar i descomprimir.
\texttt{tar\ -xf\ arxiu.tar}}{2.2 Desempaquetar i descomprimir. tar -xf arxiu.tar}}\label{desempaquetar-i-descomprimir.-tar--xf-arxiu.tar}

Esta ordre \textbf{extreu} el contingut d'un arxiu \texttt{.tar} a la
ubicació actual.

\begin{itemize}
\tightlist
\item
  \texttt{-x} → Extreu fitxers d'un arxiu \texttt{.tar}.\\
\item
  \texttt{-f} → Indica el nom de l'arxiu a extreure.
\end{itemize}

Si l'arxiu està comprimit amb \texttt{gzip} (\texttt{.tar.gz}), cal
afegir \texttt{-z} així:

\begin{Shaded}
\begin{Highlighting}[]
\FunctionTok{tar} \AttributeTok{{-}xzf}\NormalTok{ arxiu.tar.gz}
\end{Highlighting}
\end{Shaded}

\subsection{\texorpdfstring{2.3 Ús de l'opció \texttt{-C} en
\texttt{tar}}{2.3 Ús de l'opció -C en tar}}\label{uxfas-de-lopciuxf3--c-en-tar}

L'opció \texttt{-C} permet especificar el directori on s'ha de realitzar
l'operació. Això és útil tant per empaquetar com per extreure fitxers en
una ubicació diferent.

\subsubsection{Empaquetar sense incloure la carpeta
contenidora}\label{empaquetar-sense-incloure-la-carpeta-contenidora}

Si volem empaquetar el contingut d'una carpeta sense incloure la carpeta
en si, podem fer:

\begin{Shaded}
\begin{Highlighting}[]
\FunctionTok{tar} \AttributeTok{{-}cf}\NormalTok{ arxiu.tar }\AttributeTok{{-}C}\NormalTok{ carpeta/ .}
\end{Highlighting}
\end{Shaded}

Això empaquetarà \textbf{els fitxers dins} de \texttt{carpeta/}, però no
la carpeta en si.

\subsubsection{Extreure en un directori
específic}\label{extreure-en-un-directori-especuxedfic}

Per defecte, \texttt{tar} extreu els fitxers en la ubicació actual. Si
volem extreure'ls en un altre directori, fem:

\begin{Shaded}
\begin{Highlighting}[]
\FunctionTok{tar} \AttributeTok{{-}xf}\NormalTok{ arxiu.tar }\AttributeTok{{-}C}\NormalTok{ /ruta/destí}
\end{Highlighting}
\end{Shaded}

Això descomprimirà els fitxers dins de \texttt{/ruta/destí} en lloc de
fer-ho en el directori actual.

\end{document}
