% Options for packages loaded elsewhere
\PassOptionsToPackage{unicode}{hyperref}
\PassOptionsToPackage{hyphens}{url}
%
\documentclass[
  12 pt,
  a4paper,
]{article}
\usepackage{amsmath,amssymb}
\usepackage{setspace}
\usepackage{iftex}
\ifPDFTeX
  \usepackage[T1]{fontenc}
  \usepackage[utf8]{inputenc}
  \usepackage{textcomp} % provide euro and other symbols
\else % if luatex or xetex
  \usepackage{unicode-math} % this also loads fontspec
  \defaultfontfeatures{Scale=MatchLowercase}
  \defaultfontfeatures[\rmfamily]{Ligatures=TeX,Scale=1}
\fi
\usepackage{lmodern}
\ifPDFTeX\else
  % xetex/luatex font selection
  \setmainfont[]{Times New Roman}
\fi
% Use upquote if available, for straight quotes in verbatim environments
\IfFileExists{upquote.sty}{\usepackage{upquote}}{}
\IfFileExists{microtype.sty}{% use microtype if available
  \usepackage[]{microtype}
  \UseMicrotypeSet[protrusion]{basicmath} % disable protrusion for tt fonts
}{}
\makeatletter
\@ifundefined{KOMAClassName}{% if non-KOMA class
  \IfFileExists{parskip.sty}{%
    \usepackage{parskip}
  }{% else
    \setlength{\parindent}{0pt}
    \setlength{\parskip}{6pt plus 2pt minus 1pt}}
}{% if KOMA class
  \KOMAoptions{parskip=half}}
\makeatother
\usepackage{xcolor}
\usepackage[margin=1in]{geometry}
\usepackage{color}
\usepackage{fancyvrb}
\newcommand{\VerbBar}{|}
\newcommand{\VERB}{\Verb[commandchars=\\\{\}]}
\DefineVerbatimEnvironment{Highlighting}{Verbatim}{commandchars=\\\{\}}
% Add ',fontsize=\small' for more characters per line
\usepackage{framed}
\definecolor{shadecolor}{RGB}{248,248,248}
\newenvironment{Shaded}{\begin{snugshade}}{\end{snugshade}}
\newcommand{\AlertTok}[1]{\textcolor[rgb]{0.94,0.16,0.16}{#1}}
\newcommand{\AnnotationTok}[1]{\textcolor[rgb]{0.56,0.35,0.01}{\textbf{\textit{#1}}}}
\newcommand{\AttributeTok}[1]{\textcolor[rgb]{0.13,0.29,0.53}{#1}}
\newcommand{\BaseNTok}[1]{\textcolor[rgb]{0.00,0.00,0.81}{#1}}
\newcommand{\BuiltInTok}[1]{#1}
\newcommand{\CharTok}[1]{\textcolor[rgb]{0.31,0.60,0.02}{#1}}
\newcommand{\CommentTok}[1]{\textcolor[rgb]{0.56,0.35,0.01}{\textit{#1}}}
\newcommand{\CommentVarTok}[1]{\textcolor[rgb]{0.56,0.35,0.01}{\textbf{\textit{#1}}}}
\newcommand{\ConstantTok}[1]{\textcolor[rgb]{0.56,0.35,0.01}{#1}}
\newcommand{\ControlFlowTok}[1]{\textcolor[rgb]{0.13,0.29,0.53}{\textbf{#1}}}
\newcommand{\DataTypeTok}[1]{\textcolor[rgb]{0.13,0.29,0.53}{#1}}
\newcommand{\DecValTok}[1]{\textcolor[rgb]{0.00,0.00,0.81}{#1}}
\newcommand{\DocumentationTok}[1]{\textcolor[rgb]{0.56,0.35,0.01}{\textbf{\textit{#1}}}}
\newcommand{\ErrorTok}[1]{\textcolor[rgb]{0.64,0.00,0.00}{\textbf{#1}}}
\newcommand{\ExtensionTok}[1]{#1}
\newcommand{\FloatTok}[1]{\textcolor[rgb]{0.00,0.00,0.81}{#1}}
\newcommand{\FunctionTok}[1]{\textcolor[rgb]{0.13,0.29,0.53}{\textbf{#1}}}
\newcommand{\ImportTok}[1]{#1}
\newcommand{\InformationTok}[1]{\textcolor[rgb]{0.56,0.35,0.01}{\textbf{\textit{#1}}}}
\newcommand{\KeywordTok}[1]{\textcolor[rgb]{0.13,0.29,0.53}{\textbf{#1}}}
\newcommand{\NormalTok}[1]{#1}
\newcommand{\OperatorTok}[1]{\textcolor[rgb]{0.81,0.36,0.00}{\textbf{#1}}}
\newcommand{\OtherTok}[1]{\textcolor[rgb]{0.56,0.35,0.01}{#1}}
\newcommand{\PreprocessorTok}[1]{\textcolor[rgb]{0.56,0.35,0.01}{\textit{#1}}}
\newcommand{\RegionMarkerTok}[1]{#1}
\newcommand{\SpecialCharTok}[1]{\textcolor[rgb]{0.81,0.36,0.00}{\textbf{#1}}}
\newcommand{\SpecialStringTok}[1]{\textcolor[rgb]{0.31,0.60,0.02}{#1}}
\newcommand{\StringTok}[1]{\textcolor[rgb]{0.31,0.60,0.02}{#1}}
\newcommand{\VariableTok}[1]{\textcolor[rgb]{0.00,0.00,0.00}{#1}}
\newcommand{\VerbatimStringTok}[1]{\textcolor[rgb]{0.31,0.60,0.02}{#1}}
\newcommand{\WarningTok}[1]{\textcolor[rgb]{0.56,0.35,0.01}{\textbf{\textit{#1}}}}
\usepackage{graphicx}
\makeatletter
\def\maxwidth{\ifdim\Gin@nat@width>\linewidth\linewidth\else\Gin@nat@width\fi}
\def\maxheight{\ifdim\Gin@nat@height>\textheight\textheight\else\Gin@nat@height\fi}
\makeatother
% Scale images if necessary, so that they will not overflow the page
% margins by default, and it is still possible to overwrite the defaults
% using explicit options in \includegraphics[width, height, ...]{}
\setkeys{Gin}{width=\maxwidth,height=\maxheight,keepaspectratio}
% Set default figure placement to htbp
\makeatletter
\def\fps@figure{htbp}
\makeatother
\setlength{\emergencystretch}{3em} % prevent overfull lines
\providecommand{\tightlist}{%
  \setlength{\itemsep}{0pt}\setlength{\parskip}{0pt}}
\setcounter{secnumdepth}{-\maxdimen} % remove section numbering
\ifLuaTeX
\usepackage[bidi=basic]{babel}
\else
\usepackage[bidi=default]{babel}
\fi
\babelprovide[main,import]{spanish}
\ifPDFTeX
\else
\babelfont{rm}[]{Times New Roman}
\fi
% get rid of language-specific shorthands (see #6817):
\let\LanguageShortHands\languageshorthands
\def\languageshorthands#1{}
\ifLuaTeX
  \usepackage{selnolig}  % disable illegal ligatures
\fi
\usepackage{bookmark}
\IfFileExists{xurl.sty}{\usepackage{xurl}}{} % add URL line breaks if available
\urlstyle{same}
\hypersetup{
  pdfauthor={Tomàs Ferrandis Moscardó},
  pdflang={es-ES},
  hidelinks,
  pdfcreator={LaTeX via pandoc}}

\title{U7- ADMINISTRACIÓ D'UBUNTU}
\usepackage{etoolbox}
\makeatletter
\providecommand{\subtitle}[1]{% add subtitle to \maketitle
  \apptocmd{\@title}{\par {\large #1 \par}}{}{}
}
\makeatother
\subtitle{~INTRODUCCIÓ ALS SCRIPTS}
\author{Tomàs Ferrandis Moscardó}
\date{}

\begin{document}
\maketitle

\setstretch{1.5}
\section{\texorpdfstring{1. La primera línia:
\texttt{\#!/bin/bash}}{1. La primera línia: \#!/bin/bash}}\label{la-primera-luxednia-binbash}

\begin{itemize}
\tightlist
\item
  La línia \texttt{\#!/bin/bash} indica al sistema que ha d'executar
  l'script amb Bash.\\
\item
  Sense aquesta línia, el sistema podria intentar executar l'script amb
  un altre shell i donar errors.
\end{itemize}

\section{Exemple:}\label{exemple}

\begin{Shaded}
\begin{Highlighting}[]
\CommentTok{\#!/bin/bash}
\BuiltInTok{echo} \StringTok{"Aquest script s\textquotesingle{}executa amb Bash"}
\end{Highlighting}
\end{Shaded}

\section{2. Extensió i assignació de permís
d'execució}\label{extensiuxf3-i-assignaciuxf3-de-permuxeds-dexecuciuxf3}

\begin{itemize}
\tightlist
\item
  Els scripts solen tenir l'extensió \texttt{.sh} per identificar-los
  fàcilment.\\
\item
  Passos per crear i executar un script:

  \begin{enumerate}
  \def\labelenumi{\arabic{enumi}.}
  \item
    \textbf{Crea un fitxer nou:}

\begin{Shaded}
\begin{Highlighting}[]
\FunctionTok{nano}\NormalTok{ exemple.sh}
\end{Highlighting}
\end{Shaded}
  \item
    \textbf{Escriu el codi dins del fitxer i guarda'l.}\\
  \item
    \textbf{Dona permís d'execució:}

\begin{Shaded}
\begin{Highlighting}[]
\FunctionTok{chmod}\NormalTok{ +x exemple.sh}
\end{Highlighting}
\end{Shaded}
  \item
    \textbf{Executa l'script:}

\begin{Shaded}
\begin{Highlighting}[]
\ExtensionTok{./exemple.sh}
\end{Highlighting}
\end{Shaded}
  \end{enumerate}
\end{itemize}

\section{3. Assignar valor a una variable i
llegir-la}\label{assignar-valor-a-una-variable-i-llegir-la}

\begin{itemize}
\item
  Pots assignar valors a variables sense espais:

\begin{Shaded}
\begin{Highlighting}[]
\VariableTok{nom}\OperatorTok{=}\StringTok{"Maria"}
\end{Highlighting}
\end{Shaded}
\item
  Per utilitzar o llegir el valor de la variable, escriu \texttt{\$}
  seguit del nom:

\begin{Shaded}
\begin{Highlighting}[]
\BuiltInTok{echo} \StringTok{"Hola, }\VariableTok{$nom}\StringTok{!"}
\end{Highlighting}
\end{Shaded}
\item
  Per llegir dades de l'usuari, utilitza \texttt{read}:

\begin{Shaded}
\begin{Highlighting}[]
\CommentTok{\#!/bin/bash}
\BuiltInTok{echo} \StringTok{"Com et dius?"}
\BuiltInTok{read} \VariableTok{nom}
\BuiltInTok{echo} \StringTok{"Hola, }\VariableTok{$nom}\StringTok{! Encantat de conèixer{-}vos."}
\end{Highlighting}
\end{Shaded}
\end{itemize}

\section{\texorpdfstring{4. Els paràmetres especials \texttt{\$0},
\texttt{\$1}, \texttt{\$2},
\ldots{}}{4. Els paràmetres especials \$0, \$1, \$2, \ldots{}}}\label{els-paruxe0metres-especials-0-1-2}

\begin{itemize}
\tightlist
\item
  Quan executes un script, pots passar valors com a paràmetres.\\
\item
  Aquests es llegeixen amb \texttt{\$1}, \texttt{\$2}, etc.
\end{itemize}

\section{Exemple d'script amb
paràmetres:}\label{exemple-dscript-amb-paruxe0metres}

\begin{Shaded}
\begin{Highlighting}[]
\CommentTok{\#!/bin/bash}
\BuiltInTok{echo} \StringTok{"Nom de l\textquotesingle{}script: }\VariableTok{$0}\StringTok{"}
\BuiltInTok{echo} \StringTok{"Primer paràmetre: }\VariableTok{$1}\StringTok{"}
\BuiltInTok{echo} \StringTok{"Segon paràmetre: }\VariableTok{$2}\StringTok{"}
\end{Highlighting}
\end{Shaded}

\section{Com executar-lo:}\label{com-executar-lo}

\begin{Shaded}
\begin{Highlighting}[]
\ExtensionTok{./exemple.sh}\NormalTok{ Hola Món}
\end{Highlighting}
\end{Shaded}

\section{Sortida:}\label{sortida}

\begin{verbatim}
Nom de l'script: ./exemple.sh
Primer paràmetre: Hola
Segon paràmetre: Món
\end{verbatim}

\section{\texorpdfstring{5. Comprovar si un paràmetre és un fitxer o un
directori amb
\texttt{if}}{5. Comprovar si un paràmetre és un fitxer o un directori amb if}}\label{comprovar-si-un-paruxe0metre-uxe9s-un-fitxer-o-un-directori-amb-if}

\begin{itemize}
\tightlist
\item
  Amb condicions, pots comprovar si un paràmetre és un fitxer o un
  directori.
\end{itemize}

\section{Exemple:}\label{exemple-1}

\begin{Shaded}
\begin{Highlighting}[]
\CommentTok{\#!/bin/bash}

\ControlFlowTok{if} \BuiltInTok{[} \VariableTok{$\#} \OtherTok{{-}eq}\NormalTok{ 0 }\BuiltInTok{]}\KeywordTok{;} \ControlFlowTok{then}
  \BuiltInTok{echo} \StringTok{"Per favor, passa un fitxer o directori com a paràmetre."}
  \BuiltInTok{exit}\NormalTok{ 1}
\ControlFlowTok{fi}

\VariableTok{element}\OperatorTok{=}\VariableTok{$1}

\ControlFlowTok{if} \BuiltInTok{[} \OtherTok{{-}f} \StringTok{"}\VariableTok{$element}\StringTok{"} \BuiltInTok{]}\KeywordTok{;} \ControlFlowTok{then}
  \BuiltInTok{echo} \StringTok{"}\VariableTok{$element}\StringTok{ és un fitxer."}
\ControlFlowTok{elif} \BuiltInTok{[} \OtherTok{{-}d} \StringTok{"}\VariableTok{$element}\StringTok{"} \BuiltInTok{]}\KeywordTok{;} \ControlFlowTok{then}
  \BuiltInTok{echo} \StringTok{"}\VariableTok{$element}\StringTok{ és un directori."}
\ControlFlowTok{else}
  \BuiltInTok{echo} \StringTok{"}\VariableTok{$element}\StringTok{ no existeix o no és ni un fitxer ni un directori."}
\ControlFlowTok{fi}
\end{Highlighting}
\end{Shaded}

\section{Com executar-lo:}\label{com-executar-lo-1}

\begin{Shaded}
\begin{Highlighting}[]
\ExtensionTok{./exemple.sh}\NormalTok{ nom\_fitxer\_o\_directori}
\end{Highlighting}
\end{Shaded}

\section{Resum de comandes clau}\label{resum-de-comandes-clau}

\begin{enumerate}
\def\labelenumi{\arabic{enumi}.}
\tightlist
\item
  \texttt{\#!/bin/bash}: Especifica que l'script s'executa amb Bash.\\
\item
  \texttt{chmod\ +x\ nom\_script.sh}: Dona permisos d'execució.\\
\item
  \texttt{\$0}, \texttt{\$1}, \texttt{\$2}: Accedeix al nom de l'script
  i als paràmetres.\\
\item
  \texttt{read}: Llegeix una entrada de l'usuari.\\
\item
  \texttt{{[}\ -f\ {]}}, \texttt{{[}\ -d\ {]}}: Comprova si un paràmetre
  és un fitxer o un directori.
\end{enumerate}

\end{document}
