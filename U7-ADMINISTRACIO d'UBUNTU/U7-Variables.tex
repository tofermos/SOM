% Options for packages loaded elsewhere
\PassOptionsToPackage{unicode}{hyperref}
\PassOptionsToPackage{hyphens}{url}
%
\documentclass[
  12 pt,
  a4paper,
]{article}
\usepackage{amsmath,amssymb}
\usepackage{setspace}
\usepackage{iftex}
\ifPDFTeX
  \usepackage[T1]{fontenc}
  \usepackage[utf8]{inputenc}
  \usepackage{textcomp} % provide euro and other symbols
\else % if luatex or xetex
  \usepackage{unicode-math} % this also loads fontspec
  \defaultfontfeatures{Scale=MatchLowercase}
  \defaultfontfeatures[\rmfamily]{Ligatures=TeX,Scale=1}
\fi
\usepackage{lmodern}
\ifPDFTeX\else
  % xetex/luatex font selection
  \setmainfont[]{Times New Roman}
\fi
% Use upquote if available, for straight quotes in verbatim environments
\IfFileExists{upquote.sty}{\usepackage{upquote}}{}
\IfFileExists{microtype.sty}{% use microtype if available
  \usepackage[]{microtype}
  \UseMicrotypeSet[protrusion]{basicmath} % disable protrusion for tt fonts
}{}
\makeatletter
\@ifundefined{KOMAClassName}{% if non-KOMA class
  \IfFileExists{parskip.sty}{%
    \usepackage{parskip}
  }{% else
    \setlength{\parindent}{0pt}
    \setlength{\parskip}{6pt plus 2pt minus 1pt}}
}{% if KOMA class
  \KOMAoptions{parskip=half}}
\makeatother
\usepackage{xcolor}
\usepackage[margin=1in]{geometry}
\usepackage{color}
\usepackage{fancyvrb}
\newcommand{\VerbBar}{|}
\newcommand{\VERB}{\Verb[commandchars=\\\{\}]}
\DefineVerbatimEnvironment{Highlighting}{Verbatim}{commandchars=\\\{\}}
% Add ',fontsize=\small' for more characters per line
\usepackage{framed}
\definecolor{shadecolor}{RGB}{248,248,248}
\newenvironment{Shaded}{\begin{snugshade}}{\end{snugshade}}
\newcommand{\AlertTok}[1]{\textcolor[rgb]{0.94,0.16,0.16}{#1}}
\newcommand{\AnnotationTok}[1]{\textcolor[rgb]{0.56,0.35,0.01}{\textbf{\textit{#1}}}}
\newcommand{\AttributeTok}[1]{\textcolor[rgb]{0.13,0.29,0.53}{#1}}
\newcommand{\BaseNTok}[1]{\textcolor[rgb]{0.00,0.00,0.81}{#1}}
\newcommand{\BuiltInTok}[1]{#1}
\newcommand{\CharTok}[1]{\textcolor[rgb]{0.31,0.60,0.02}{#1}}
\newcommand{\CommentTok}[1]{\textcolor[rgb]{0.56,0.35,0.01}{\textit{#1}}}
\newcommand{\CommentVarTok}[1]{\textcolor[rgb]{0.56,0.35,0.01}{\textbf{\textit{#1}}}}
\newcommand{\ConstantTok}[1]{\textcolor[rgb]{0.56,0.35,0.01}{#1}}
\newcommand{\ControlFlowTok}[1]{\textcolor[rgb]{0.13,0.29,0.53}{\textbf{#1}}}
\newcommand{\DataTypeTok}[1]{\textcolor[rgb]{0.13,0.29,0.53}{#1}}
\newcommand{\DecValTok}[1]{\textcolor[rgb]{0.00,0.00,0.81}{#1}}
\newcommand{\DocumentationTok}[1]{\textcolor[rgb]{0.56,0.35,0.01}{\textbf{\textit{#1}}}}
\newcommand{\ErrorTok}[1]{\textcolor[rgb]{0.64,0.00,0.00}{\textbf{#1}}}
\newcommand{\ExtensionTok}[1]{#1}
\newcommand{\FloatTok}[1]{\textcolor[rgb]{0.00,0.00,0.81}{#1}}
\newcommand{\FunctionTok}[1]{\textcolor[rgb]{0.13,0.29,0.53}{\textbf{#1}}}
\newcommand{\ImportTok}[1]{#1}
\newcommand{\InformationTok}[1]{\textcolor[rgb]{0.56,0.35,0.01}{\textbf{\textit{#1}}}}
\newcommand{\KeywordTok}[1]{\textcolor[rgb]{0.13,0.29,0.53}{\textbf{#1}}}
\newcommand{\NormalTok}[1]{#1}
\newcommand{\OperatorTok}[1]{\textcolor[rgb]{0.81,0.36,0.00}{\textbf{#1}}}
\newcommand{\OtherTok}[1]{\textcolor[rgb]{0.56,0.35,0.01}{#1}}
\newcommand{\PreprocessorTok}[1]{\textcolor[rgb]{0.56,0.35,0.01}{\textit{#1}}}
\newcommand{\RegionMarkerTok}[1]{#1}
\newcommand{\SpecialCharTok}[1]{\textcolor[rgb]{0.81,0.36,0.00}{\textbf{#1}}}
\newcommand{\SpecialStringTok}[1]{\textcolor[rgb]{0.31,0.60,0.02}{#1}}
\newcommand{\StringTok}[1]{\textcolor[rgb]{0.31,0.60,0.02}{#1}}
\newcommand{\VariableTok}[1]{\textcolor[rgb]{0.00,0.00,0.00}{#1}}
\newcommand{\VerbatimStringTok}[1]{\textcolor[rgb]{0.31,0.60,0.02}{#1}}
\newcommand{\WarningTok}[1]{\textcolor[rgb]{0.56,0.35,0.01}{\textbf{\textit{#1}}}}
\usepackage{longtable,booktabs,array}
\usepackage{calc} % for calculating minipage widths
% Correct order of tables after \paragraph or \subparagraph
\usepackage{etoolbox}
\makeatletter
\patchcmd\longtable{\par}{\if@noskipsec\mbox{}\fi\par}{}{}
\makeatother
% Allow footnotes in longtable head/foot
\IfFileExists{footnotehyper.sty}{\usepackage{footnotehyper}}{\usepackage{footnote}}
\makesavenoteenv{longtable}
\usepackage{graphicx}
\makeatletter
\def\maxwidth{\ifdim\Gin@nat@width>\linewidth\linewidth\else\Gin@nat@width\fi}
\def\maxheight{\ifdim\Gin@nat@height>\textheight\textheight\else\Gin@nat@height\fi}
\makeatother
% Scale images if necessary, so that they will not overflow the page
% margins by default, and it is still possible to overwrite the defaults
% using explicit options in \includegraphics[width, height, ...]{}
\setkeys{Gin}{width=\maxwidth,height=\maxheight,keepaspectratio}
% Set default figure placement to htbp
\makeatletter
\def\fps@figure{htbp}
\makeatother
\setlength{\emergencystretch}{3em} % prevent overfull lines
\providecommand{\tightlist}{%
  \setlength{\itemsep}{0pt}\setlength{\parskip}{0pt}}
\setcounter{secnumdepth}{-\maxdimen} % remove section numbering
\ifLuaTeX
\usepackage[bidi=basic]{babel}
\else
\usepackage[bidi=default]{babel}
\fi
\babelprovide[main,import]{spanish}
\ifPDFTeX
\else
\babelfont{rm}[]{Times New Roman}
\fi
% get rid of language-specific shorthands (see #6817):
\let\LanguageShortHands\languageshorthands
\def\languageshorthands#1{}
\ifLuaTeX
  \usepackage{selnolig}  % disable illegal ligatures
\fi
\usepackage{bookmark}
\IfFileExists{xurl.sty}{\usepackage{xurl}}{} % add URL line breaks if available
\urlstyle{same}
\hypersetup{
  pdfauthor={Tomàs Ferrandis Moscardó},
  pdflang={es-ES},
  hidelinks,
  pdfcreator={LaTeX via pandoc}}

\title{U7- ADMINISTRACIÓ D'UBUNTU}
\usepackage{etoolbox}
\makeatletter
\providecommand{\subtitle}[1]{% add subtitle to \maketitle
  \apptocmd{\@title}{\par {\large #1 \par}}{}{}
}
\makeatother
\subtitle{~VARIABLES}
\author{Tomàs Ferrandis Moscardó}
\date{}

\begin{document}
\maketitle

\setstretch{1.5}
\section{1 Variables creades per
l'usuari}\label{variables-creades-per-lusuari}

\paragraph{\texorpdfstring{\textbf{1. Assignar un valor a una
variable}}{1. Assignar un valor a una variable}}\label{assignar-un-valor-a-una-variable}

En Bash, assignar un valor a una variable és senzill. No hi ha espais
abans o després del signe \texttt{=}.

\begin{Shaded}
\begin{Highlighting}[]
\VariableTok{variable}\OperatorTok{=}\StringTok{"valor"}
\end{Highlighting}
\end{Shaded}

\begin{itemize}
\tightlist
\item
  \textbf{Punts clau:}

  \begin{itemize}
  \tightlist
  \item
    Les variables són sensibles a majúscules i minúscules (\texttt{nom}
    i \texttt{NOM} són diferents).
  \item
    No cal declarar el tipus de variable; Bash interpreta automàticament
    el tipus.
  \item
    Si el valor conté espais o caràcters especials, s'ha d'envoltar amb
    cometes dobles \texttt{"..."} o simples
    \texttt{\textquotesingle{}...\textquotesingle{}}.
  \end{itemize}
\end{itemize}

\textbf{Exemples:}

\begin{Shaded}
\begin{Highlighting}[]
\VariableTok{nom}\OperatorTok{=}\StringTok{"Joan"}
\VariableTok{cognoms}\OperatorTok{=}\StringTok{"Pérez López"}
\VariableTok{frase}\OperatorTok{=}\StringTok{\textquotesingle{}Hola, món!\textquotesingle{}}

\BuiltInTok{echo} \StringTok{"Nom: }\VariableTok{$nom}\StringTok{"}
\BuiltInTok{echo} \StringTok{"Cognoms: }\VariableTok{$cognoms}\StringTok{"}
\BuiltInTok{echo} \StringTok{"Frase: }\VariableTok{$frase}\StringTok{"}
\end{Highlighting}
\end{Shaded}

\begin{center}\rule{0.5\linewidth}{0.5pt}\end{center}

\paragraph{\texorpdfstring{\textbf{2. Assignar el resultat d'una comanda
a una
variable}}{2. Assignar el resultat d'una comanda a una variable}}\label{assignar-el-resultat-duna-comanda-a-una-variable}

Per assignar el resultat d'una comanda, pots utilitzar \textbf{cometes
invertides} \texttt{\textasciigrave{}...\textasciigrave{}} o
\textbf{substitució de comandes} amb \texttt{\$()}.

\begin{Shaded}
\begin{Highlighting}[]
\VariableTok{variable}\OperatorTok{=}\VariableTok{$(}\ExtensionTok{comanda}\VariableTok{)}
\CommentTok{\# o equivalent}
\VariableTok{variable}\OperatorTok{=}\KeywordTok{\textasciigrave{}}\ExtensionTok{comanda}\KeywordTok{\textasciigrave{}}
\end{Highlighting}
\end{Shaded}

\begin{itemize}
\item
  \textbf{Exemples:}

\begin{Shaded}
\begin{Highlighting}[]
\VariableTok{data}\OperatorTok{=}\VariableTok{$(}\FunctionTok{date}\VariableTok{)}
\VariableTok{usuaris}\OperatorTok{=}\VariableTok{$(}\FunctionTok{who} \KeywordTok{|} \FunctionTok{wc} \AttributeTok{{-}l}\VariableTok{)}

\BuiltInTok{echo} \StringTok{"La data actual és: }\VariableTok{$data}\StringTok{"}
\BuiltInTok{echo} \StringTok{"Hi ha }\VariableTok{$usuaris}\StringTok{ usuaris connectats."}
\end{Highlighting}
\end{Shaded}
\item
  \textbf{Nota:} Es recomana utilitzar \texttt{\$()} perquè és més
  llegible i permet l'anidament de comandes.
\end{itemize}

\begin{center}\rule{0.5\linewidth}{0.5pt}\end{center}

\paragraph{\texorpdfstring{\textbf{3. Llegir el valor d'una
variable}}{3. Llegir el valor d'una variable}}\label{llegir-el-valor-duna-variable}

Per utilitzar el valor d'una variable, precedeix el nom amb \texttt{\$}.

\begin{Shaded}
\begin{Highlighting}[]
\BuiltInTok{echo} \StringTok{"}\VariableTok{$variable}\StringTok{"}
\end{Highlighting}
\end{Shaded}

\begin{itemize}
\item
  \textbf{Exemples:}

\begin{Shaded}
\begin{Highlighting}[]
\VariableTok{nom}\OperatorTok{=}\StringTok{"Maria"}
\BuiltInTok{echo} \StringTok{"Hola, }\VariableTok{$nom}\StringTok{!"}
\end{Highlighting}
\end{Shaded}
\end{itemize}

Si necessites accedir a una variable dins d'un text llarg, pots
utilitzar \texttt{\{\}} per delimitar el nom:

\begin{Shaded}
\begin{Highlighting}[]
\VariableTok{prefix}\OperatorTok{=}\StringTok{"arxiu"}
\VariableTok{nom}\OperatorTok{=}\StringTok{"}\VariableTok{$\{prefix\}}\StringTok{\_config"}
\BuiltInTok{echo} \StringTok{"Nom del fitxer: }\VariableTok{$nom}\StringTok{"}
\end{Highlighting}
\end{Shaded}

\begin{center}\rule{0.5\linewidth}{0.5pt}\end{center}

\paragraph{\texorpdfstring{\textbf{4. Llegir valors des de l'entrada de
l'usuari}}{4. Llegir valors des de l'entrada de l'usuari}}\label{llegir-valors-des-de-lentrada-de-lusuari}

La comanda \texttt{read} permet assignar valors a una variable llegits
de l'entrada estàndard (teclat).

\begin{Shaded}
\begin{Highlighting}[]
\BuiltInTok{read} \VariableTok{variable}
\end{Highlighting}
\end{Shaded}

\begin{itemize}
\item
  \textbf{Exemple:}

\begin{Shaded}
\begin{Highlighting}[]
\BuiltInTok{echo} \StringTok{"Com et dius?"}
\BuiltInTok{read} \VariableTok{nom}
\BuiltInTok{echo} \StringTok{"Hola, }\VariableTok{$nom}\StringTok{!"}
\end{Highlighting}
\end{Shaded}
\end{itemize}

\begin{center}\rule{0.5\linewidth}{0.5pt}\end{center}

\subsubsection{\texorpdfstring{\textbf{Resum de bones
pràctiques}}{Resum de bones pràctiques}}\label{resum-de-bones-pruxe0ctiques}

\begin{enumerate}
\def\labelenumi{\arabic{enumi}.}
\item
  \textbf{Declarar variables sense espais}:

\begin{Shaded}
\begin{Highlighting}[]
\VariableTok{variable}\OperatorTok{=}\StringTok{"valor"} \CommentTok{\# Correcte}
\ExtensionTok{variable}\NormalTok{ = }\StringTok{"valor"} \CommentTok{\# Incorrecte}
\end{Highlighting}
\end{Shaded}
\item
  \textbf{Utilitzar cometes per protegir valors amb espais}:

\begin{Shaded}
\begin{Highlighting}[]
\VariableTok{nom}\OperatorTok{=}\StringTok{"Maria López"}
\end{Highlighting}
\end{Shaded}
\item
  \textbf{Assignar resultats de comandes amb \texttt{\$()}}:

\begin{Shaded}
\begin{Highlighting}[]
\VariableTok{fitxers}\OperatorTok{=}\VariableTok{$(}\FunctionTok{ls}\NormalTok{ /etc}\VariableTok{)}
\end{Highlighting}
\end{Shaded}
\item
  \textbf{Llegir l'entrada de l'usuari amb \texttt{read}}:

\begin{Shaded}
\begin{Highlighting}[]
\BuiltInTok{echo} \StringTok{"Escriu el teu nom:"}
\BuiltInTok{read} \VariableTok{nom}
\BuiltInTok{echo} \StringTok{"Hola, }\VariableTok{$nom}\StringTok{!"}
\end{Highlighting}
\end{Shaded}
\end{enumerate}

\section{2 Variables del sistema}\label{variables-del-sistema}

Les variables del sistema són variables predefinides a Bash que contenen
informació sobre l'entorn del sistema, l'usuari, els processos i més.
Algunes són només de lectura i altres es poden modificar.

\paragraph{\texorpdfstring{\textbf{1. Variables d'entorn
generals}}{1. Variables d'entorn generals}}\label{variables-dentorn-generals}

\begin{longtable}[]{@{}
  >{\raggedright\arraybackslash}p{(\columnwidth - 2\tabcolsep) * \real{0.1604}}
  >{\raggedright\arraybackslash}p{(\columnwidth - 2\tabcolsep) * \real{0.8396}}@{}}
\toprule\noalign{}
\begin{minipage}[b]{\linewidth}\raggedright
\textbf{Variable}
\end{minipage} & \begin{minipage}[b]{\linewidth}\raggedright
\textbf{Descripció}
\end{minipage} \\
\midrule\noalign{}
\endhead
\bottomrule\noalign{}
\endlastfoot
\texttt{HOME} & El directori personal de l'usuari. \\
\texttt{USER} & El nom de l'usuari actual. \\
\texttt{LOGNAME} & Nom de l'usuari que ha iniciat sessió. \\
\texttt{PATH} & Llista de directoris on el sistema busca executables
(separats per \texttt{:}). \\
\texttt{PWD} & El directori de treball actual. \\
\texttt{OLDPWD} & El directori anterior abans del canvi
(\texttt{cd}). \\
\texttt{SHELL} & El shell predeterminat de l'usuari. \\
\texttt{LANG} & Configuració regional de l'idioma. \\
\texttt{HOME} & El directori personal de l'usuari. \\
\texttt{HOSTNAME} & Nom de l'ordinador o host. \\
\texttt{TERM} & El tipus de terminal en ús (ex. \texttt{xterm},
\texttt{vt100}). \\
\end{longtable}

\begin{center}\rule{0.5\linewidth}{0.5pt}\end{center}

\paragraph{\texorpdfstring{\textbf{2. Variables del sistema relacionades
amb
processos}}{2. Variables del sistema relacionades amb processos}}\label{variables-del-sistema-relacionades-amb-processos}

\begin{longtable}[]{@{}
  >{\raggedright\arraybackslash}p{(\columnwidth - 2\tabcolsep) * \real{0.1604}}
  >{\raggedright\arraybackslash}p{(\columnwidth - 2\tabcolsep) * \real{0.8396}}@{}}
\toprule\noalign{}
\begin{minipage}[b]{\linewidth}\raggedright
\textbf{Variable}
\end{minipage} & \begin{minipage}[b]{\linewidth}\raggedright
\textbf{Descripció}
\end{minipage} \\
\midrule\noalign{}
\endhead
\bottomrule\noalign{}
\endlastfoot
\texttt{UID} & ID de l'usuari actual (numerat, ex. \texttt{0} per a
root). \\
\texttt{GID} & ID del grup principal de l'usuari. \\
\texttt{PPID} & ID del procés pare (el procés que va llançar
l'actual). \\
\texttt{PID} & ID del procés actual. \\
\texttt{\_} & L'últim argument de l'última comanda executada. \\
\end{longtable}

\begin{center}\rule{0.5\linewidth}{0.5pt}\end{center}

\paragraph{\texorpdfstring{\textbf{3. Variables especials del
shell}}{3. Variables especials del shell}}\label{variables-especials-del-shell}

Les vorem en la lliçó següent

\begin{longtable}[]{@{}
  >{\raggedright\arraybackslash}p{(\columnwidth - 2\tabcolsep) * \real{0.1604}}
  >{\raggedright\arraybackslash}p{(\columnwidth - 2\tabcolsep) * \real{0.8396}}@{}}
\toprule\noalign{}
\begin{minipage}[b]{\linewidth}\raggedright
\textbf{Variable}
\end{minipage} & \begin{minipage}[b]{\linewidth}\raggedright
\textbf{Descripció}
\end{minipage} \\
\midrule\noalign{}
\endhead
\bottomrule\noalign{}
\endlastfoot
\texttt{\$0} & El nom de l'script o la comanda executada. \\
\texttt{\$1,\ \$2,\ ...} & Paràmetres passats a l'script (primer, segon,
etc.). \\
\texttt{\$\#} & Nombre total de paràmetres passats. \\
\texttt{\$@} & Tots els paràmetres passats, com una llista separada per
espais. \\
\texttt{\$*} & Tots els paràmetres passats, com una sola cadena de
text. \\
\texttt{\$?} & Codi de sortida de l'última comanda (0 si ha tingut èxit,
un altre valor si ha fallat). \\
\texttt{\$\$} & PID del procés actual (l'script mateix). \\
\texttt{\$!} & PID de l'últim procés executat en segon pla. \\
\end{longtable}

\begin{center}\rule{0.5\linewidth}{0.5pt}\end{center}

\paragraph{\texorpdfstring{\textbf{4. Variables relacionades amb
l'execució i
l'entorn}}{4. Variables relacionades amb l'execució i l'entorn}}\label{variables-relacionades-amb-lexecuciuxf3-i-lentorn}

\begin{longtable}[]{@{}
  >{\raggedright\arraybackslash}p{(\columnwidth - 2\tabcolsep) * \real{0.1604}}
  >{\raggedright\arraybackslash}p{(\columnwidth - 2\tabcolsep) * \real{0.8396}}@{}}
\toprule\noalign{}
\begin{minipage}[b]{\linewidth}\raggedright
\textbf{Variable}
\end{minipage} & \begin{minipage}[b]{\linewidth}\raggedright
\textbf{Descripció}
\end{minipage} \\
\midrule\noalign{}
\endhead
\bottomrule\noalign{}
\endlastfoot
\texttt{ENV} & Defineix l'script que s'executa quan s'inicia el
shell. \\
\texttt{BASH\_VERSION} & Versió del shell Bash que s'està utilitzant. \\
\texttt{SHLVL} & Nivell del shell (quants shells s'han obert). \\
\texttt{RANDOM} & Retorna un número aleatori cada vegada que
s'accedeix. \\
\texttt{SECONDS} & Nombre de segons que el shell ha estat actiu. \\
\texttt{PS1} & El prompt principal del shell (predeterminat:
\texttt{\$}). \\
\end{longtable}

\begin{center}\rule{0.5\linewidth}{0.5pt}\end{center}

\subsubsection{\texorpdfstring{\textbf{Com utilitzar aquestes
variables}}{Com utilitzar aquestes variables}}\label{com-utilitzar-aquestes-variables}

\begin{enumerate}
\def\labelenumi{\arabic{enumi}.}
\item
  \textbf{Consultar el valor d'una variable:}

\begin{Shaded}
\begin{Highlighting}[]
\BuiltInTok{echo} \VariableTok{$USER}
\BuiltInTok{echo} \VariableTok{$PWD}
\end{Highlighting}
\end{Shaded}
\item
  \textbf{Modificar variables (si és permès):}
\end{enumerate}

És convenient fer una còpia abans:

\begin{Shaded}
\begin{Highlighting}[]
\VariableTok{PATH\_copia}\OperatorTok{=} \VariableTok{$PATH}
\end{Highlighting}
\end{Shaded}

Ara la modifiquem

\begin{Shaded}
\begin{Highlighting}[]
\BuiltInTok{export} \VariableTok{PATH}\OperatorTok{=}\VariableTok{$PATH}\NormalTok{:/nou/directori}
\end{Highlighting}
\end{Shaded}

\begin{enumerate}
\def\labelenumi{\arabic{enumi}.}
\setcounter{enumi}{2}
\tightlist
\item
  \textbf{Exemples pràctics:}

  \begin{itemize}
  \item
    Mostrar el directori personal:

\begin{Shaded}
\begin{Highlighting}[]
\BuiltInTok{echo} \StringTok{"El teu directori personal és: }\VariableTok{$HOME}\StringTok{"}
\end{Highlighting}
\end{Shaded}
  \item
    Mostrar els directoris del \texttt{PATH}:

\begin{Shaded}
\begin{Highlighting}[]
\BuiltInTok{echo} \StringTok{"Els directoris del PATH són: }\VariableTok{$PATH}\StringTok{"}
\end{Highlighting}
\end{Shaded}

    \subsection{export}\label{export}
  \end{itemize}
\end{enumerate}

L'ordre export en Bash s'utilitza per convertir una variable de shell en
una variable d'entorn. Això permet que estiga accessible no només al
shell actual, sinó també a tots els processos fills o sub-shells que es
generen a partir d'aquest shell.

\end{document}
