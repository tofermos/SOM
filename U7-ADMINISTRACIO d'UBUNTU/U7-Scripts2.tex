% Options for packages loaded elsewhere
\PassOptionsToPackage{unicode}{hyperref}
\PassOptionsToPackage{hyphens}{url}
%
\documentclass[
  12 pt,
  a4paper,
]{article}
\usepackage{amsmath,amssymb}
\usepackage{setspace}
\usepackage{iftex}
\ifPDFTeX
  \usepackage[T1]{fontenc}
  \usepackage[utf8]{inputenc}
  \usepackage{textcomp} % provide euro and other symbols
\else % if luatex or xetex
  \usepackage{unicode-math} % this also loads fontspec
  \defaultfontfeatures{Scale=MatchLowercase}
  \defaultfontfeatures[\rmfamily]{Ligatures=TeX,Scale=1}
\fi
\usepackage{lmodern}
\ifPDFTeX\else
  % xetex/luatex font selection
  \setmainfont[]{Times New Roman}
\fi
% Use upquote if available, for straight quotes in verbatim environments
\IfFileExists{upquote.sty}{\usepackage{upquote}}{}
\IfFileExists{microtype.sty}{% use microtype if available
  \usepackage[]{microtype}
  \UseMicrotypeSet[protrusion]{basicmath} % disable protrusion for tt fonts
}{}
\makeatletter
\@ifundefined{KOMAClassName}{% if non-KOMA class
  \IfFileExists{parskip.sty}{%
    \usepackage{parskip}
  }{% else
    \setlength{\parindent}{0pt}
    \setlength{\parskip}{6pt plus 2pt minus 1pt}}
}{% if KOMA class
  \KOMAoptions{parskip=half}}
\makeatother
\usepackage{xcolor}
\usepackage[margin=1in]{geometry}
\usepackage{color}
\usepackage{fancyvrb}
\newcommand{\VerbBar}{|}
\newcommand{\VERB}{\Verb[commandchars=\\\{\}]}
\DefineVerbatimEnvironment{Highlighting}{Verbatim}{commandchars=\\\{\}}
% Add ',fontsize=\small' for more characters per line
\usepackage{framed}
\definecolor{shadecolor}{RGB}{248,248,248}
\newenvironment{Shaded}{\begin{snugshade}}{\end{snugshade}}
\newcommand{\AlertTok}[1]{\textcolor[rgb]{0.94,0.16,0.16}{#1}}
\newcommand{\AnnotationTok}[1]{\textcolor[rgb]{0.56,0.35,0.01}{\textbf{\textit{#1}}}}
\newcommand{\AttributeTok}[1]{\textcolor[rgb]{0.13,0.29,0.53}{#1}}
\newcommand{\BaseNTok}[1]{\textcolor[rgb]{0.00,0.00,0.81}{#1}}
\newcommand{\BuiltInTok}[1]{#1}
\newcommand{\CharTok}[1]{\textcolor[rgb]{0.31,0.60,0.02}{#1}}
\newcommand{\CommentTok}[1]{\textcolor[rgb]{0.56,0.35,0.01}{\textit{#1}}}
\newcommand{\CommentVarTok}[1]{\textcolor[rgb]{0.56,0.35,0.01}{\textbf{\textit{#1}}}}
\newcommand{\ConstantTok}[1]{\textcolor[rgb]{0.56,0.35,0.01}{#1}}
\newcommand{\ControlFlowTok}[1]{\textcolor[rgb]{0.13,0.29,0.53}{\textbf{#1}}}
\newcommand{\DataTypeTok}[1]{\textcolor[rgb]{0.13,0.29,0.53}{#1}}
\newcommand{\DecValTok}[1]{\textcolor[rgb]{0.00,0.00,0.81}{#1}}
\newcommand{\DocumentationTok}[1]{\textcolor[rgb]{0.56,0.35,0.01}{\textbf{\textit{#1}}}}
\newcommand{\ErrorTok}[1]{\textcolor[rgb]{0.64,0.00,0.00}{\textbf{#1}}}
\newcommand{\ExtensionTok}[1]{#1}
\newcommand{\FloatTok}[1]{\textcolor[rgb]{0.00,0.00,0.81}{#1}}
\newcommand{\FunctionTok}[1]{\textcolor[rgb]{0.13,0.29,0.53}{\textbf{#1}}}
\newcommand{\ImportTok}[1]{#1}
\newcommand{\InformationTok}[1]{\textcolor[rgb]{0.56,0.35,0.01}{\textbf{\textit{#1}}}}
\newcommand{\KeywordTok}[1]{\textcolor[rgb]{0.13,0.29,0.53}{\textbf{#1}}}
\newcommand{\NormalTok}[1]{#1}
\newcommand{\OperatorTok}[1]{\textcolor[rgb]{0.81,0.36,0.00}{\textbf{#1}}}
\newcommand{\OtherTok}[1]{\textcolor[rgb]{0.56,0.35,0.01}{#1}}
\newcommand{\PreprocessorTok}[1]{\textcolor[rgb]{0.56,0.35,0.01}{\textit{#1}}}
\newcommand{\RegionMarkerTok}[1]{#1}
\newcommand{\SpecialCharTok}[1]{\textcolor[rgb]{0.81,0.36,0.00}{\textbf{#1}}}
\newcommand{\SpecialStringTok}[1]{\textcolor[rgb]{0.31,0.60,0.02}{#1}}
\newcommand{\StringTok}[1]{\textcolor[rgb]{0.31,0.60,0.02}{#1}}
\newcommand{\VariableTok}[1]{\textcolor[rgb]{0.00,0.00,0.00}{#1}}
\newcommand{\VerbatimStringTok}[1]{\textcolor[rgb]{0.31,0.60,0.02}{#1}}
\newcommand{\WarningTok}[1]{\textcolor[rgb]{0.56,0.35,0.01}{\textbf{\textit{#1}}}}
\usepackage{longtable,booktabs,array}
\usepackage{calc} % for calculating minipage widths
% Correct order of tables after \paragraph or \subparagraph
\usepackage{etoolbox}
\makeatletter
\patchcmd\longtable{\par}{\if@noskipsec\mbox{}\fi\par}{}{}
\makeatother
% Allow footnotes in longtable head/foot
\IfFileExists{footnotehyper.sty}{\usepackage{footnotehyper}}{\usepackage{footnote}}
\makesavenoteenv{longtable}
\usepackage{graphicx}
\makeatletter
\def\maxwidth{\ifdim\Gin@nat@width>\linewidth\linewidth\else\Gin@nat@width\fi}
\def\maxheight{\ifdim\Gin@nat@height>\textheight\textheight\else\Gin@nat@height\fi}
\makeatother
% Scale images if necessary, so that they will not overflow the page
% margins by default, and it is still possible to overwrite the defaults
% using explicit options in \includegraphics[width, height, ...]{}
\setkeys{Gin}{width=\maxwidth,height=\maxheight,keepaspectratio}
% Set default figure placement to htbp
\makeatletter
\def\fps@figure{htbp}
\makeatother
\setlength{\emergencystretch}{3em} % prevent overfull lines
\providecommand{\tightlist}{%
  \setlength{\itemsep}{0pt}\setlength{\parskip}{0pt}}
\setcounter{secnumdepth}{-\maxdimen} % remove section numbering
\ifLuaTeX
\usepackage[bidi=basic]{babel}
\else
\usepackage[bidi=default]{babel}
\fi
\babelprovide[main,import]{spanish}
\ifPDFTeX
\else
\babelfont{rm}[]{Times New Roman}
\fi
% get rid of language-specific shorthands (see #6817):
\let\LanguageShortHands\languageshorthands
\def\languageshorthands#1{}
\ifLuaTeX
  \usepackage{selnolig}  % disable illegal ligatures
\fi
\usepackage{bookmark}
\IfFileExists{xurl.sty}{\usepackage{xurl}}{} % add URL line breaks if available
\urlstyle{same}
\hypersetup{
  pdfauthor={Tomàs Ferrandis Moscardó},
  pdflang={es-ES},
  hidelinks,
  pdfcreator={LaTeX via pandoc}}

\title{U7- ADMINISTRACIÓ D'UBUNTU}
\usepackage{etoolbox}
\makeatletter
\providecommand{\subtitle}[1]{% add subtitle to \maketitle
  \apptocmd{\@title}{\par {\large #1 \par}}{}{}
}
\makeatother
\subtitle{~BUCLES EN ELS SCRIPTS}
\author{Tomàs Ferrandis Moscardó}
\date{}

\begin{document}
\maketitle

\setstretch{1.5}
En Bash, els bucles \textbf{\texttt{for}} i
\textbf{\texttt{for\ ...\ in}} són essencialment dues formes del mateix
concepte. No obstant això, la seva sintaxi i ús poden diferir
lleugerament segons el context.

\section{\texorpdfstring{\textbf{El bucle \texttt{for} en
Bash}}{El bucle for en Bash}}\label{el-bucle-for-en-bash}

\subsection{\texorpdfstring{\textbf{1.
\texttt{for\ ...\ in}}}{1. for ... in}}\label{for-...-in}

Aquest és el mètode més comú i flexible per fer un bucle \texttt{for} a
Bash. Permet iterar sobre una llista explícita, una seqüència numèrica o
el resultat d'una comanda.

\subsubsection{\texorpdfstring{\textbf{Sintaxi}:}{Sintaxi:}}\label{sintaxi}

\begin{Shaded}
\begin{Highlighting}[]
\ControlFlowTok{for}\NormalTok{ variable }\KeywordTok{in}\NormalTok{ valor1 valor2 valor3}\KeywordTok{;} \ControlFlowTok{do}
    \CommentTok{\# Accions amb la variable}
\ControlFlowTok{done}
\end{Highlighting}
\end{Shaded}

\subsubsection{\texorpdfstring{\textbf{Exemples}:}{Exemples:}}\label{exemples}

\paragraph{Iterar sobre una llista
fixa:}\label{iterar-sobre-una-llista-fixa}

\begin{Shaded}
\begin{Highlighting}[]
\CommentTok{\#!/bin/bash}
\ControlFlowTok{for}\NormalTok{ fruita }\KeywordTok{in}\NormalTok{ poma taronja raïm}\KeywordTok{;} \ControlFlowTok{do}
    \BuiltInTok{echo} \StringTok{"M\textquotesingle{}agrada la }\VariableTok{$fruita}\StringTok{!"}
\ControlFlowTok{done}
\end{Highlighting}
\end{Shaded}

\textbf{Sortida:}

\begin{verbatim}
M'agrada la poma!
M'agrada la taronja!
M'agrada la raïm!
\end{verbatim}

\paragraph{Iterar sobre una seqüència de
números:}\label{iterar-sobre-una-sequxfcuxe8ncia-de-nuxfameros}

\begin{Shaded}
\begin{Highlighting}[]
\CommentTok{\#!/bin/bash}
\ControlFlowTok{for}\NormalTok{ num }\KeywordTok{in} \DataTypeTok{\{}\DecValTok{1}\DataTypeTok{..}\DecValTok{5}\DataTypeTok{\}}\KeywordTok{;} \ControlFlowTok{do}
    \BuiltInTok{echo} \StringTok{"Número: }\VariableTok{$num}\StringTok{"}
\ControlFlowTok{done}
\end{Highlighting}
\end{Shaded}

\textbf{Sortida:}

\begin{verbatim}
Número: 1
Número: 2
Número: 3
Número: 4
Número: 5
\end{verbatim}

\paragraph{Iterar sobre resultats d'una
comanda:}\label{iterar-sobre-resultats-duna-comanda}

\begin{Shaded}
\begin{Highlighting}[]
\CommentTok{\#!/bin/bash}
\ControlFlowTok{for}\NormalTok{ fitxer }\KeywordTok{in} \VariableTok{$(}\FunctionTok{ls} \PreprocessorTok{*}\NormalTok{.txt}\VariableTok{)}\KeywordTok{;} \ControlFlowTok{do}
    \BuiltInTok{echo} \StringTok{"Fitxer de text: }\VariableTok{$fitxer}\StringTok{"}
\ControlFlowTok{done}
\end{Highlighting}
\end{Shaded}

\textbf{Sortida (depèn del sistema):}

\begin{verbatim}
Fitxer de text: document1.txt
Fitxer de text: llista.txt
Fitxer de text: notes.txt
\end{verbatim}

\begin{center}\rule{0.5\linewidth}{0.5pt}\end{center}

\subsection{\texorpdfstring{\textbf{2. \texttt{for} (sense
\texttt{in})}}{2. for (sense in)}}\label{for-sense-in}

Quan no es proporciona cap llista després del \texttt{for}, Bash
assumeix una seqüència implícita (\texttt{\{1..N\}} o una seqüència
numèrica generada d'altres maneres). Aquest mètode s'utilitza
principalment en combinació amb \texttt{C-style\ for\ loops}.

\subsubsection{\texorpdfstring{\textbf{Sintaxi}:}{Sintaxi:}}\label{sintaxi-1}

\begin{Shaded}
\begin{Highlighting}[]
\ControlFlowTok{for} \KeywordTok{((} \VariableTok{inicialitzaci}\NormalTok{ó}\KeywordTok{;} \VariableTok{condici}\NormalTok{ó}\KeywordTok{;} \VariableTok{increment} \KeywordTok{));} \ControlFlowTok{do}
    \CommentTok{\# Accions dins del bucle}
\ControlFlowTok{done}
\end{Highlighting}
\end{Shaded}

\subsubsection{\texorpdfstring{\textbf{Exemple}:}{Exemple:}}\label{exemple}

\paragraph{Bucles d'estil C:}\label{bucles-destil-c}

\begin{Shaded}
\begin{Highlighting}[]
\CommentTok{\#!/bin/bash}
\ControlFlowTok{for} \KeywordTok{((} \VariableTok{i}\OperatorTok{=}\DecValTok{1}\KeywordTok{;} \VariableTok{i}\OperatorTok{\textless{}=}\DecValTok{5}\KeywordTok{;} \VariableTok{i}\OperatorTok{++} \KeywordTok{));} \ControlFlowTok{do}
    \BuiltInTok{echo} \StringTok{"Número: }\VariableTok{$i}\StringTok{"}
\ControlFlowTok{done}
\end{Highlighting}
\end{Shaded}

\textbf{Sortida:}

\begin{verbatim}
Número: 1
Número: 2
Número: 3
Número: 4
Número: 5
\end{verbatim}

\begin{center}\rule{0.5\linewidth}{0.5pt}\end{center}

\section{\texorpdfstring{\textbf{Comparació entre \texttt{for} i
\texttt{for\ ...\ in}}}{Comparació entre for i for ... in}}\label{comparaciuxf3-entre-for-i-for-...-in}

\begin{longtable}[]{@{}
  >{\raggedright\arraybackslash}p{(\columnwidth - 4\tabcolsep) * \real{0.2393}}
  >{\raggedright\arraybackslash}p{(\columnwidth - 4\tabcolsep) * \real{0.4017}}
  >{\raggedright\arraybackslash}p{(\columnwidth - 4\tabcolsep) * \real{0.3590}}@{}}
\toprule\noalign{}
\begin{minipage}[b]{\linewidth}\raggedright
Característica
\end{minipage} & \begin{minipage}[b]{\linewidth}\raggedright
\texttt{for\ ...\ in}
\end{minipage} & \begin{minipage}[b]{\linewidth}\raggedright
\texttt{for} (estil C)
\end{minipage} \\
\midrule\noalign{}
\endhead
\bottomrule\noalign{}
\endlastfoot
\textbf{Iteració explícita} & Itera sobre una llista o seqüència donada.
& Control total sobre inicialització i increment. \\
\textbf{Usos comuns} & Treballar amb llistes o sortides de comandes. &
Bucle numèric controlat manualment. \\
\textbf{Simplicitat} & Més senzill i llegible per llistes. & Més
complex, però més flexible en casos numèrics. \\
\textbf{Exemple típic} & \texttt{for\ item\ in\ a\ b\ c;\ do\ ...\ done}
& \texttt{for\ ((\ i=1;\ i\textless{}=10;\ i++\ ));\ do\ ...\ done} \\
\end{longtable}

\begin{center}\rule{0.5\linewidth}{0.5pt}\end{center}

\section{\texorpdfstring{\textbf{Quan utilitzar cada
tipus?}}{Quan utilitzar cada tipus?}}\label{quan-utilitzar-cada-tipus}

\begin{itemize}
\tightlist
\item
  \textbf{\texttt{for\ ...\ in}}:

  \begin{itemize}
  \tightlist
  \item
    Ideal per treballar amb llistes definides o resultats de comandes.
  \item
    Exemple: Processar fitxers, iterar sobre paraules, etc.
  \end{itemize}
\item
  \textbf{\texttt{for} (estil C)}:

  \begin{itemize}
  \tightlist
  \item
    Ideal per a bucles numèrics o quan necessites més control sobre les
    condicions d'inici, finalització i increment.
  \end{itemize}
\end{itemize}

\end{document}
