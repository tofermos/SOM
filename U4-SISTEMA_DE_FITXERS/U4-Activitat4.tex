% Options for packages loaded elsewhere
\PassOptionsToPackage{unicode}{hyperref}
\PassOptionsToPackage{hyphens}{url}
%
\documentclass[
  12 pt,
  a4paper,
]{article}
\usepackage{amsmath,amssymb}
\usepackage{setspace}
\usepackage{iftex}
\ifPDFTeX
  \usepackage[T1]{fontenc}
  \usepackage[utf8]{inputenc}
  \usepackage{textcomp} % provide euro and other symbols
\else % if luatex or xetex
  \usepackage{unicode-math} % this also loads fontspec
  \defaultfontfeatures{Scale=MatchLowercase}
  \defaultfontfeatures[\rmfamily]{Ligatures=TeX,Scale=1}
\fi
\usepackage{lmodern}
\ifPDFTeX\else
  % xetex/luatex font selection
  \setmainfont[]{Times New Roman}
\fi
% Use upquote if available, for straight quotes in verbatim environments
\IfFileExists{upquote.sty}{\usepackage{upquote}}{}
\IfFileExists{microtype.sty}{% use microtype if available
  \usepackage[]{microtype}
  \UseMicrotypeSet[protrusion]{basicmath} % disable protrusion for tt fonts
}{}
\makeatletter
\@ifundefined{KOMAClassName}{% if non-KOMA class
  \IfFileExists{parskip.sty}{%
    \usepackage{parskip}
  }{% else
    \setlength{\parindent}{0pt}
    \setlength{\parskip}{6pt plus 2pt minus 1pt}}
}{% if KOMA class
  \KOMAoptions{parskip=half}}
\makeatother
\usepackage{xcolor}
\usepackage[margin=1in]{geometry}
\usepackage{longtable,booktabs,array}
\usepackage{calc} % for calculating minipage widths
% Correct order of tables after \paragraph or \subparagraph
\usepackage{etoolbox}
\makeatletter
\patchcmd\longtable{\par}{\if@noskipsec\mbox{}\fi\par}{}{}
\makeatother
% Allow footnotes in longtable head/foot
\IfFileExists{footnotehyper.sty}{\usepackage{footnotehyper}}{\usepackage{footnote}}
\makesavenoteenv{longtable}
\usepackage{graphicx}
\makeatletter
\def\maxwidth{\ifdim\Gin@nat@width>\linewidth\linewidth\else\Gin@nat@width\fi}
\def\maxheight{\ifdim\Gin@nat@height>\textheight\textheight\else\Gin@nat@height\fi}
\makeatother
% Scale images if necessary, so that they will not overflow the page
% margins by default, and it is still possible to overwrite the defaults
% using explicit options in \includegraphics[width, height, ...]{}
\setkeys{Gin}{width=\maxwidth,height=\maxheight,keepaspectratio}
% Set default figure placement to htbp
\makeatletter
\def\fps@figure{htbp}
\makeatother
\setlength{\emergencystretch}{3em} % prevent overfull lines
\providecommand{\tightlist}{%
  \setlength{\itemsep}{0pt}\setlength{\parskip}{0pt}}
\setcounter{secnumdepth}{-\maxdimen} % remove section numbering
\ifLuaTeX
\usepackage[bidi=basic]{babel}
\else
\usepackage[bidi=default]{babel}
\fi
\babelprovide[main,import]{spanish}
\ifPDFTeX
\else
\babelfont{rm}[]{Times New Roman}
\fi
% get rid of language-specific shorthands (see #6817):
\let\LanguageShortHands\languageshorthands
\def\languageshorthands#1{}
\ifLuaTeX
  \usepackage{selnolig}  % disable illegal ligatures
\fi
\usepackage{bookmark}
\IfFileExists{xurl.sty}{\usepackage{xurl}}{} % add URL line breaks if available
\urlstyle{same}
\hypersetup{
  pdfauthor={tofermos},
  pdflang={es-ES},
  hidelinks,
  pdfcreator={LaTeX via pandoc}}

\title{U4: Sistema de fitxers (I)}
\author{tofermos}
\date{}

\begin{document}
\maketitle

{
\setcounter{tocdepth}{2}
\tableofcontents
}
\setstretch{1.5}
\newpage
\renewcommand\tablename{Tabla}

\begin{center}\rule{0.5\linewidth}{0.5pt}\end{center}

\subsection{Exercicis sobre fitxers}\label{exercicis-sobre-fitxers}

\begin{enumerate}
\def\labelenumi{\arabic{enumi}.}
\tightlist
\item
  Llista fitxers del directori actual i, després, el del pare.
\item
  Crea \texttt{document.txt} a \texttt{/home/usuari/Documents}.\\
\item
  Mou \texttt{document.txt} a \texttt{Baixades}.\\
\item
  Elimina \texttt{document.txt}.\\
\item
  Crea \texttt{informe.txt}.\\
\item
  Mostra el contingut de \texttt{informe.txt}.\\
\item
  Consulta \texttt{informe.txt} paginat.
\item
  Crea còpia de \texttt{informe.txt} com a \texttt{copia\_informe.txt}
\item
  Mou \texttt{copia\_informe.txt} a \texttt{/tmp}.\\
\item
  Elimina \texttt{copia\_informe.txt}.\\
\item
  Crea una carpeta \texttt{Projecte} dins de Documents
\item
  Crea \texttt{items} dins \texttt{Projecte}.\\
\item
  Elimina \texttt{items} dins \texttt{Projecte}.\\
\item
  Navega a \texttt{/home/usuari/Documents}.\\
\item
  Llista \texttt{/etc}.\\
\item
  Llista \texttt{Projecte}.\\
\item
  Canvia a \texttt{/tmp}.\\
\item
  Crea \texttt{Desenvolupament} dins \texttt{/home/usuari/Documents}.\\
\item
  Elimina \texttt{Desenvolupament}.\\
\item
  Canvia a \texttt{Documents} amb \textbf{ruta absoluta}.\\
\item
  Canvia a \texttt{Documents} amb \textbf{ruta relativa}.\\
\item
  Llista \texttt{/var/log} amb \textbf{ruta absoluta}.\\
\item
  Llista \texttt{logs} amb \textbf{ruta relativa}.\\
\item
  Classifica rutes com relatives o absolutes.
\end{enumerate}

\begin{longtable}[]{@{}ll@{}}
\toprule\noalign{}
Ruta & Tipus \\
\midrule\noalign{}
\endhead
\bottomrule\noalign{}
\endlastfoot
../papers & \\
./fitxer1.txt & \\
/home/tomas & \\
Documents/curs2024 & \\
\end{longtable}

\begin{enumerate}
\def\labelenumi{\arabic{enumi}.}
\setcounter{enumi}{25}
\tightlist
\item
  Crea un enllaç simbòlic a \texttt{document.txt}.\\
\item
  Comprova si el nombre d'inodes ha canviat després de crear l'enllaç
  simbòlic. (stat o ls -li)
\item
  Crea un enllaç dur a \texttt{document.txt}.\\
\item
  Comprova si el nombre d'inodes ha augmentat després de crear l'enllaç
  dur.
\item
  Mostra informació detallada d'un fitxer amb \texttt{stat}.
\item
  Mostra el número d'inode de l'enllaç simbòlic.\\
\item
  Comprova si un enllaç dur comparteix inode amb l'original.
\item
  Elimina el primer enllaç i repeteix la comprovació
\item
  Elimina el segon enllaç i repeteix la comprovació
\item
  Crea un fitxer buit i fes dos enllaços durs
\item
  Fes un stat dels tres. Explica les diferències.
\item
  Elimina el fitxer ``original''. Fes un stat dels dos que queden i
  explica els canvis que veus.
\end{enumerate}

\end{document}
