% Options for packages loaded elsewhere
\PassOptionsToPackage{unicode}{hyperref}
\PassOptionsToPackage{hyphens}{url}
%
\documentclass[
  a4paper,
]{article}
\usepackage{amsmath,amssymb}
\usepackage{setspace}
\usepackage{iftex}
\ifPDFTeX
  \usepackage[T1]{fontenc}
  \usepackage[utf8]{inputenc}
  \usepackage{textcomp} % provide euro and other symbols
\else % if luatex or xetex
  \usepackage{unicode-math} % this also loads fontspec
  \defaultfontfeatures{Scale=MatchLowercase}
  \defaultfontfeatures[\rmfamily]{Ligatures=TeX,Scale=1}
\fi
\usepackage{lmodern}
\ifPDFTeX\else
  % xetex/luatex font selection
\fi
% Use upquote if available, for straight quotes in verbatim environments
\IfFileExists{upquote.sty}{\usepackage{upquote}}{}
\IfFileExists{microtype.sty}{% use microtype if available
  \usepackage[]{microtype}
  \UseMicrotypeSet[protrusion]{basicmath} % disable protrusion for tt fonts
}{}
\makeatletter
\@ifundefined{KOMAClassName}{% if non-KOMA class
  \IfFileExists{parskip.sty}{%
    \usepackage{parskip}
  }{% else
    \setlength{\parindent}{0pt}
    \setlength{\parskip}{6pt plus 2pt minus 1pt}}
}{% if KOMA class
  \KOMAoptions{parskip=half}}
\makeatother
\usepackage{xcolor}
\usepackage[margin=1in]{geometry}
\usepackage{graphicx}
\makeatletter
\def\maxwidth{\ifdim\Gin@nat@width>\linewidth\linewidth\else\Gin@nat@width\fi}
\def\maxheight{\ifdim\Gin@nat@height>\textheight\textheight\else\Gin@nat@height\fi}
\makeatother
% Scale images if necessary, so that they will not overflow the page
% margins by default, and it is still possible to overwrite the defaults
% using explicit options in \includegraphics[width, height, ...]{}
\setkeys{Gin}{width=\maxwidth,height=\maxheight,keepaspectratio}
% Set default figure placement to htbp
\makeatletter
\def\fps@figure{htbp}
\makeatother
\setlength{\emergencystretch}{3em} % prevent overfull lines
\providecommand{\tightlist}{%
  \setlength{\itemsep}{0pt}\setlength{\parskip}{0pt}}
\setcounter{secnumdepth}{-\maxdimen} % remove section numbering
\ifLuaTeX
\usepackage[bidi=basic]{babel}
\else
\usepackage[bidi=default]{babel}
\fi
\babelprovide[main,import]{catalan}
% get rid of language-specific shorthands (see #6817):
\let\LanguageShortHands\languageshorthands
\def\languageshorthands#1{}
\ifLuaTeX
  \usepackage{selnolig}  % disable illegal ligatures
\fi
\usepackage{bookmark}
\IfFileExists{xurl.sty}{\usepackage{xurl}}{} % add URL line breaks if available
\urlstyle{same}
\hypersetup{
  pdftitle={ACTIVITAT BOTIGA (Permisos i propietaris)},
  pdfauthor={@tofermos 2024},
  pdflang={ca-ES},
  hidelinks,
  pdfcreator={LaTeX via pandoc}}

\title{ACTIVITAT BOTIGA (Permisos i propietaris)}
\author{@tofermos 2024}
\date{}

\begin{document}
\maketitle

{
\setcounter{tocdepth}{2}
\tableofcontents
}
\setstretch{1.5}
\newpage
\renewcommand\tablename{Tabla}

\section{Activitat 1: PC mostrador botiga amb 2
usuaris}\label{activitat-1-pc-mostrador-botiga-amb-2-usuaris}

\subsection{Descripció del problema}\label{descripciuxf3-del-problema}

Tenim un PC en un mostrador d'una botiga de telefonia mòbil on treballen
2 técnics-venedors i anem a fer la FCT-Dual. Cadascun treballarem en un
torn diferent i necessitem compartir informació. Per a què quede
contància de què fa cadascun de nosaltres hem de crear un usuari
diferent per a cadascú:

\begin{itemize}
\tightlist
\item
  tecnicMati
\item
  tecnicVesprada
\item
  tecnicFCTDual
\end{itemize}

Veiem que, en crear-los, el sistema els ha assignat un grup per defecte
però nosaltres necessitem per administrar recursos organitzar-nos. Convé
que els incloem a tots al mateix grup així usarem el grup per assignar
permisos a carpetes, per exemple i no un a un:

\begin{itemize}
\tightlist
\item
  gr\_tecnics
\end{itemize}

Crearem una carpeta on compartirem documents entre els tres en
\emph{\texttt{/home/catalegs}}

Ningú més ha de poder llegir els fitxers d'esta carpeta.

\subsection{Indicacions a seguir.}\label{indicacions-a-seguir.}

Resol i contesta:

1- Crea els usuaris. Fes un inici de sessió en cadascun

2- Crea el grup i assigna els usuaris amb \emph{usermod}

3- Crea la carpeta amb \emph{\texttt{mkdir}} o el GUI. Indica els
problemes que trobes i como ho soluciones.

4- Intenta entrar a la carpeta amb qualsevol usuari i crear un fitxer

5- Indica el problema i com ho soluciones.

6- Torna a provar

\end{document}
