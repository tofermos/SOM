% Options for packages loaded elsewhere
\PassOptionsToPackage{unicode}{hyperref}
\PassOptionsToPackage{hyphens}{url}
%
\documentclass[
  12 pt,
  a4paper,
]{article}
\usepackage{amsmath,amssymb}
\usepackage{setspace}
\usepackage{iftex}
\ifPDFTeX
  \usepackage[T1]{fontenc}
  \usepackage[utf8]{inputenc}
  \usepackage{textcomp} % provide euro and other symbols
\else % if luatex or xetex
  \usepackage{unicode-math} % this also loads fontspec
  \defaultfontfeatures{Scale=MatchLowercase}
  \defaultfontfeatures[\rmfamily]{Ligatures=TeX,Scale=1}
\fi
\usepackage{lmodern}
\ifPDFTeX\else
  % xetex/luatex font selection
  \setmainfont[]{Times New Roman}
\fi
% Use upquote if available, for straight quotes in verbatim environments
\IfFileExists{upquote.sty}{\usepackage{upquote}}{}
\IfFileExists{microtype.sty}{% use microtype if available
  \usepackage[]{microtype}
  \UseMicrotypeSet[protrusion]{basicmath} % disable protrusion for tt fonts
}{}
\makeatletter
\@ifundefined{KOMAClassName}{% if non-KOMA class
  \IfFileExists{parskip.sty}{%
    \usepackage{parskip}
  }{% else
    \setlength{\parindent}{0pt}
    \setlength{\parskip}{6pt plus 2pt minus 1pt}}
}{% if KOMA class
  \KOMAoptions{parskip=half}}
\makeatother
\usepackage{xcolor}
\usepackage[margin=1in]{geometry}
\usepackage{longtable,booktabs,array}
\usepackage{calc} % for calculating minipage widths
% Correct order of tables after \paragraph or \subparagraph
\usepackage{etoolbox}
\makeatletter
\patchcmd\longtable{\par}{\if@noskipsec\mbox{}\fi\par}{}{}
\makeatother
% Allow footnotes in longtable head/foot
\IfFileExists{footnotehyper.sty}{\usepackage{footnotehyper}}{\usepackage{footnote}}
\makesavenoteenv{longtable}
\usepackage{graphicx}
\makeatletter
\def\maxwidth{\ifdim\Gin@nat@width>\linewidth\linewidth\else\Gin@nat@width\fi}
\def\maxheight{\ifdim\Gin@nat@height>\textheight\textheight\else\Gin@nat@height\fi}
\makeatother
% Scale images if necessary, so that they will not overflow the page
% margins by default, and it is still possible to overwrite the defaults
% using explicit options in \includegraphics[width, height, ...]{}
\setkeys{Gin}{width=\maxwidth,height=\maxheight,keepaspectratio}
% Set default figure placement to htbp
\makeatletter
\def\fps@figure{htbp}
\makeatother
\setlength{\emergencystretch}{3em} % prevent overfull lines
\providecommand{\tightlist}{%
  \setlength{\itemsep}{0pt}\setlength{\parskip}{0pt}}
\setcounter{secnumdepth}{-\maxdimen} % remove section numbering
\ifLuaTeX
\usepackage[bidi=basic]{babel}
\else
\usepackage[bidi=default]{babel}
\fi
\babelprovide[main,import]{spanish}
\ifPDFTeX
\else
\babelfont{rm}[]{Times New Roman}
\fi
% get rid of language-specific shorthands (see #6817):
\let\LanguageShortHands\languageshorthands
\def\languageshorthands#1{}
\ifLuaTeX
  \usepackage{selnolig}  % disable illegal ligatures
\fi
\usepackage{bookmark}
\IfFileExists{xurl.sty}{\usepackage{xurl}}{} % add URL line breaks if available
\urlstyle{same}
\hypersetup{
  pdflang={es-ES},
  hidelinks,
  pdfcreator={LaTeX via pandoc}}

\title{U8 Tipus d'instal·lacions i consideracions prèvies}
\usepackage{etoolbox}
\makeatletter
\providecommand{\subtitle}[1]{% add subtitle to \maketitle
  \apptocmd{\@title}{\par {\large #1 \par}}{}{}
}
\makeatother
\subtitle{Informació sobre la instal·lació}
\author{}
\date{\vspace{-2.5em}}

\begin{document}
\maketitle

{
\setcounter{tocdepth}{2}
\tableofcontents
}
\setstretch{1.5}
\newpage
\renewcommand\tablename{Tabla}

\begin{center}\rule{0.5\linewidth}{0.5pt}\end{center}

\section{1 Tipus d'instal·lacions}\label{tipus-dinstallacions}

\begin{itemize}
\item
  Instal·lació neta: Esborra tot el disc i instal·la el sistema des de
  zero.
\item
  Actualització des de Windows 10: Conserva dades i programes.
\item
  Instal·lació en màquina virtual: Per provar Windows 11 sense modificar
  el sistema actual. ( La que fem en VirtualBox)
\item
  Instal·lació en un altre disc o partició: Per tenir Windows 11 al
  costat d'un altre sistema operatiu (dual boot: si volem en un PC
  tindre Windows 11 i Linux, per exemple)
\end{itemize}

\section{2. Consideracions prèvies abans de la
instal·lació}\label{consideracions-pruxe8vies-abans-de-la-installaciuxf3}

\begin{itemize}
\item
  Verificar compatibilitat del hardware ( maquinari) amb la ferramenta
  PC Health Check de Microsoft.
\item
  Fer una còpia de seguretat de les dades per evitar pèrdues.
\item
  Assegurar-se que el BIOS/UEFI està configurat correctament (Secure
  Boot i TPM activats).
\item
  Decidir quin tipus d'instal·lació fer: neta, actualització o màquina
  virtual.
\item
  Disposar d'una connexió a internet estable per descarregar
  actualitzacions i controladors.
\end{itemize}

\section{3. Instal·lació manual de Windows
11}\label{installaciuxf3-manual-de-windows-11}

\textbf{Este apartat 3 NO CAL estudiar-lo perquè estem veient com fer-ho
amb VENTOY}

\begin{enumerate}
\def\labelenumi{\arabic{enumi}.}
\item
  Descarregar l'eina de creació de suports de Microsoft (Media Creation
  Tool).
\item
  Crear un USB d'instal·lació (mínim 8 GB) amb l'eina o mitjançant
  Rufus.
\item
  Accedir a la BIOS/UEFI i configurar l'ordre d'arrencada perquè el PC
  iniciï des del USB.
\item
  Iniciar l'instal·lador i seguir els passos:

  \begin{itemize}
  \tightlist
  \item
    Seleccionar idioma i configuració regional.\\
  \item
    Triar ``Instal·lació personalitzada'' per fer una instal·lació
    neta.\\
  \item
    Esborrar particions antigues (opcional) i instal·lar Windows 11 en
    l'espai lliure.\\
  \end{itemize}
\item
  Finalitzar la instal·lació i configurar el compte d'usuari.
\end{enumerate}

\begin{center}\rule{0.5\linewidth}{0.5pt}\end{center}

\section{4. Actualització del sistema
operatiu}\label{actualitzaciuxf3-del-sistema-operatiu}

Si tens Windows 10 i el teu dispositiu és compatible, pots
actualitzar-lo sense perdre dades.

\subsection{Mètodes d'actualització}\label{muxe8todes-dactualitzaciuxf3}

\begin{itemize}
\item
  Via Windows Update:

  \begin{itemize}
  \item
    Anar a Configuració → Actualització i seguretat → Windows Update.
  \item
    Si apareix l'opció d'actualitzar a Windows 11, seguir les
    instruccions.
  \end{itemize}
\item
  Amb l'Assistent d'actualització de Windows 11:

  \begin{itemize}
  \tightlist
  \item
    Descarregar des del web oficial de Microsoft i seguir els passos.
  \end{itemize}
\end{itemize}

Avantatge: Manté fitxers i programes.\\
Desavantatge: Pot haver-hi problemes de compatibilitat amb controladors
antics.

\section{5. Documentació respecte a la instal·lació i
incidències}\label{documentaciuxf3-respecte-a-la-installaciuxf3-i-inciduxe8ncies}

\subsection{Documents útils sobre Windows
11}\label{documents-uxfatils-sobre-windows-11}

\begin{itemize}
\item
  Guia oficial de Microsoft:
  \href{https://support.microsoft.com}{support.microsoft.com}
\item
  Fòrum d'ajuda de Microsoft:
  \href{https://answers.microsoft.com}{answers.microsoft.com}
\item
  Documents tècnics sobre compatibilitat i instal·lació:

  \begin{itemize}
  \item
    \href{https://docs.microsoft.com/en-us/windows-hardware/}{Windows
    Hardware Certification}
  \item
    \href{https://docs.microsoft.com/en-us/windows/deployment/}{Windows
    Deployment Toolkit}
  \end{itemize}
\end{itemize}

\subsection{Incidències més comunes durant la instal·lació i com
solucionar-les}\label{inciduxe8ncies-muxe9s-comunes-durant-la-installaciuxf3-i-com-solucionar-les}

\begin{longtable}[]{@{}
  >{\raggedright\arraybackslash}p{(\columnwidth - 2\tabcolsep) * \real{0.5200}}
  >{\raggedright\arraybackslash}p{(\columnwidth - 2\tabcolsep) * \real{0.4800}}@{}}
\toprule\noalign{}
\begin{minipage}[b]{\linewidth}\raggedright
Problema
\end{minipage} & \begin{minipage}[b]{\linewidth}\raggedright
Solució
\end{minipage} \\
\midrule\noalign{}
\endhead
\bottomrule\noalign{}
\endlastfoot
No es pot instal·lar per falta de TPM 2.0 & Activar TPM des de la
BIOS/UEFI \\
Windows 11 no apareix a Windows Update & Descarregar i utilitzar
l'Assistent d'Actualització \\
Error ``No es pot instal·lar en aquest PC'' & Comprovar compatibilitat
amb PC Health Check \\
Falta d'espai en disc & Alliberar espai o instal·lar en un altre disc \\
\end{longtable}

\section{Conclusió}\label{conclusiuxf3}

Windows 11 representa un pas endavant en seguretat i rendiment, però
també imposa nous requisits de maquinari (hardware). Seguir una guia
estructurada per a la instal·lació ajuda a evitar problemes i optimitzar
el sistema.

\end{document}
