% Options for packages loaded elsewhere
\PassOptionsToPackage{unicode}{hyperref}
\PassOptionsToPackage{hyphens}{url}
%
\documentclass[
  12 pt,
  a4paper,
]{article}
\usepackage{amsmath,amssymb}
\usepackage{setspace}
\usepackage{iftex}
\ifPDFTeX
  \usepackage[T1]{fontenc}
  \usepackage[utf8]{inputenc}
  \usepackage{textcomp} % provide euro and other symbols
\else % if luatex or xetex
  \usepackage{unicode-math} % this also loads fontspec
  \defaultfontfeatures{Scale=MatchLowercase}
  \defaultfontfeatures[\rmfamily]{Ligatures=TeX,Scale=1}
\fi
\usepackage{lmodern}
\ifPDFTeX\else
  % xetex/luatex font selection
  \setmainfont[]{Times New Roman}
\fi
% Use upquote if available, for straight quotes in verbatim environments
\IfFileExists{upquote.sty}{\usepackage{upquote}}{}
\IfFileExists{microtype.sty}{% use microtype if available
  \usepackage[]{microtype}
  \UseMicrotypeSet[protrusion]{basicmath} % disable protrusion for tt fonts
}{}
\makeatletter
\@ifundefined{KOMAClassName}{% if non-KOMA class
  \IfFileExists{parskip.sty}{%
    \usepackage{parskip}
  }{% else
    \setlength{\parindent}{0pt}
    \setlength{\parskip}{6pt plus 2pt minus 1pt}}
}{% if KOMA class
  \KOMAoptions{parskip=half}}
\makeatother
\usepackage{xcolor}
\usepackage[margin=1in]{geometry}
\usepackage{longtable,booktabs,array}
\usepackage{calc} % for calculating minipage widths
% Correct order of tables after \paragraph or \subparagraph
\usepackage{etoolbox}
\makeatletter
\patchcmd\longtable{\par}{\if@noskipsec\mbox{}\fi\par}{}{}
\makeatother
% Allow footnotes in longtable head/foot
\IfFileExists{footnotehyper.sty}{\usepackage{footnotehyper}}{\usepackage{footnote}}
\makesavenoteenv{longtable}
\usepackage{graphicx}
\makeatletter
\def\maxwidth{\ifdim\Gin@nat@width>\linewidth\linewidth\else\Gin@nat@width\fi}
\def\maxheight{\ifdim\Gin@nat@height>\textheight\textheight\else\Gin@nat@height\fi}
\makeatother
% Scale images if necessary, so that they will not overflow the page
% margins by default, and it is still possible to overwrite the defaults
% using explicit options in \includegraphics[width, height, ...]{}
\setkeys{Gin}{width=\maxwidth,height=\maxheight,keepaspectratio}
% Set default figure placement to htbp
\makeatletter
\def\fps@figure{htbp}
\makeatother
\setlength{\emergencystretch}{3em} % prevent overfull lines
\providecommand{\tightlist}{%
  \setlength{\itemsep}{0pt}\setlength{\parskip}{0pt}}
\setcounter{secnumdepth}{-\maxdimen} % remove section numbering
\ifLuaTeX
\usepackage[bidi=basic]{babel}
\else
\usepackage[bidi=default]{babel}
\fi
\babelprovide[main,import]{spanish}
\ifPDFTeX
\else
\babelfont{rm}[]{Times New Roman}
\fi
% get rid of language-specific shorthands (see #6817):
\let\LanguageShortHands\languageshorthands
\def\languageshorthands#1{}
\ifLuaTeX
  \usepackage{selnolig}  % disable illegal ligatures
\fi
\usepackage{bookmark}
\IfFileExists{xurl.sty}{\usepackage{xurl}}{} % add URL line breaks if available
\urlstyle{same}
\hypersetup{
  pdflang={es-ES},
  hidelinks,
  pdfcreator={LaTeX via pandoc}}

\title{U8 DISTRIBUCIONS DE WINDOWS 11}
\usepackage{etoolbox}
\makeatletter
\providecommand{\subtitle}[1]{% add subtitle to \maketitle
  \apptocmd{\@title}{\par {\large #1 \par}}{}{}
}
\makeatother
\subtitle{Resum característiques}
\author{}
\date{\vspace{-2.5em}}

\begin{document}
\maketitle

\setstretch{1.5}
\newpage
\renewcommand\tablename{Tabla}

\section{1 Introducció}\label{introducciuxf3}

Windows 11 té diverses edicions o ``distros'' dissenyades per a
diferents tipus d'usuaris i necessitats. Aquí tens una explicació de les
principals:

\subsection{1. Windows 11 Home}\label{windows-11-home}

\begin{itemize}
\tightlist
\item
  Destinat a usuaris domèstics.
\item
  Inclou funcions bàsiques com Microsoft Edge, Microsoft Store, el mode
  fosc, i la compatibilitat amb aplicacions d'Android (a partir de
  Windows Subsystem for Android).
\item
  Té integració amb Microsoft Teams i la possibilitat d'iniciar sessió
  amb un compte de Microsoft.
\item
  No permet unir-se a dominis d'empresa ni té BitLocker nadiu.
\end{itemize}

\subsection{2. Windows 11 Pro}\label{windows-11-pro}

\begin{itemize}
\tightlist
\item
  Inclou totes les funcions de Windows 11 Home, però amb
  característiques addicionals per a professionals i empreses.
\item
  Suporta BitLocker (xifrat de disc), Hyper-V (virtualització), Windows
  Sandbox, i la unió a Active Directory i Azure AD.
\item
  Permet un millor control sobre les actualitzacions i més opcions de
  seguretat avançada.
\end{itemize}

\subsection{3. Windows 11 Pro for
Workstations}\label{windows-11-pro-for-workstations}

\begin{itemize}
\tightlist
\item
  Una versió millorada per a ordinadors d'alt rendiment.
\item
  Ofereix suport per a ReFS (Resilient File System), memòria persistent,
  i Windows Subsystem for Linux (WSL).
\item
  Optimitzat per a processadors d'alt rendiment i per a treballs
  exigents, com l'edició de vídeo o modelatge 3D.
\end{itemize}

\subsection{4. Windows 11 Enterprise}\label{windows-11-enterprise}

\begin{itemize}
\tightlist
\item
  Dissenyat per a grans empreses.
\item
  Té totes les funcions de Windows 11 Pro, però amb eines avançades per
  a gestió de seguretat i desplegament a gran escala.
\item
  Inclou Windows Defender Application Guard, protecció contra amenaces
  avançades i gestió remota millorada.
\end{itemize}

\subsection{5. Windows 11 Education}\label{windows-11-education}

\begin{itemize}
\tightlist
\item
  Similar a Windows 11 Enterprise, però adaptat a centres educatius.
\item
  Ofereix eines per a gestió de dispositius d'estudiants i seguretat
  millorada per a institucions educatives.
\item
  No inclou algunes funcions empresarials innecessàries en l'àmbit
  educatiu.
\end{itemize}

\subsection{6. Windows 11 SE}\label{windows-11-se}

\begin{itemize}
\tightlist
\item
  Versió més lleugera i simplificada per a dispositius educatius,
  especialment portàtils destinats a estudiants.
\item
  Optimitzat per a aplicacions en el núvol i integració amb Microsoft
  365.
\item
  Limita la instal·lació d'apps de tercers per millorar la seguretat i
  el rendiment.
\end{itemize}

\subsection{7. Windows 11 IoT (Internet of
Things)}\label{windows-11-iot-internet-of-things}

\begin{itemize}
\item
  Dissenyat per a dispositius especialitzats com caixers automàtics,
  dispositius mèdics i terminals de punt de venda.
\item
  Ofereix una versió més personalitzable i segura per a aplicacions
  industrials.
\end{itemize}

\newpage

\section{2 Quina versió triar?}\label{quina-versiuxf3-triar}

\begin{itemize}
\tightlist
\item
  Per a ús domèstic: Windows 11 Home.
\item
  Per a treball i empreses menudes: Windows 11 Pro.
\item
  Per a treballs exigents (disseny, edició, etc.): Windows 11 Pro for
  Workstations.
\item
  Per a empreses grans: Windows 11 Enterprise.
\item
  Per a educació: Windows 11 Education o Windows 11 SE.
\item
  Per a dispositius especialitzats: Windows 11 IoT.
\end{itemize}

\section{3 Taula resum}\label{taula-resum}

\begin{longtable}[]{@{}
  >{\raggedright\arraybackslash}p{(\columnwidth - 4\tabcolsep) * \real{0.1986}}
  >{\raggedright\arraybackslash}p{(\columnwidth - 4\tabcolsep) * \real{0.5106}}
  >{\raggedright\arraybackslash}p{(\columnwidth - 4\tabcolsep) * \real{0.2908}}@{}}
\toprule\noalign{}
\begin{minipage}[b]{\linewidth}\raggedright
Edició
\end{minipage} & \begin{minipage}[b]{\linewidth}\raggedright
Característiques principals
\end{minipage} & \begin{minipage}[b]{\linewidth}\raggedright
Destinat a\ldots{}
\end{minipage} \\
\midrule\noalign{}
\endhead
\bottomrule\noalign{}
\endlastfoot
\textbf{Windows 11 Home} & Funcions bàsiques, integració amb Microsoft
Teams, sense BitLocker & Usuaris domèstics \\
\textbf{Windows 11 Pro} & BitLocker, Hyper-V, Active Directory, gestió
empresarial & Professionals i xicotes empreses \\
\textbf{Windows 11 Pro for Workstations} & ReFS, optimitzat per a
treballs exigents, WSL & Usuaris avançats i treballs d'alt rendiment \\
\textbf{Windows 11 Enterprise} & Seguretat avançada, gestió d'equips,
protecció contra amenaces & Empreses grans \\
\textbf{Windows 11 Education} & Adaptat a institucions educatives, eines
per a estudiants i professors & Centres educatius \\
\textbf{Windows 11 SE} & Versió simplificada per a estudiants,
optimitzada per a núvol & Estudiants i escoles \\
\textbf{Windows 11 IoT} & Personalització per a dispositius industrials
i comercials & Dispositius especialitzats (caixers, terminals) \\
\end{longtable}

\end{document}
