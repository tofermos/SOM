% Options for packages loaded elsewhere
\PassOptionsToPackage{unicode}{hyperref}
\PassOptionsToPackage{hyphens}{url}
%
\documentclass[
  12 pt,
  a4paper,
]{article}
\usepackage{amsmath,amssymb}
\usepackage{setspace}
\usepackage{iftex}
\ifPDFTeX
  \usepackage[T1]{fontenc}
  \usepackage[utf8]{inputenc}
  \usepackage{textcomp} % provide euro and other symbols
\else % if luatex or xetex
  \usepackage{unicode-math} % this also loads fontspec
  \defaultfontfeatures{Scale=MatchLowercase}
  \defaultfontfeatures[\rmfamily]{Ligatures=TeX,Scale=1}
\fi
\usepackage{lmodern}
\ifPDFTeX\else
  % xetex/luatex font selection
  \setmainfont[]{Times New Roman}
\fi
% Use upquote if available, for straight quotes in verbatim environments
\IfFileExists{upquote.sty}{\usepackage{upquote}}{}
\IfFileExists{microtype.sty}{% use microtype if available
  \usepackage[]{microtype}
  \UseMicrotypeSet[protrusion]{basicmath} % disable protrusion for tt fonts
}{}
\makeatletter
\@ifundefined{KOMAClassName}{% if non-KOMA class
  \IfFileExists{parskip.sty}{%
    \usepackage{parskip}
  }{% else
    \setlength{\parindent}{0pt}
    \setlength{\parskip}{6pt plus 2pt minus 1pt}}
}{% if KOMA class
  \KOMAoptions{parskip=half}}
\makeatother
\usepackage{xcolor}
\usepackage[margin=1in]{geometry}
\usepackage{graphicx}
\makeatletter
\def\maxwidth{\ifdim\Gin@nat@width>\linewidth\linewidth\else\Gin@nat@width\fi}
\def\maxheight{\ifdim\Gin@nat@height>\textheight\textheight\else\Gin@nat@height\fi}
\makeatother
% Scale images if necessary, so that they will not overflow the page
% margins by default, and it is still possible to overwrite the defaults
% using explicit options in \includegraphics[width, height, ...]{}
\setkeys{Gin}{width=\maxwidth,height=\maxheight,keepaspectratio}
% Set default figure placement to htbp
\makeatletter
\def\fps@figure{htbp}
\makeatother
\setlength{\emergencystretch}{3em} % prevent overfull lines
\providecommand{\tightlist}{%
  \setlength{\itemsep}{0pt}\setlength{\parskip}{0pt}}
\setcounter{secnumdepth}{-\maxdimen} % remove section numbering
\ifLuaTeX
\usepackage[bidi=basic]{babel}
\else
\usepackage[bidi=default]{babel}
\fi
\babelprovide[main,import]{spanish}
\ifPDFTeX
\else
\babelfont{rm}[]{Times New Roman}
\fi
% get rid of language-specific shorthands (see #6817):
\let\LanguageShortHands\languageshorthands
\def\languageshorthands#1{}
\ifLuaTeX
  \usepackage{selnolig}  % disable illegal ligatures
\fi
\usepackage{bookmark}
\IfFileExists{xurl.sty}{\usepackage{xurl}}{} % add URL line breaks if available
\urlstyle{same}
\hypersetup{
  pdflang={es-ES},
  hidelinks,
  pdfcreator={LaTeX via pandoc}}

\title{U8 Evolució històrica del Windows i requisits 11}
\usepackage{etoolbox}
\makeatletter
\providecommand{\subtitle}[1]{% add subtitle to \maketitle
  \apptocmd{\@title}{\par {\large #1 \par}}{}{}
}
\makeatother
\subtitle{Evolución històrica de Windows 11 i les necessitats per
instal·lar}
\author{}
\date{\vspace{-2.5em}}

\begin{document}
\maketitle

{
\setcounter{tocdepth}{2}
\tableofcontents
}
\setstretch{1.5}
\newpage
\renewcommand\tablename{Tabla}

\section{1. Windows 11. Evolució històrica de
Windows}\label{windows-11.-evoluciuxf3-histuxf2rica-de-windows}

Windows 11 és la darrera versió del sistema operatiu de Microsoft,
llançat el 5 d'octubre de 2021. Ha evolucionat des dels seus
predecessors amb un nou disseny, millores en rendiment i una major
integració amb serveis en el núvol.

\subsection{Versions principals de Windows fins a Windows
11}\label{versions-principals-de-windows-fins-a-windows-11}

\begin{itemize}
\tightlist
\item
  Windows 95/98/ME: Primera interfície gràfica integrada amb el menú
  d'inici.\\
\item
  Windows XP (2001): Gran èxit pel seu rendiment i estabilitat.\\
\item
  Windows Vista (2006): Introducció de l'efecte Aero, però amb crítiques
  per requeriments alts.\\
\item
  Windows 7 (2009): Millores en rendiment i compatibilitat.\\
\item
  Windows 8/8.1 (2012-2013): Eliminació del menú d'inici, introducció de
  la interfície Metro.\\
\item
  Windows 10 (2015): Unificació de dispositius i suport continu amb
  actualitzacions periòdiques.\\
\item
  Windows 11 (2021): Nou disseny, millor compatibilitat amb aplicacions
  Android, millor optimització en jocs i enfocament en seguretat.
\end{itemize}

\section{2. Requisits del sistema
operatiu}\label{requisits-del-sistema-operatiu}

\subsection{Requisits mínims per instal·lar Windows
11}\label{requisits-muxednims-per-installar-windows-11}

\begin{itemize}
\tightlist
\item
  Processador: 1 GHz o més, amb 2 o més nuclis i arquitectura de 64
  bits.\\
\item
  Memòria RAM: 4 GB mínim.\\
\item
  Emmagatzematge: 64 GB d'espai lliure.\\
\item
  Firmware del sistema: UEFI amb Secure Boot activat.\\
\item
  TPM (Trusted Platform Module): Versió 2.0 obligatòria.\\
\item
  Targeta gràfica: Compatible amb DirectX 12 o superior.\\
\item
  Pantalla: Resolució mínima de 720p (9 polzades o més).
\end{itemize}

\subsection{Explicació dels components
clau}\label{explicaciuxf3-dels-components-clau}

\subsubsection{Firmware del sistema: UEFI amb Secure Boot
activat}\label{firmware-del-sistema-uefi-amb-secure-boot-activat}

UEFI (Unified Extensible Firmware Interface) és el sistema que gestiona
l'arrencada del sistema operatiu i reemplaça la BIOS tradicional.
Ofereix avantatges com arrencada més ràpida, suport per a discos grans
(més de 2 TB) i major seguretat. Secure Boot és una funcionalitat d'UEFI
que impedeix l'execució de codi maliciós en iniciar el sistema,
garantint que només s'executin sistemes operatius verificats.

\subsubsection{TPM (Trusted Platform Module) versió 2.0
obligatòria}\label{tpm-trusted-platform-module-versiuxf3-2.0-obligatuxf2ria}

El TPM és un mòdul de seguretat integrat en el maquinari que protegeix
dades sensibles mitjançant xifratge i claus criptogràfiques. Windows 11
requereix la versió 2.0 perquè millora la protecció contra atacs basats
en maquinari i facilita l'ús de tecnologies com BitLocker per xifrar
unitats de disc.

\subsubsection{Targeta gràfica compatible amb DirectX 12 o
superior}\label{targeta-gruxe0fica-compatible-amb-directx-12-o-superior}

DirectX 12 és un conjunt d'API que permet un millor rendiment gràfic en
aplicacions i videojocs. Una targeta gràfica compatible amb DirectX 12
assegura una experiència fluida en Windows 11, especialment en
aplicacions exigents com jocs, edició de vídeo o disseny 3D.

\end{document}
