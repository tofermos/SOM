% Options for packages loaded elsewhere
\PassOptionsToPackage{unicode}{hyperref}
\PassOptionsToPackage{hyphens}{url}
%
\documentclass[
  12 pt,
  a4paper,
]{article}
\usepackage{amsmath,amssymb}
\usepackage{setspace}
\usepackage{iftex}
\ifPDFTeX
  \usepackage[T1]{fontenc}
  \usepackage[utf8]{inputenc}
  \usepackage{textcomp} % provide euro and other symbols
\else % if luatex or xetex
  \usepackage{unicode-math} % this also loads fontspec
  \defaultfontfeatures{Scale=MatchLowercase}
  \defaultfontfeatures[\rmfamily]{Ligatures=TeX,Scale=1}
\fi
\usepackage{lmodern}
\ifPDFTeX\else
  % xetex/luatex font selection
  \setmainfont[]{Times New Roman}
\fi
% Use upquote if available, for straight quotes in verbatim environments
\IfFileExists{upquote.sty}{\usepackage{upquote}}{}
\IfFileExists{microtype.sty}{% use microtype if available
  \usepackage[]{microtype}
  \UseMicrotypeSet[protrusion]{basicmath} % disable protrusion for tt fonts
}{}
\makeatletter
\@ifundefined{KOMAClassName}{% if non-KOMA class
  \IfFileExists{parskip.sty}{%
    \usepackage{parskip}
  }{% else
    \setlength{\parindent}{0pt}
    \setlength{\parskip}{6pt plus 2pt minus 1pt}}
}{% if KOMA class
  \KOMAoptions{parskip=half}}
\makeatother
\usepackage{xcolor}
\usepackage[margin=1in]{geometry}
\usepackage{color}
\usepackage{fancyvrb}
\newcommand{\VerbBar}{|}
\newcommand{\VERB}{\Verb[commandchars=\\\{\}]}
\DefineVerbatimEnvironment{Highlighting}{Verbatim}{commandchars=\\\{\}}
% Add ',fontsize=\small' for more characters per line
\usepackage{framed}
\definecolor{shadecolor}{RGB}{248,248,248}
\newenvironment{Shaded}{\begin{snugshade}}{\end{snugshade}}
\newcommand{\AlertTok}[1]{\textcolor[rgb]{0.94,0.16,0.16}{#1}}
\newcommand{\AnnotationTok}[1]{\textcolor[rgb]{0.56,0.35,0.01}{\textbf{\textit{#1}}}}
\newcommand{\AttributeTok}[1]{\textcolor[rgb]{0.13,0.29,0.53}{#1}}
\newcommand{\BaseNTok}[1]{\textcolor[rgb]{0.00,0.00,0.81}{#1}}
\newcommand{\BuiltInTok}[1]{#1}
\newcommand{\CharTok}[1]{\textcolor[rgb]{0.31,0.60,0.02}{#1}}
\newcommand{\CommentTok}[1]{\textcolor[rgb]{0.56,0.35,0.01}{\textit{#1}}}
\newcommand{\CommentVarTok}[1]{\textcolor[rgb]{0.56,0.35,0.01}{\textbf{\textit{#1}}}}
\newcommand{\ConstantTok}[1]{\textcolor[rgb]{0.56,0.35,0.01}{#1}}
\newcommand{\ControlFlowTok}[1]{\textcolor[rgb]{0.13,0.29,0.53}{\textbf{#1}}}
\newcommand{\DataTypeTok}[1]{\textcolor[rgb]{0.13,0.29,0.53}{#1}}
\newcommand{\DecValTok}[1]{\textcolor[rgb]{0.00,0.00,0.81}{#1}}
\newcommand{\DocumentationTok}[1]{\textcolor[rgb]{0.56,0.35,0.01}{\textbf{\textit{#1}}}}
\newcommand{\ErrorTok}[1]{\textcolor[rgb]{0.64,0.00,0.00}{\textbf{#1}}}
\newcommand{\ExtensionTok}[1]{#1}
\newcommand{\FloatTok}[1]{\textcolor[rgb]{0.00,0.00,0.81}{#1}}
\newcommand{\FunctionTok}[1]{\textcolor[rgb]{0.13,0.29,0.53}{\textbf{#1}}}
\newcommand{\ImportTok}[1]{#1}
\newcommand{\InformationTok}[1]{\textcolor[rgb]{0.56,0.35,0.01}{\textbf{\textit{#1}}}}
\newcommand{\KeywordTok}[1]{\textcolor[rgb]{0.13,0.29,0.53}{\textbf{#1}}}
\newcommand{\NormalTok}[1]{#1}
\newcommand{\OperatorTok}[1]{\textcolor[rgb]{0.81,0.36,0.00}{\textbf{#1}}}
\newcommand{\OtherTok}[1]{\textcolor[rgb]{0.56,0.35,0.01}{#1}}
\newcommand{\PreprocessorTok}[1]{\textcolor[rgb]{0.56,0.35,0.01}{\textit{#1}}}
\newcommand{\RegionMarkerTok}[1]{#1}
\newcommand{\SpecialCharTok}[1]{\textcolor[rgb]{0.81,0.36,0.00}{\textbf{#1}}}
\newcommand{\SpecialStringTok}[1]{\textcolor[rgb]{0.31,0.60,0.02}{#1}}
\newcommand{\StringTok}[1]{\textcolor[rgb]{0.31,0.60,0.02}{#1}}
\newcommand{\VariableTok}[1]{\textcolor[rgb]{0.00,0.00,0.00}{#1}}
\newcommand{\VerbatimStringTok}[1]{\textcolor[rgb]{0.31,0.60,0.02}{#1}}
\newcommand{\WarningTok}[1]{\textcolor[rgb]{0.56,0.35,0.01}{\textbf{\textit{#1}}}}
\usepackage{graphicx}
\makeatletter
\def\maxwidth{\ifdim\Gin@nat@width>\linewidth\linewidth\else\Gin@nat@width\fi}
\def\maxheight{\ifdim\Gin@nat@height>\textheight\textheight\else\Gin@nat@height\fi}
\makeatother
% Scale images if necessary, so that they will not overflow the page
% margins by default, and it is still possible to overwrite the defaults
% using explicit options in \includegraphics[width, height, ...]{}
\setkeys{Gin}{width=\maxwidth,height=\maxheight,keepaspectratio}
% Set default figure placement to htbp
\makeatletter
\def\fps@figure{htbp}
\makeatother
\setlength{\emergencystretch}{3em} % prevent overfull lines
\providecommand{\tightlist}{%
  \setlength{\itemsep}{0pt}\setlength{\parskip}{0pt}}
\setcounter{secnumdepth}{-\maxdimen} % remove section numbering
\ifLuaTeX
\usepackage[bidi=basic]{babel}
\else
\usepackage[bidi=default]{babel}
\fi
\babelprovide[main,import]{spanish}
\ifPDFTeX
\else
\babelfont{rm}[]{Times New Roman}
\fi
% get rid of language-specific shorthands (see #6817):
\let\LanguageShortHands\languageshorthands
\def\languageshorthands#1{}
\ifLuaTeX
  \usepackage{selnolig}  % disable illegal ligatures
\fi
\usepackage{bookmark}
\IfFileExists{xurl.sty}{\usepackage{xurl}}{} % add URL line breaks if available
\urlstyle{same}
\hypersetup{
  pdftitle={U4. Sistema de fitxers. Ubuntu (III)},
  pdfauthor={@tofermos 2024},
  pdflang={es-ES},
  hidelinks,
  pdfcreator={LaTeX via pandoc}}

\title{U4. Sistema de fitxers. Ubuntu (III)}
\usepackage{etoolbox}
\makeatletter
\providecommand{\subtitle}[1]{% add subtitle to \maketitle
  \apptocmd{\@title}{\par {\large #1 \par}}{}{}
}
\makeatother
\subtitle{TRACTAMENT DE TEXT (I)}
\author{@tofermos 2024}
\date{}

\begin{document}
\maketitle

{
\setcounter{tocdepth}{2}
\tableofcontents
}
\setstretch{1.5}
\newpage
\renewcommand\tablename{Tabla}

\section{Resum}\label{resum}

Tot seguit coneixerem unes ordres de Linux que ens serveixen per al
tractament de fitxers de text pla. En el món de Linux són molt útils a
l'hora de consultar o fer modificacions en fitxers de configuració del
sistema i també per consultar altres fitxers de registres sobre estats o
l'exida a pantalla que donen les ordres de Linux.

\section{\texorpdfstring{1. \texttt{grep}}{1. grep}}\label{grep}

L'ordre \texttt{grep} s'utilitza per buscar text dins de fitxers. Mostra
línies que coincideixen amb un patró específic.

\subsection{Modificadors essencials:}\label{modificadors-essencials}

\begin{itemize}
\tightlist
\item
  \texttt{-i}: ignora majúscules i minúscules.
\item
  \texttt{-v}: mostra línies que \textbf{no} coincideixen amb el patró.
\item
  \texttt{-r}: busca de manera recursiva dins de directoris.
\item
  \texttt{-n}: mostra el número de línia on es troba la coincidència.
\end{itemize}

\subsection{Exemple:}\label{exemple}

\begin{Shaded}
\begin{Highlighting}[]
\CommentTok{\# Busca "error" en el fitxer log.txt, ignorant majúscules i minúscules}
\FunctionTok{grep} \AttributeTok{{-}i} \StringTok{"error"}\NormalTok{ log.txt}

\CommentTok{\# Busca "warning" recursivament dins el directori logs i mostra el número de línia}
\FunctionTok{grep} \AttributeTok{{-}r} \AttributeTok{{-}n} \StringTok{"warning"}\NormalTok{ logs/}

\CommentTok{\# Busca línies que comencen amb "Start" en el fitxer text.txt}
\FunctionTok{grep} \StringTok{"\^{}Start"}\NormalTok{ text.txt}

\CommentTok{\# Busca línies que acaben amb "End" en el fitxer text.txt}
\FunctionTok{grep} \StringTok{"End$"}\NormalTok{ text.txt}
\end{Highlighting}
\end{Shaded}

\begin{center}\rule{0.5\linewidth}{0.5pt}\end{center}

\section{\texorpdfstring{2. \texttt{tr}}{2. tr}}\label{tr}

L'ordre \texttt{tr} s'utilitza per traduir o suprimir caràcters.
Generalment es fa servir per convertir majúscules a minúscules, eliminar
espais, etc.

\subsection{Modificadors essencials:}\label{modificadors-essencials-1}

\begin{itemize}
\tightlist
\item
  \texttt{-d}: suprimeix els caràcters especificats.
\item
  \texttt{-s}: substitueix seqüències repetides del mateix caràcter amb
  una sola instància.
\end{itemize}

\subsection{Exemple:}\label{exemple-1}

\begin{Shaded}
\begin{Highlighting}[]
\CommentTok{\# Converteix majúscules a minúscules}
\BuiltInTok{echo} \StringTok{"HELLO WORLD"} \KeywordTok{|} \FunctionTok{tr} \StringTok{\textquotesingle{}A{-}Z\textquotesingle{}} \StringTok{\textquotesingle{}a{-}z\textquotesingle{}}

\CommentTok{\# Suprimeix tots els espais}
\BuiltInTok{echo} \StringTok{"Hello   World"} \KeywordTok{|} \FunctionTok{tr} \AttributeTok{{-}d} \StringTok{\textquotesingle{} \textquotesingle{}}

\CommentTok{\# Substitueix seqüències repetides d\textquotesingle{}espais amb un sol espai}
\BuiltInTok{echo} \StringTok{"Hello      World"} \KeywordTok{|} \FunctionTok{tr} \AttributeTok{{-}s} \StringTok{\textquotesingle{} \textquotesingle{}}
\end{Highlighting}
\end{Shaded}

\begin{center}\rule{0.5\linewidth}{0.5pt}\end{center}

\section{\texorpdfstring{3. \texttt{cut}}{3. cut}}\label{cut}

L'ordre \texttt{cut} s'utilitza per seleccionar parts de cada línia d'un
fitxer o flux d'entrada (com ara una columna específica d'un CSV).

\subsection{Modificadors essencials:}\label{modificadors-essencials-2}

\begin{itemize}
\tightlist
\item
  \texttt{-d}: defineix el delimitador (per defecte és el tabulador).
\item
  \texttt{-f}: selecciona els camps específics.
\end{itemize}

\subsection{Exemple:}\label{exemple-2}

\begin{Shaded}
\begin{Highlighting}[]
\CommentTok{\# Mostra la primera columna (assumint que és un CSV)}
\FunctionTok{cat}\NormalTok{ dades.csv }\KeywordTok{|} \FunctionTok{cut} \AttributeTok{{-}d} \StringTok{\textquotesingle{},\textquotesingle{}} \AttributeTok{{-}f}\NormalTok{ 1}

\CommentTok{\# Mostra la segona i tercera columna separades per espais}
\BuiltInTok{echo} \StringTok{"nom edat ciutat"} \KeywordTok{|} \FunctionTok{cut} \AttributeTok{{-}d} \StringTok{\textquotesingle{} \textquotesingle{}} \AttributeTok{{-}f}\NormalTok{ 2,3}
\end{Highlighting}
\end{Shaded}

\begin{center}\rule{0.5\linewidth}{0.5pt}\end{center}

\section{\texorpdfstring{4. \texttt{sort}}{4. sort}}\label{sort}

L'ordre \texttt{sort} s'utilitza per ordenar les línies d'un fitxer o
entrada. Es pot utilitzar amb diversos criteris d'ordenació.

\subsection{Modificadors essencials:}\label{modificadors-essencials-3}

\begin{itemize}
\tightlist
\item
  \texttt{-r}: ordena en ordre invers (descendent).
\item
  \texttt{-n}: ordena numèricament en lloc d'alfabèticament.
\item
  \texttt{-k}: especifica una columna per ordenar.
\end{itemize}

\subsection{Exemple:}\label{exemple-3}

\begin{Shaded}
\begin{Highlighting}[]
\CommentTok{\# Ordena el fitxer numèricament (de menor a major)}
\FunctionTok{sort} \AttributeTok{{-}n}\NormalTok{ nombres.txt}

\CommentTok{\# Ordena el fitxer en ordre descendent}
\FunctionTok{sort} \AttributeTok{{-}r}\NormalTok{ paraules.txt}

\CommentTok{\# Ordena el fitxer per la segona columna}
\FunctionTok{sort} \AttributeTok{{-}k}\NormalTok{ 2 dades.txt}
\end{Highlighting}
\end{Shaded}

\section{\texorpdfstring{5 \texttt{wc}}{5 wc}}\label{wc}

L'ordre \texttt{wc} (``word count'' o comptador de paraules) serveix per
\textbf{comptar línies, paraules i caràcters (bytes)} en un fitxer o a
l'eixida d'una altra ordre. Té diversos modificadors que permeten
filtrar quina informació es vol veure. Aquestes opcions són:

\begin{itemize}
\tightlist
\item
  \texttt{-l}: Comptar el nombre de línies.
\item
  \texttt{-w}: Comptar el nombre de paraules.
\item
  \texttt{-c}: Comptar el nombre de caràcters (bytes).
\item
  \texttt{-m}: Comptar el nombre de caràcters (tenint en compte
  caràcters multibyte com accents i símbols).
\item
  \texttt{-L}: Mostrar la línia més llarga en nombre de caràcters.
\end{itemize}

\subsection{\texorpdfstring{Exemples d'ús de
\texttt{wc}}{Exemples d'ús de wc}}\label{exemples-duxfas-de-wc}

\begin{enumerate}
\def\labelenumi{\arabic{enumi}.}
\item
  \textbf{Comptar línies en un fitxer}

\begin{Shaded}
\begin{Highlighting}[]
\FunctionTok{wc} \AttributeTok{{-}l}\NormalTok{ fitxer.txt}
\end{Highlighting}
\end{Shaded}

  Això mostrarà el nombre de línies en el fitxer \texttt{fitxer.txt}.
\item
  \textbf{Comptar paraules en un fitxer}

\begin{Shaded}
\begin{Highlighting}[]
\FunctionTok{wc} \AttributeTok{{-}w}\NormalTok{ fitxer.txt}
\end{Highlighting}
\end{Shaded}
\end{enumerate}

Això mostrarà el nombre de paraules en el fitxer \texttt{fitxer.txt}.

\begin{enumerate}
\def\labelenumi{\arabic{enumi}.}
\setcounter{enumi}{2}
\item
  \textbf{Comptar caràcters en un fitxer}

\begin{Shaded}
\begin{Highlighting}[]
\FunctionTok{wc} \AttributeTok{{-}c}\NormalTok{ fitxer.txt}
\end{Highlighting}
\end{Shaded}
\end{enumerate}

Això mostrarà el nombre de caràcters (bytes) en el fitxer
\texttt{fitxer.txt}.

\begin{enumerate}
\def\labelenumi{\arabic{enumi}.}
\setcounter{enumi}{3}
\item
  \textbf{Comptar caràcters multibyte en un fitxer}

\begin{Shaded}
\begin{Highlighting}[]
\FunctionTok{wc} \AttributeTok{{-}m}\NormalTok{ fitxer.txt}
\end{Highlighting}
\end{Shaded}
\end{enumerate}

Aquest comptarà tots els caràcters, tenint en compte els multibyte com
accents o símbols especials.

\begin{enumerate}
\def\labelenumi{\arabic{enumi}.}
\setcounter{enumi}{4}
\item
  \textbf{Mostrar la línia més llarga en nombre de caràcters}

\begin{Shaded}
\begin{Highlighting}[]
\FunctionTok{wc} \AttributeTok{{-}L}\NormalTok{ fitxer.txt}
\end{Highlighting}
\end{Shaded}
\end{enumerate}

Aquest comptarà els caràcters de la línia més llarga del fitxer
\texttt{fitxer.txt}.

\begin{enumerate}
\def\labelenumi{\arabic{enumi}.}
\setcounter{enumi}{5}
\item
  \textbf{Mostrar totes les estadístiques a la vegada}

\begin{Shaded}
\begin{Highlighting}[]
\FunctionTok{wc}\NormalTok{ fitxer.txt}
\end{Highlighting}
\end{Shaded}
\end{enumerate}

Això mostrarà el nombre de línies, paraules i bytes del fitxer
\texttt{fitxer.txt} en un sol ordre.

\subsection{\texorpdfstring{Ús de \texttt{wc} amb combinacions de pipes
(\texttt{\textbar{}})}{Ús de wc amb combinacions de pipes (\textbar)}}\label{uxfas-de-wc-amb-combinacions-de-pipes}

L'ús de \texttt{wc} combinat amb altres ordres \texttt{cat} o
\texttt{grep} és prou habitual per filtrar o comptar el nombre
d'ocurrències d'un patró.

\begin{enumerate}
\def\labelenumi{\arabic{enumi}.}
\item
  \textbf{Comptar el nombre de línies que contenen una paraula
  específica amb \texttt{grep} i \texttt{wc}}

\begin{Shaded}
\begin{Highlighting}[]
\FunctionTok{grep} \StringTok{"paraula"}\NormalTok{ fitxer.txt }\KeywordTok{|} \FunctionTok{wc} \AttributeTok{{-}l}
\end{Highlighting}
\end{Shaded}

  Aquesta ordre cerca la paraula ``paraula'' en el fitxer
  \texttt{fitxer.txt} i després compta el nombre de línies que contenen
  aquesta paraula.
\item
  \textbf{Comptar el nombre de paraules en diversos fitxers amb
  \texttt{cat} i \texttt{wc}}

\begin{Shaded}
\begin{Highlighting}[]
\FunctionTok{cat}\NormalTok{ fitxer1.txt fitxer2.txt }\KeywordTok{|} \FunctionTok{wc} \AttributeTok{{-}w}
\end{Highlighting}
\end{Shaded}

  Aquest combinat mostra el nombre total de paraules en els fitxers
  \texttt{fitxer1.txt} i \texttt{fitxer2.txt} sumades.
\item
  \textbf{Comptar línies d'un fitxer excloent línies buides amb
  \texttt{grep} i \texttt{wc}}

\begin{Shaded}
\begin{Highlighting}[]
\FunctionTok{grep} \AttributeTok{{-}v} \StringTok{"\^{}$"}\NormalTok{ fitxer.txt }\KeywordTok{|} \FunctionTok{wc} \AttributeTok{{-}l}
\end{Highlighting}
\end{Shaded}

  Aquesta ordre cerca totes les línies no buides (\texttt{\^{}\$}
  significa línies buides) en \texttt{fitxer.txt} i compta el nombre de
  línies no buides trobades.
\end{enumerate}

Aquests exemples mostren com utilitzar \texttt{wc} de diferents maneres
per analitzar fitxers i també com combinar-lo amb altres ordres
mitjançant pipelines per obtenir informació més específica i detallada.

\begin{enumerate}
\def\labelenumi{\arabic{enumi}.}
\setcounter{enumi}{3}
\tightlist
\item
  \textbf{opció -c del grep}
\end{enumerate}

Com a curiositat podem veure l'opció -c del grep.

\begin{Shaded}
\begin{Highlighting}[]
\ExtensionTok{tomas@portatil:\textasciitilde{}$}\NormalTok{ grep \^{}tomas /etc/passwd}
\ExtensionTok{tomas:x:1000:1000:tomas,,,:/home/tomas:/bin/bash}
\ExtensionTok{tomas@portatil:\textasciitilde{}$}\NormalTok{ grep \^{}tomas /etc/passwd }\AttributeTok{{-}c}
\ExtensionTok{1}
\ExtensionTok{tomas@portatil:\textasciitilde{}$}\NormalTok{ grep \^{}tomas /etc/passwd }\KeywordTok{|}\FunctionTok{wc} \AttributeTok{{-}l}
\ExtensionTok{1}
\end{Highlighting}
\end{Shaded}


\end{document}
